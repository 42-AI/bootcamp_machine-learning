\chapter{Exercise 01}
%******************************************************************************%
%                                                                              %
%                                 Interlude                                    %
%                         for Machine Learning module                          %
%                                                                              %
%******************************************************************************%

% =============================== %
\section*{Linear Algebra Tricks part II}
% ******************************* %

If you tried to run your code on a very large dataset, you would find that it sometimes takes a (very) long time to execute!
That's because it doesn't use the power of Python libraries that are optimized for matrix operations.\\
\newline
Remember the linear algebra trick from the previous module? Let's use it again!  
If you concatenate a column of $1$'s to the left of the $x$ vector, you get what we called matrix $X'$.   
$$
X' = \begin{bmatrix} 1 & x^{(1)} \\ \vdots & \vdots \\ 1 & x^{(m)}\end{bmatrix}
$$
This transformation is very convenient because we can rewrite each $1$ as $x_0^{(i)}$, and each $x^{(i)}$ as $x_1^{(i)}$.
So now the $X'$ matrix looks like this:
$$
X' = \begin{bmatrix} x_0^{(1)} & x_1^{(1)} \\ \vdots & \vdots \\ x_0^{(m)} & x_1^{(m)}\end{bmatrix}
$$
Notice that each $x^{(i)}$ example becomes a vector made of $(x^{(i)}_0, x^{(i)}_1)$.  
The $0$ and $1$ indices on the $x$ features correspond to the indices of the $\theta$ parameters with which they will be multiplied.\\
\newline
Why does this matter?
Well, if we take the equation from the previous exercise:

$$
\nabla(J)_0 = \frac{1}{m}\sum_{i=1}^{m}(h_{\theta}(x^{(i)}) - y^{(i)})
$$
We can multiply it by $1$ without changing its value:
$$
\nabla(J)_0 = \frac{1}{m}\sum_{i=1}^{m}(h_{\theta}(x^{(i)}) - y^{(i)}) \cdot 1
$$
And rewrite $1$ as $x_0^{(i)}$:
$$
\nabla(J)_0 = \frac{1}{m}\sum_{i=1}^{m}(h_{\theta}(x^{(i)}) - y^{(i)})x_{0}^{(i)}
$$
This means that the equation for $\nabla(J)_0$ is now similar to the equation we had for $\nabla(J)_1$, so they can both be captured by ONE \textbf{generic equation}:
$$
\begin{matrix}
\nabla(J)_j = \frac{1}{m}\sum_{i=1}^{m}(h_{\theta}(x^{(i)}) - y^{(i)})x_{j}^{(i)} & & \text{ for j = 0, 1}    
\end{matrix}
$$
And as you probably suspected, a generic equation opens the door to vectorization...

% =============================== %
\subsection*{Vectorizing the Gradient Calculation}
% ******************************* %
Now it's time to learn how to calculate the entire gradient in one short, pretty, linear algebra equation!  
\begin{itemize}
    \item First, we'll use the $X'$ matrix and our vectorized hypothesis equation $h_{\theta}(x)=X'\theta$
    $$
    \begin{matrix}
    \nabla(J)_j = \frac{1}{m} (X'\theta - y)X'_{j} & & \text{ for j = 0, 1}
    \end{matrix}
    $$
    
    \item Second, we need to tweak the equation a bit so that it directly returns a $\nabla(J)$ vector containing both $\nabla(J)_0$ and $\nabla(J)_1$.
    
    $$
    \nabla(J) = \frac{1}{m} {X'}^T(X'\theta - y)    
    $$
\end{itemize}
If the equation does not seems obvious, play a bit with your vectors, on paper and in your code, until you get it.\\

% =============================== %
\subsubsection*{Notation Remark}
% ******************************* %
${X'}^T$: You might wonder what the $^T$ is for.
It means the $X'$ matrix must be \textbf{transposed}.\\
\newline
Transposing a matrix flips it on its diagonal so that its rows become its columns and \textit{vice-versa}.
Here we need to make sure that matrix dimensions are appropriate and allow for multiplication, and to multiply the right items together.
\newpage
\extitle{Multivariate hypothesis - vectorized version}
\turnindir{ex01}
\exnumber{01}
\exfiles{prediction.py}
\exforbidden{None}
\makeheaderfilesforbidden

% ================================= %
\section*{Objective}
% --------------------------------- %
Manipulate the hypothesis to make a prediction.\\
\newline
You must implement the following formula as a function:  

$$
\hat{y} = X' \cdot \theta = 
\begin{bmatrix} 
1 & x_{1}^{(1)} & \dots & x_{n}^{(1)}\\
\vdots & \vdots & \ddots & \vdots\\
1 & x_{1}^{(m)} & \dots &  x_{n}^{(m)}\end{bmatrix}
\cdot
\begin{bmatrix}
\theta_0 \\ 
\theta_1 \\
\vdots \\
\theta_n
\end{bmatrix} 
= 
\begin{bmatrix} 
\theta_0 + \theta_{1} x_{1}^{(1)} + \dots + \theta_{n} x_{n}^{(1)}\\ 
\vdots \\ 
\theta_0 + \theta_{1} x_{1}^{(m)} + \dots + \theta_{n} x_{n}^{(m)}
\end{bmatrix}
$$
Where:
\begin{itemize}
  \item $\hat{y}$ is a vector of dimension $m$: the vector of predicted values
  \item $X$ is a matrix of dimensions $(m \times n)$: the design matrix
  \item $X'$ is a matrix of dimensions $(m \times (n + 1))$: the design matrix onto which a column of $1$'s was added as a first column
  \item $\theta$ is a vector of dimension $(n + 1)$: the parameter vector
  \item $x^{(i)}$ is the i$^\text{th}$ row of the $X$ matrix
  \item $x_{j}$ is the j$^\text{th}$ column of the $X$ matrix
  \item $x_j^{(i)}$ is the intersection of the i$^\text{th}$ row and the j$^\text{th}$ column of the $X$ matrix: the j$^\text{th}$ feature of the i$^\text{th}$ training example.
\end{itemize}
Be careful: 
\begin{itemize}
  \item The \texttt{x} argument your function will receive as an input corresponds to $X$, the $(m \times n)$ matrix.
        Not $X'$.
  \item \texttt{theta} is an $(n + 1)$ vector.
  \item You have to transform \texttt{x} to fit \texttt{theta}'s dimensions
\end{itemize}
\newpage
% ================================= %
\section*{Instructions}
% --------------------------------- %
In the \texttt{prediction.py} file, write the \texttt{predict\_} function as per the instructions given below:

\begin{minted}[bgcolor=darcula-back,formatcom=\color{lightgrey},fontsize=\scriptsize]{python}
def predict_(x, theta):
    """Computes the prediction vector y_hat from two non-empty numpy.array.
    Args:
      x: has to be an numpy.array, a vector of dimensions m * n.
      theta: has to be an numpy.array, a vector of dimensions (n + 1) * 1.
    Return:
      y_hat as a numpy.array, a vector of dimensions m * 1.
      None if x or theta are empty numpy.array.
      None if x or theta dimensions are not appropriate.
      None if x or theta is not of expected type.
    Raises:
      This function should not raise any Exception.
    """
    ... Your code ...
\end{minted}

% ================================= %
\section*{Examples}
% --------------------------------- %

\begin{minted}[bgcolor=darcula-back,formatcom=\color{lightgrey},fontsize=\scriptsize]{python}
import numpy as np
x = np.arange(1,13).reshape((4,-1))

# Example 1:
theta1 = np.array([5, 0, 0, 0]).reshape((-1, 1))
predict_(x, theta1)
# Ouput:
array([[5.], [5.], [5.], [5.]])
# Do you understand why y_hat contains only 5's here?  

# Example 2:
theta2 = np.array([0, 1, 0, 0]).reshape((-1, 1))
predict_(x, theta2)
# Output:
array([[ 1.], [ 4.], [ 7.], [10.]])
# Do you understand why y_hat == x[:,0] here?  


# Example 3:
theta3 = np.array([-1.5, 0.6, 2.3, 1.98]).reshape((-1, 1))
predict_(x, theta3)
# Output:
array([[ 9.64], [24.28], [38.92], [53.56]])


# Example 4:
theta4 = np.array([-3, 1, 2, 3.5]).reshape((-1, 1))
predict_(x, theta4)
# Output:
array([[12.5], [32. ], [51.5], [71. ]])
\end{minted}