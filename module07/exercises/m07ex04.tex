\chapter{Exercise 04}
\extitle{Multivariate Gradient Descent}
%******************************************************************************%
%                                                                              %
%                                 Interlude                                    %
%                         for Machine Learning module                          %
%                                                                              %
%******************************************************************************%

% =============================================== %
\section*{Interlude}
% =============================================== %
\subsection*{Improve}
% ----------------------------------------------- %

\begin{figure}[!h]
    \centering
    \includegraphics[scale=0.25]{assets/Improve.png}
    %\caption{The Learning Cycle: Improve}
\end{figure}
\noindent{Now we want to improve the algorithm's 
performance, or in other words, reduce the loss of its predictions.}\\
\\
This brings us (again) to calculating the gradient, which will tell us by
how much and in which direction the theta parameters belonging to the model should be adjusted.

\newpage
% =============================================== %
\subsection*{The logistic gradient}
% ----------------------------------------------- %
If you remember, to calculate the gradient, we start with the loss function and we derive it 
with respect to each of the theta parameters.\\
\\
If you know multivariate calculus already, you can try it for yourself, otherwise we've got you covered:\\

$$
\begin{matrix}
\nabla(J)_0 &  = &\cfrac{1}{m}\sum_{i=1}^{m}(h_{\theta}(x^{(i)}) - y^{(i)}) & \\
\nabla(J)_j & = &\cfrac{1}{m}\sum_{i=1}^{m}(h_{\theta}(x^{(i)}) - y^{(i)})x_{j}^{(i)} & \text{ for j = 1, ..., n}    
\end{matrix}
$$
Where:
\begin{itemize}
    \item $\nabla(J)$ is a vector of dimension $(n + 1)$, the gradient vector
    \item $\nabla(J)_j$ is the j$^\text{th}$ component of $\nabla(J)$, 
    the partial derivative of $J$ with respect to $\theta_j$
    \item $y$ is a vector of dimension $m$, the vector of expected values
    \item $y^{(i)}$ is a scalar, the i$^\text{th}$ component of vector $y$
    \item $x^{(i)}$ is the feature vector of the i$^\text{th}$ example
    \item $x^{(i)}_j$ is a scalar, the j$^\text{th}$ feature value of the i$^\text{th}$ example
    \item $h_{\theta}(x^{(i)})$ is a scalar, the model's estimation of $y^{(i)}$\\
\end{itemize}
This formula should be very familiar to you, as it's the same one you used to calculate the linear regression gradient!\\
\\
The only difference is that $h_{\theta}(x^{(i)})$ corresponds to \textbf{the logistic regression hypothesis instead of the linear regression hypothesis}.\\
\\
In other words:\\
$$
h_{\theta}(x^{(i)}) = \text{sigmoid}( \theta \cdot x'^{(i)}) = \cfrac{1} {1 + e^{-\theta \cdot x'^{(i)}}}
$$
\\
Instead of:
\\
$$
\cancel{h_{\theta}(x^{(i)}) = \theta \cdot x'^{(i)}}
$$

\newpage
\turnindir{ex04}
\exnumber{04}
\exfiles{fit.py}
\exforbidden{any function that performs derivatives for you}
\makeheaderfilesforbidden


% ================================= %
\section*{Objective}
% --------------------------------- %
Understand and manipulate the concept of gradient descent in the case of multivariate linear regression.\\
\newline
Implement a function to perform linear gradient descent (LGD) for multivariate linear regression.

% ================================= %
\section*{Instructions}
% --------------------------------- %
In this exercise, you will implement linear gradient descent to fit your multivariate model to the dataset.\\
\newline
The pseudocode of the algorithm is the following:

$$
\begin{matrix}
    &   \text{repeat until convergence} \hspace{1cm} &  \{  \\
    &   \text{compute } \nabla{(J)}  \\
    &	\theta := \theta - \alpha \nabla(J)                 \\ 
\} 
\end{matrix}
$$
Where:
\begin{itemize}
  \item $\nabla{(J)}$ is the entire gradient vector
  \item $\theta$ is the entire parameter vector
  \item $\alpha$ (alpha) is the learning rate (a small number, usually between 0 and 1)
\end{itemize}
% ================================= %
\newpage
\noindent{You are expected to write a function named \textit{fit\_} as per the instructions bellow:\\}
\newline
\begin{minted}[bgcolor=darcula-back,formatcom=\color{lightgrey},fontsize=\scriptsize]{python}
def fit_(x, y, theta, alpha, max_iter):
  """
  Description:
    Fits the model to the training dataset contained in x and y.
  Args:
    x: has to be a numpy.array, a matrix of dimension m * n:
                  (number of training examples, number of features).
    y: has to be a numpy.array, a vector of dimension m * 1:
                  (number of training examples, 1).
    theta: has to be a numpy.array, a vector of dimension (n + 1) * 1:
                  (number of features + 1, 1).
    alpha: has to be a float, the learning rate
    max_iter: has to be an int, the number of iterations done during the gradient descent
  Return:
    new_theta: numpy.array, a vector of dimension (number of features + 1, 1).
    None if there is a matching dimension problem.
    None if x, y, theta, alpha or max_iter is not of expected type.
  Raises:
    This function should not raise any Exception.
  """
    ... your code here ...
\end{minted}
\\
Hopefully, you have already implemented a function to calculate the multivariate gradient.
% ================================= %
\section*{Examples}
% ================================= %
\begin{minted}[bgcolor=darcula-back,formatcom=\color{lightgrey},fontsize=\scriptsize]{python}
import numpy as np
x = np.array([[0.2, 2., 20.], [0.4, 4., 40.], [0.6, 6., 60.], [0.8, 8., 80.]])
y = np.array([[19.6], [-2.8], [-25.2], [-47.6]])
theta = np.array([[42.], [1.], [1.], [1.]])

# Example 0:
theta2 = fit_(x, y, theta,  alpha = 0.0005, max_iter=42000)
theta2
# Output:
array([[41.99..],[0.97..], [0.77..], [-1.20..]])

# Example 1:
predict_(x, theta2)
# Output:
array([[19.5992..], [-2.8003..], [-25.1999..], [-47.5996..]])
\end{minted}

\info{
  \begin{itemize}
    \item You can create more training data by generating an $x$ array with random values and computing the corresponding $y$ vector as a linear expression of $x$.
          You can then fit a model on this artificial data and find out if it comes out with the same $\theta$ coefficients that first you used.
    \item It is possible that $\theta_0$ and $\theta_1$ become \texttt{"nan"}.
          In that case, it means you probably used a learning rate that is too large.
  \end{itemize}
}