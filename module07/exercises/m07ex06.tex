\chapter{Exercise 06}
\extitle{Practicing Multivariate Linear Regression}
\turnindir{ex06}
\exnumber{06}
\exfiles{multivariate\_linear\_model.py}
\exforbidden{sklearn}
\makeheaderfilesforbidden

% ================================= %
\section*{Objective}
% --------------------------------- %
Fit a linear regression model to a dataset with multiple features.\\
\newline
Plot the model's predictions and interpret the graphs.

% ================================= %
\section*{Dataset}
% --------------------------------- %
In the last module, you performed a univariate linear regression on a dataset to make predictions based on ONE feature (well done!).
Now, it's time to dream bigger.\\
\newline
Lucky you, you will a new dataset with multiple features find in the resources attached.\\
\\
The dataset is called \texttt{spacecraft\_data.csv} and it describes a set of spacecrafts with their prices, as well as a few other features.
A description of the dataset is provided in the file named \texttt{spacecraft\_data\_description.txt}.
\newpage
% ================================= %
\subsection*{Part One: Univariate Linear Regression}
% --------------------------------- %
To start, we'll build on the previous module's work and see how a univariate model can predict spaceship prices.
As you know, univariate models can only process ONE feature at a time.
So to train each model, you need to select a feature and ignore the other ones.
% ================================= %
\subsubsection*{Instructions}
% --------------------------------- %
In the first part of the exercise, you will train three different univariate models to predict spaceship prices.
Each model will use a different feature of the spaceships.
For each feature, your program has to perform a gradient descent from a new set of thetas, plot or generate a plot,
print the final value of the thetas and the MSE of the corresponding model.

% ================================= %
\subsubsection*{Age}
% --------------------------------- %
Select the \texttt{Age} feature as your $x$ vector, and \texttt{Sell\_price} as your $y$ vector.
Train a first model, \texttt{myLR\_age}, and generate price predictions ($\hat{y}$).
Output a scatter plot with both sets of data points on the same graph, as follows:
\begin{itemize}
  \item The actual prices, given by $(x_{age}^{(i)}$, $y^{(i)})$  for $i=0....m$
  \item The predicted prices, represented by  $(x_{age}^{(i)}$, $\hat{y}^{(i)})$  for $i=0....m$ (see example below)
\end{itemize}

\begin{figure}[!h]
  \centering
  \includegraphics[scale=0.6]{assets/ex07_price_vs_age_part1.png}
  \caption{Plot of the selling prices of spacecrafts with respect to their age, as well as our first model's price predictions.}
\end{figure}

% ================================= %
\subsubsection*{Thrust}
% --------------------------------- %
Select the \textit{Thrust\_power} feature as your $x$ vector, and \textit{Sell\_price} as your $y$ vector.
Train a second model, \texttt{myLR\_thrust}, and generate price predictions ($\hat{y}$).
Output a scatter plot with both sets of data points on the same graph, as follows:
\begin{itemize}
  \item The actual prices, given by $(x_{thrust}^{(i)}$, $y^{(i)})$  for $i=0....m$,
  \item The predicted prices, represented by  $(x_{thrust}^{(i)}$, $\hat{y}^{(i)})$  for $i=0....m$ (see example below).
\end{itemize}

\begin{figure}[!h]
  \centering
  \includegraphics[scale=0.6]{assets/ex07_price_vs_thrust_part1.png}
  \caption{Plot of the selling prices of spacecrafts with respect to the thrust power of their engines, as well as our second model's price predictions.}
\end{figure}
% ================================= %
\subsubsection*{Total distance}
% --------------------------------- %
Select the \textit{Terameters} feature as your $x$ vector, and \textit{Sell\_price} as your $y$ vector.
Train a third model, \texttt{myLR\_distance}, and make price predictions ($\hat{y}$).
Output a scatter plot with both sets of data points on the same graph, as follows:
\begin{itemize}
  \item The actual prices, given by $(x_{distance}^{(i)}$, $y^{(i)})$  for $i=0....m$, 
  \item The predicted prices, represented by  $(x_{distance}^{(i)}$, $\hat{y}^{(i)})$  for $i=0....m$  (see example below),
\end{itemize}

\begin{figure}[!h]
  \centering
  \includegraphics[scale=0.6]{assets/ex07_price_vs_Tmeters_part1.png}
  \caption{Plot of the selling prices of spacecrafts with respect to the terameters driven, as well as our third model's price predictions.}
\end{figure}

% ================================= %
\subsubsection*{Reminder}
% --------------------------------- %
\begin{itemize}
  \item After executing the \texttt{fit\_} method, you may obtain  $\theta = array([["nan", "nan"]])$.
    If it happens, try reducing your learning rate.
  \item Be aware that you also need to set the appropriate number of cycles for the \texttt{fit\_} function.
        If it's too low, you might not have completed enough cycles for the gradient descent to carry out properly.
        Try to find a value that gets you the best score, while not making the training last forever.
\end{itemize}

\hint{
  First, try plotting the data points $(x_{j},y)$.
  Then you can guess initial theta values that are not too far off.
  This will help your algorithm converge more easily.
}

% ================================= %
\subsubsection*{Examples}
% --------------------------------- %

\begin{minted}[bgcolor=darcula-back,formatcom=\color{lightgrey},fontsize=\scriptsize]{python}
import pandas as pd
import numpy as np
from mylinearregression import MyLinearRegression as MyLR

data = pd.read_csv("spacecraft_data.csv")
X = np.array(data[['Age']])
Y = np.array(data[['Sell_price']])
myLR_age = MyLR(thetas = [[1000.0], [-1.0]], alpha = 2.5e-5, max_iter = 100000)
myLR_age.fit_(X, Y)

y_pred = myLR_age.predict_(X)
myLR_age.mse_(Y, y_pred)
#Output
55736.867198...
\end{minted}
\\
How accurate is your model when you only take one feature into account?\\
\newpage

% ================================= %
\subsection*{Part Two: Multivariate Linear Regression (A New Hope)}
% --------------------------------- %

Now, it's time for your first multivariate linear regression!

% ================================= %
\subsubsection*{Instructions}
% --------------------------------- %
Here, you will train a single model that will take all features into account.
Your program is expected to perform steps similar to the ones in part one (fitting, displaying or generating 3 graphs, printing the thetas and the MSE).

% ================================= %
\subsubsection*{Training the model}
% --------------------------------- %
\begin{itemize}
  \item Train a single multivariate linear regression model on all three features.
  \item Display and interpret the resulting theta parameters.
        What can you say about the role that each feature plays in the price prediction?
  \item Evaluate the model with the Mean Squared Error.
        How good is your model doing, compared to the other three that you trained in Part One of this exercise?
\end{itemize}

\info{
  You can obtain a better fit if you increase the number of cycles.
}

% ================================= %
\subsubsection*{Examples}
% --------------------------------- %
\begin{minted}[bgcolor=darcula-back,formatcom=\color{lightgrey},fontsize=\scriptsize]{python}
import pandas as pd
import numpy as np
from mylinearregression import MyLinearRegression as MyLR

data = pd.read_csv("spacecraft_data.csv")
X = np.array(data[['Age','Thrust_power','Terameters']])
Y = np.array(data[['Sell_price']])
my_lreg = MyLR(thetas=[1.0, 1.0, 1.0, 1.0], alpha=9e-5, max_iter=500000)


# Example 0:
my_lreg.mse_(Y, my_lreg.predict_(X))
# Output:
144044.877...


# Example 1:
my_lreg.fit_(X, Y)
my_lreg.thetas
# Output:
array([[367.28849...]
       [-23.69939...]
       [  5.73622...]
       [ -2.63855...]])


# Example 2:
print(my_lreg.mse_(Y, my_lreg.predict_(X)))
# Output:
435.9325695...
\end{minted}

% ================================= %
\subsubsection*{Plotting the predictions}
% --------------------------------- %
Here we'll plot the model's predictions just like we did in Part One.
We'll make three graphs, each one displaying the predictions and the actual prices as a function of ONE of the features.

\begin{itemize}
  \item On the same graph, plot the actual and predicted prices on the $y$ axis , and the $age$ feature on the $x$ axis. (see figure below)
  \begin{figure}[!h]
    \centering
    \includegraphics[scale=0.6]{assets/ex07_price_vs_age_part2.png}
    \caption{Spacecraft sell prices of and predicted sell prices  with the multivariate hypothesis, with respect to the \textit{age} feature}
  \end{figure}
  
  \item On the same graph, plot the actual and predicted prices on the $y$ axis , and the $thrust power$ feature on the $x$ axis. (see figure below)
  \begin{figure}[!h]
    \centering
    \includegraphics[scale=0.6]{assets/ex07_price_vs_thrust_part2.png}
    \caption{Spacecraft sell prices predicted sell prices with the multivariate hypothesis, with respect to the thrust power of the engines}
  \end{figure}
  
  \item On the same graph, plot the actual and predicted prices on the $y$ axis , and the $distance$ feature on the $x$ axis. (see figure below)
  \begin{figure}[!h]
    \centering
    \includegraphics[scale=0.6]{assets/ex07_price_vs_Tmeters_part2.png}
    \caption{Spacecraft sell prices and predicted sell prices with the multivariage hypothesis, with respect to the driven distance (in terameters)}
  \end{figure}
\end{itemize}
\info{Can you see any improvement on these three graphs, compared to the three that you obtained in Part One?}
\info{Can you relate your observations to the MSE value that you just calculated?}