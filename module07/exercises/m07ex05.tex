\chapter{Exercise 05}
\extitle{Multivariate Linear Regression with Class}
\turnindir{ex05}
\exnumber{05}
\exfiles{mylinearregression.py}
\exforbidden{sklearn}
\makeheaderfilesforbidden

% ================================= %
\section*{Objective}
% --------------------------------- %
Upgrade your Linear Regression class so it can handle multivariate hypotheses.

% ================================= %
\section*{Instructions}
% --------------------------------- %
You are expected to upgrade your own \texttt{MyLinearRegression} class from \textbf{Module01}.
You will upgrade (at least) the following methods to support multivariate linear regression:
\begin{itemize}
  \item \texttt{predict\_(self, x)}, 
  \item \texttt{fit\_(self, x, y)}.
\end{itemize}
Depending on how you implement your methods, you might need to update other methods.

% ================================= %
\section*{Examples}
% --------------------------------- %
\begin{minted}[bgcolor=darcula-back,formatcom=\color{lightgrey},fontsize=\scriptsize]{python}
import numpy as np
from mylinearregression import MyLinearRegression as MyLR
X = np.array([[1., 1., 2., 3.], [5., 8., 13., 21.], [34., 55., 89., 144.]])
Y = np.array([[23.], [48.], [218.]])
mylr = MyLR([[1.], [1.], [1.], [1.], [1]])


# Example 0:
y_hat = mylr.predict_(X)
# Output:
array([[8.], [48.], [323.]])


# Example 1:
mylr.loss_elem_(Y, y_hat)
# Output:
array([[225.], [0.], [11025.]])


# Example 2:
mylr.loss_(Y, y_hat)
# Output:
1875.0


# Example 3:
mylr.alpha = 1.6e-4
mylr.max_iter = 200000
mylr.fit_(X, Y)
mylr.thetas
# Output:
array([[18.188..], [2.767..], [-0.374..], [1.392..], [0.017..]])


# Example 4:
y_hat = mylr.predict_(X)
# Output:
array([[23.417..], [47.489..], [218.065...]])


# Example 5:
mylr.loss_elem_(Y, y_hat)
# Output:
array([[0.174..], [0.260..], [0.004..]])


# Example 6:
mylr.loss_(Y, y_hat)
# Output:
0.0732..
\end{minted}