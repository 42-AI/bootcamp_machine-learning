\chapter{Exercise 08}
\extitle{Let's Train Polynomial Models!}
%******************************************************************************%
%                                                                              %
%                                 Interlude                                    %
%                         for Machine Learning module                          %
%                                                                              %
%******************************************************************************%

\section*{Interlude - Fifty Shades of Linear Algebra}

In the last exercise, we implemented the loss function in two subfunctions.
It worked, but it's not very pretty.
What if we could do it all in one step, with linear algebra?   

As we did with the hypothesis, we can use a vectorized equation to improve the calculations of the loss function.

So now let's look at how squaring and averaging can be performed (more or less) in a single matrix multiplication!

$$
J(\theta) = \frac{1}{2m}\sum_{i=1}^{m}(\hat{y}^{(i)} - y^{(i)})^2
$$
$$
J(\theta) = \frac{1}{2m}\sum_{i=1}^{m}[(\hat{y}^{(i)} - y^{(i)}) (\hat{y}^{(i)} - y^{(i)})]
$$

Now, if we apply the definition of the dot product:

$$
J(\theta) = \frac{1}{2m}(\hat{y} - y) \cdot(\hat{y}- y)
$$
\newpage
\turnindir{ex08}
\exnumber{08}
\exfiles{polynomial\_train.py}
\exforbidden{sklearn}
\makeheaderfilesforbidden


% ================================= %
\section*{Objective}
% --------------------------------- %
Manipulation of polynomial hypothesis.\\
\newline
It's training time! Let's train some polynomial models, and see if those with higher polynomial degree perform better!\\
\\
Write a program which:
\begin{itemize}
  \item Reads and loads \texttt{are\_blue\_pills\_magics.csv} dataset
  \item Trains \textbf{six} separate Linear Regression models with polynomial hypothesis with degrees ranging from 1 to 6
  \item Evaluates and prints evaluation score (MSE) of each of the six models
  \item Plots a bar plot showing the MSE score of the models in function of the polynomial degree of the hypothesis
  \item Plots the 6 models and the data points on the same figure
        Use lineplot style for the models and scatterplot for the data points
        Add more prediction points to have smooth curves for the models
\end{itemize}
You will use \texttt{Micrograms} as feature and \texttt{Score} as target.\\
\\
The implementation of the method \texttt{fit\_} based on the simple gradient descent lacks of efficiency and sturdiness,
which will lead to the impossibility of converging for polynomial models with high degree or with features having several orders of magnitude of difference.
See the starting values below for some thetas to help you get acceptable parameters values for your models.\\
\\
\hint{
  According to evaluation score only, what is the best hypothesis (or model) between the trained models?
  According to the last plot, why is it not true?
  Which phenomenon do you observe here?
}

\newpage
% ================================= %
\subsection*{Starting points}
% --------------------------------- %
You will not be able to get acceptable parameters for models 4, 5 and 6.
Thus you can start the fit process for those models with:\\
\newline
\begin{minted}[bgcolor=darcula-back,formatcom=\color{lightgrey},fontsize=\scriptsize]{python}
theta4 = np.array([[-20],[ 160],[ -80],[ 10],[ -1]]).reshape(-1,1)
theta5 = np.array([[1140],[ -1850],[ 1110],[ -305],[ 40],[ -2]]).reshape(-1,1)
theta6 = np.array([[9110],[ -18015],[ 13400],[ -4935],[ 966],[ -96.4],[ 3.86]]).reshape(-1,1)
\end{minted}

% ================================= %
\subsection*{Teminology Note}
% --------------------------------- %
The \textbf{degree} of a polynomial expression is its highest exponent.  
E.g.: The polynomial degree of $5x^3 - x^6 + 2 x^2$ is $6$.\\
\\
In this equation, you don't see any terms with $x$, $x^4$ and $x^5$,but we can still say they exist. It's just that their coefficient is $0$.
This means that a polynomial linear regression model can lower the impact of any term by bringing its corresponding $\theta_j$ closer to $0$.

% ================================= %
\subsection*{Remark}
% --------------------------------- %
When you are evaluated, it will be wise to run your program at the beginning of the evaluation as it can take several minutes to train the models.
