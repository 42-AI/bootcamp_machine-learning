%******************************************************************************%
%                                                                              %
%                        Common Instructions                                   %
%                          for C Projects                                      %
%                                                                              %
%******************************************************************************%

\chapter{Common Instructions}
    \begin{itemize}

	\item Il tuo progetto deve essere scritto seguendo le regole imposte
	dalla Norma. Se hai dei file o funzioni bonus, saranno inclusi nel controllo
	della Norma e riceverai uno \texttt{0} se vi sono errori.

	\item Le tue funzioni non devono terminare inaspettatamente(segementation
	fault, bus error, doppio free, etc) a meno che non si tratti di un comportamento
	non definito. Se ciò dovesse accadere, il tuo progetto sarà considerato come
	non funzionante e riceverai uno \texttt{0} durante la valutazione.

	\item Tutta la memoria allocata sull'heap deve essere liberata correttamente 
	quando necessario. Non sarà tollerato nessun leak.
	
	\item Dovrai consegnare un \texttt{Makefile} quando richiesto dal subject.
	Dovrà compilare i tuoi file sorgente nell;output richiesto utilizzando le flag 
	\texttt{-Wall}, \texttt{-Wextra} e \texttt{-Werror}.
	Non deve ricollegare.
	
	\item Il tuo \texttt{Makefile} deve contenere almeno le regole
        \texttt{\$(NAME)}, \texttt{all}, \texttt{clean},
        \texttt{fclean} e \texttt{re}.
      
	\item Per consegnare dei bonus, devi includere la regola \texttt{bonus} 
	nel tuo Makefile, che aggiungerà tutti gli header, librerie o funzioni proibite
	nella parte obbligatoria del progetto.
	I Bonus devono essere in dei file \texttt{\*\_bonus.\{c/h\}}. La valutazione della
	parte obbligatoria e dei bonus saranno condotte separatamente.
	
	\item Se il progetto consente l'utilizzo della tua \texttt{libft}, devi copiare i suoi sorgente
	ed il suo \texttt{Makefile} in una cartella \texttt{libft}.
	Il \texttt{Makefile} del tuo progetto deve compilare la libreria usando il suo \texttt{Makefile}, per poi compilare il progetto.

	\item Ti incoraggiamo a creare i tuoi programmi di test
	per il progetto anche se questi programmi \textbf{non dorvanno
	essere consegnati e non saranno valutati}. Ti darà la possibilità
	di testare facilmente il tuo lavoro e quello dei tuoi peer. Troverai questi
	programmi specialmente utili durante la tua difesa, durante la quale 
	sarai libero di utilizzare i tuoi, o quelli del tuo peer, file di test.

	 \item Consegna il tuo lavoro nella repository git che ti è stata assegnata.
	 Verrà valutato solo il lavoro presente nella repository git. Se Deepthought 
	 dovrà valutare il tuo lavoro, lo farà solo dopo le valutazioni tra pari.
	 Se Deepthought incontra un errore in qualsiasi sezione del tuo porgetto,
	 terminerà la valutazione immediatamente.
     
    \end{itemize}
