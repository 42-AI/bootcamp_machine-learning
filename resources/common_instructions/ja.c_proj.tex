%******************************************************************************%
%                                                                              %
%                        Common Instructions                                   %
%                          for C Projects                                      %
%                                                                              %
%******************************************************************************%

\chapter{Common Instructions}
    \begin{itemize}

      \item ノームに沿ってプロジェクトを進めなさい。
        ボーナスで追加されたファイルや関すもノームチェックが入ります。
        もしノームエラーがある場合、\texttt{0}となります。

      \item 関数が未定義の動作以外の予期しない終了(セグメーション違反、
        バスエラー、ダブルフリーなど)をしてはならない。これが起こった場合、
        あなたのプロジェクトは機能しないということになり、レビューでは
        \texttt{0}の評価になる。

      \item 確保された全てのメモリは、ちゃんとフリーされないといけない。
        メモリリークは許容されない。

      \item もし課題が要求している場合、あなたのソースファイルを適切なフラグ
        \texttt{-Wall}と\texttt{-Wextra}と\texttt{-Werror}でコンパイルし
        求められているアウトプットを出す\texttt{Makefile}を
        提出しなければならない。

      \item \texttt{Makefile} には最低限これらのルールを作成しなければならない。
        \texttt{\$(NAME)}, \texttt{all}, \texttt{clean},
        \texttt{fclean}と\texttt{re}。

      \item ボーナスを追加したい場合、Makefileの中に\texttt{bonus}というルールを作成しなさい。
        このルールは禁止されている様々なヘッダーやライブラリ、関数などをプロジェクトに追加
        するためのルールです。ボーナスはこのようなファイル\texttt{\*\_bonus.\{c/h\}}の中に
        追加してください。必須パートとボーナスパートは別々に評価されます。

      \item もし課題に自分の\texttt{libft}が提出可能と書かれている場合、
        そのソースファイルや\texttt{Makefile}を\texttt{libft}というフォルダに入れなさい。
        プロジェクトの\texttt{Makefile}はまずライブラリをコンパイルするための
        適切な\texttt{Makefile}を呼び出し、そのあとにプロジェクトをコンパイルする。
      \item \textbf{提出義務はなく評価もされないが}課題用の
        テストプログラムを作ることを推奨する。あなたの課題や他の生徒の課題を
        チェックしやすくなります。それらのテストケースはレビュー時に
        あなたの課題の正当性を証明するのに役立ちます。
        あなたのレビュー中や他の生徒のレビュー中に、テストを使っても問題ありません。

      \item アサインされているギットリポジトリーに提出してください。
        ギットリポジトリーに提出されたものだけが評価されます。Deepthoughtがレビュー後に
        あなたの評価を行います。もしDeepthoughtの評価中にエラーが起こった場合はその時点で評価が
        計算されます。
    \end{itemize}
