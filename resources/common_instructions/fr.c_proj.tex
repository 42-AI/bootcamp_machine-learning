%******************************************************************************%
%                                                                              %
%                        Common Instructions                                   %
%                          for C Projects                                      %
%                                                                              %
%******************************************************************************%

\chapter{Règles communes}
    \begin{itemize}

      \item Votre projet doit être écrit en C.

      \item Votre projet doit être codé à la Norme. Si vous avez des fichiers ou
        fonctions bonus, celles-ci seront inclues dans la vérification de la norme
        et vous aurez 0 au projet en cas de faute de norme.

      \item Vos fonctions de doivent pas s'arrêter de manière inattendue (segmentation
      fault, bus error, double free, etc) mis à part dans le cas d'un comportement
      indéfini. Si cela arrive, votre projet sera considéré non fonctionnel et vous 
      aurez 0 au projet.

      \item Toute mémoire allouée sur la heap doit être libéré lorsque c'est nécessaire.
        Aucun leak ne sera toléré.

      \item Si le projet le demande, vous devez rendre un Makefile qui compilera vos 
        sources pour créer la sortie demandée, en utilisant les flags \texttt{-Wall},
        \texttt{-Wextra} et \texttt{-Werror}. Votre Makefile ne doit pas relink.

      \item Si le projet demande un Makefile, votre Makefile doit au minimum
        contenir les règles \texttt{\$(NAME)}, \texttt{all}, \texttt{clean},
        \texttt{fclean} et \texttt{re}.

      \item Pour rendre des bonus, vous devez inclure une règle \texttt{bonus} à votre
        Makefile qui ajoutera les divers headers, librairies ou fonctions qui ne sont
        pas autorisées dans la partie principale du projet. Les bonus doivent être dans 
        un fichier différent :  \texttt{\*\_bonus.\{c/h\}}. L'évaluation de la partie 
        obligatoire et de la partie bonus sont faites séparément.

      \item Si le projet autorise votre \texttt{libft}, vous devez copier ses sources et 
        son Makefile associé dans un dossier libft contenu à la racine.
        Le Makefile de votre projet doit compiler la librairie à l'aide de son Makefile,
        puis compiler le projet.

      \item Nous vous recommandons de créer des programmes de test pour votre projet,
        bien que ce travail \textbf{ne sera pas rendu ni noté}. Cela vous donnera une chance
        de tester facilement votre travail ainsi que celui de vos pairs. 

      \item Vous devez rendre votre travail sur le git qui vous est assigné. Seul le travail
        déposé sur git sera évalué. Si Deepthought doit corriger votre travail, cela sera fait
        à la fin des peer-evaluations.
        Si une erreur se produit pendant l'évaluation Deepthought, celle-ci s'arrête.
    \end{itemize}
