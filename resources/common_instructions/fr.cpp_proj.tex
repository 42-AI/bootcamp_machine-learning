%******************************************************************************%
%                                                                              %
%                  instructions.tex for 42's Piscine C++                       %
%                  Created on : Mon Sep  8 15:57:19 2014                       %
%                  Made by : David "Thor" GIRON <thor@42.fr>                   %
%                                                                              %
%******************************************************************************%


\chapter{Règles Générales}


    \begin{itemize}
        \item Toute fonction implémentée dans une header (sauf pour les templates)
            ou tout header non-protégé, signifie 0 à l'exercice.
        
		\item Tout output doit être affiché sur stdout et terminé par une newline,
            sauf si autre chose est précisé.
		\item Les noms de fichiers imposés doivent être suivis à la lettre,
            tout comme les noms de classe, les noms de fonction, et les 
            noms de méthodes.
		\item Rappel : vous codez maintenant en \texttt{C++}, et plus en
            \texttt{C}. C'est pourquoi :
            \begin{itemize}
                \item Les fonctions suivantes sont \textbf{INTERDITES}, et 
                    leur usage se soldera par un 0: \texttt{*alloc}, 
                    \texttt{*printf} et \texttt{free}
                \item Vous avez l'autorisation d'utiliser à peu près toute 
                    la librairie standard. CEPENDANT, il serait intelligent
                    d'essayer d'utiliser la version C++ de ce à quoi vous êtes
                    habitués en C, plutôt que de vous reposer sur vos acquis.
                    Et vous n'êtes pas autorisés à utiliser la STL jusqu'au 
                    moment où vous commencez à travailler dessus (module 08).
                    Ca signifie pas de Vector/List/Map/etc... ou quoi que 
                    ce soit qui requiert une include <algorithm> jusque là.
		  \end{itemize}
        \item L'utilisation d'une fonction ou mécanique explicitement interdite 
          sera sanctionnée par un \texttt{0}
        \item Notez également que sauf si la consigne l'autorise, les mot-clés
          \texttt{using namespace} et \texttt{friend} sont interdits.
          Leur utilisation sera punie d'un \texttt{0}.
        \item Les fichiers associés à une classe seront toujours nommés \texttt{ClassName.cpp}
          et \texttt{ClassName.hpp}, sauf si la consigne demande autre chose.
        \item Vous devez lire les exemples minutieusement. Ils peuvent contenir
          des prérequis qui ne sont pas précisés dans les consignes.
        \item Vous n'êtes pas autorisés à utiliser des librairies externes,
          incluant \texttt{C++11}, \texttt{Boost}, et tous les autres outils
          que votre ami super fort vous a recommandé.
        \item Vous allez surement devoir rendre beaucoup de fichiers de classe,
          ce qui peut paraitre répétitif jusqu'à ce que vous appreniez a scripter
          ca dans votre éditeur de code préferé.
        \item Lisez complètement chaque exercice avant de le commencer.
        \item Le compilateur est \texttt{clang++}
        \item Votre code sera compilé avec les flags \texttt{-Wall -Wextra -Werror -std=c++98}
        \item Chaque include doit pouvoir être incluse indépendamment des autres includes.
          Un include doit donc inclure toutes ses dépendances.
        \item Il n'y a pas de norme à respecter en \texttt{C++}. Vous pouvez utiliser 
          le style que vous préferez. Cependant, un code illisible est un code 
          que l'on ne peut pas noter.
        \item Important : vous ne serez pas noté par un programme (sauf si précisé
          dans le sujet). Cela signifie que vous avez un degré de liberté dans 
          votre méthode de résolution des exercices. 
        \item Faites attention aux contraintes, et ne soyez pas fainéant, vous pourriez
          manquer beaucoup de ce que les exercices ont à offrir
        \item Ce n'est pas un problème si vous avez des fichiers additionnels. Vous pouvez 
          choisir de séparer votre code dans plus de fichiers que ce qui est demandé, 
          tant qu'il n'y a pas de moulinette.
        \item Même si un sujet est court, cela vaut la peine de passer un peu de temps dessus
          afin d'être sûr que vous comprenez bien ce qui est attendu de vous, et que vous l'avez 
          bien fait de la meilleure manière possible.

\end{itemize}

%******************************************************************************%
\newpage
