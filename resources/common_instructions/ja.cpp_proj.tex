%******************************************************************************%
%                                                                              %
%                  instructions.tex for 42's Piscine C++                       %
%                  Created on : Mon Sep  8 15:57:19 2014                       %
%                  Made by : David "Thor" GIRON <thor@42.fr>                   %
%                                                                              %
%******************************************************************************%


\chapter{General rules}


    \begin{itemize}

		\item ヘッダファイルに定義されている関数がある場合や
            保護されていないヘッダファイルの場合は
            その問題に対して0点評価となります。

		\item 全てのアウトプットは標準出力へとアウトプットしなさい。
            それと何も記載されていない場合、改行を付け足しなさい。

		\item 指定されたファイルの名前は一寸違わずに同じものにしてください。
            クラスの名前や関数の名前、メソッドの名前も同様に。

		\item 注意点:現時点では\texttt{C++}でプログラミングを行なっています。
            \texttt{C}ではありません。なので:
	  
		  \begin{itemize}
		  
		  \item	\texttt{*alloc}、\texttt{*printf}と\texttt{free}。
            これらの関数の使用は禁止です。
            これらの関数を使用した場合、評価は0点となります。
            質問は受け付けません。

		  \item 標準C++ライブラリにある全ての関数は使用可能です。しかし、
            Cで学んで使い慣れている関数よりC++っぽい関数を試す方が賢いです。
            新しい言語なので新しいものを学びましょう。
            使用許可が下りるまで(モジュール08から)STLの使用は禁止です。
            つまり、ベクトル・リスト・マップなど<algorithm>を用いるものは禁止です。

		  \end{itemize}

		\item 使用が許可されていない関数やメカニックは0点となります。
            質問は受け付けません。

        \item それと記載されていない場合、\texttt{C++}のキーワード、
            \texttt{"using namespace"}と\texttt{"friend"}の使用は禁止です。
            \texttt{-42}点となります。質問は受け付けません。

        \item クラス関連のファイルは指定がない場合、
            \texttt{ClassName.hpp}や\texttt{ClassName.cpp}という風に
            名付けてください。

        \item 提出するディレクトリは、\texttt{ex00/}, \texttt{ex01/},
          \dots, \texttt{exn/}になります。

        \item 例を徹底的に読んでください。定かではない要求が問題の詳細にあるかもしれません。
            もし何かが曖昧な場合、\texttt{C++}の理解が足りないということです。

        \item 初めの頃から学んだ\texttt{C++}のツールは使用可能ですが、
            外部のライブラリは使用不可です。質問する前に、こちらが明確にすることは
            \texttt{C++11}やその派生、\texttt{Boost}の使用や技術力が高い
            友達が教えてくれたこれなしでは生きていけない\texttt{C++}関連も
            使用不可です。

        \item かなりのクラスを提出しないといけない場面があります。
            退屈な作業になるかもしれませんが、あなたの好きなテキストエディタ上に
            スクリプトを付け足せばその作業を行わなくてもよくなるかもしれません。

        \item 各問題に手をつける前にちゃんと問題を読んでください!お願い、マジで。

        \item 使用するコンパイラは\texttt{clang++}です。

        \item これらのフラグを用いてコンパイルしてください。
            フラグ:\texttt{-Wall -Wextra -Werror}

        \item 各インクルードは依存せずにインクルード可能でなければならない。
             もちろん、インクルードは依存するインクルードを入れなければならない。

        \item 疑問に思っている場合、\texttt{C++}ように強制されている
            コーディング規約はありません。
            自分の好きなコードの書き方をを用いてもらっても構いません。
            しかし、忘れてはいけないのは他の人でも読めるような、レビューできるような
            コードを書きましょう。

	    \item 重要点:課題で記載されていない限り、プログラムによる評価はありません。
            なので問題の取り組み方には自由度が増します。
            しかし各問題の制約については忘れないでください。怠けないでください。

		\item 自分のコードを綺麗に分担するために追加でファイルを提出するのは問題ありません。
            プログラムに評価されないのであれば、ご自由にどうぞ。

		\item 問題の内容が小さくとも、完璧に理解しベストを尽くすように取り組みましょう。
            時間をかけるほどの価値はあります。

        \item 頭を使ってください!

\end{itemize}

%******************************************************************************%
\newpage
