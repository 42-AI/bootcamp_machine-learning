%******************************************************************************%
%                                                                              %
%                        Common Instructions                                   %
%                          for Python Projects                                 %
%                                                                              %
%******************************************************************************%

\chapter{Common Instructions}
\begin{itemize}
  \item The version of Python recommended to use is 3.7. You can
  check your Python's version with the following command: \texttt{python -V}
  
  \item The norm: during this bootcamp, it is recommended to follow the
  \href{https://www.python.org/dev/peps/pep-0008/}{PEP 8 standards}, though it is not mandatory.
  You can install \href{https://pypi.org/project/pycodestyle}{pycodestyle} or 
  \href{https://black.readthedocs.io/en/stable/}{Black}, which are convenient 
  packages to check your code.
  
  \item The function \texttt{eval} is never allowed.
  
  \item The exercises are ordered from the easiest to the hardest.
  
  \item Your exercises are going to be evaluated by someone else,
  so make sure that your variable names and function names are appropriate and civil.

  \item Your manual is the internet.

  \item If you're planning on using an AI assistant such as a LLM, make sure it is helpful 
  for you to \textbf{learn and practice}, not to provide you with hands-on solution ! Own your tool, don't let it own you.
  
  \item If you are a student from 42, you can access our Discord server 
  on \href{https://discord.com/channels/887850395697807362/887850396314398720}{42 student's associations portal} and ask your
  questions to your peers in the dedicated Bootcamp channel. 

  \item You can learn more about 42 Artificial Intelligence by visiting \href{https://42-ai.github.io}{our website}.

  \item If you find any issue or mistake in the subject please create an issue on 
  \href{https://github.com/42-AI/bootcamp_machine-learning/issues}{42AI repository on Github}.
  
  \item We encourage you to create test programs for your
  project even though this work \textbf{won't have to be
  submitted and won't be graded}. It will give you a chance
  to easily test your work and your peers’ work. You will find
  those tests especially useful during your defence. Indeed,
  during defence, you are free to use your tests and/or the
  tests of the peer you are evaluating.

\end{itemize}