%******************************************************************************%
%                                                                              %
%                                 Interlude                                    %
%                         for Machine Learning module                          %
%                                                                              %
%******************************************************************************%

% =============================== %
\section*{Interlude - Gradient Descent}
% ------------------------------- %

So far we've calculated the \textit{gradient},
which indicates whether and by how much we should increase or decrease $\theta_0$ and $\theta_1$ in order to reduce the loss.
What we have to do next is update the theta parameters accordingly,
step by step, until we reach the minimum.
This iterative process, called \textbf{Gradient Descent},
will progressively improve the performance of your regression model on the training data. 

The gradient descent \textbf{algorithm} can be summarized like this:
for a certain number of cycles, at each step,
both $\theta$ parameters are slightly moved in the opposite directions than what the gradient indicates.


The algorithm can be expressed in pseudocode as the following:
$$
\begin{matrix}
&\text{repeat until convergence:} & \{\\
&    \text{compute } \nabla{(J)}  \\
&	\theta_0 := \theta_0 - \alpha \nabla(J)_0  \\ 
&	\theta_1 := \theta_1 - \alpha \nabla(J)_1\\
	\} \hspace{0.5cm} 
\end{matrix}
$$

Few remarks on this algorithm:
\begin{itemize}
  \item If you directly subtracted the gradient from $\theta$,
        your steps would be too big and you would quickly overshoot past the minimum.
        That's why we use $\alpha$ (alpha), called the \textit{learning rate}.
        It's a small float number (usually between 0 and 1) that decreases the magnitude of each update.
  \item The pseudocode says "repeat until convergence",
        but in your implementation, you will not actually check for convergence at each iteration.
        You will instead set a number of cycles that is sufficient for your gradient descent to converge. 
  \item When training a linear regression model on a new dataset,
        you will have to choose appropriate alpha and the number of cycles through trial and error.
\end{itemize}

