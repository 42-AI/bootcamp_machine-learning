% vim: set ts=4 sw=4 tw=80 noexpandtab:

\documentclass{42-en}

%******************************************************************************%
%                                                                              %
%                                   Prologue                                   %
%                                                                              %
%******************************************************************************%
\usepackage[
    type={CC},
    modifier={by-nc-sa},
    version={4.0},
]{doclicense}
\usepackage{amsmath} % The amsmath package provides commands to typeset matrices with different delimiters. 
\usepackage{epigraph}
\setlength\epigraphwidth{.95\textwidth}
%****************************************************************%
%                  Re/definition of commands                     %
%****************************************************************%

\newcommand{\ailogo}[1]{\def \@ailogo {#1}}\ailogo{assets/42ai_logo.pdf}

%%  Redefine \maketitle
\makeatletter
\def \maketitle {
  \begin{titlepage}
    \begin{center}
	%\begin{figure}[t]
	  %\includegraphics[height=8cm]{\@ailogo}
	  \includegraphics[height=8cm]{assets/42ai_logo.pdf}
	%\end{figure}
      \vskip 5em
      {\huge \@title}
      \vskip 2em
      {\LARGE \@subtitle}
      \vskip 4em
    \end{center}
    %\begin{center}
	  %\@author
    %\end{center}
	%\vskip 5em
  \vfill
  \begin{center}
    \emph{\summarytitle : \@summary}
  \end{center}
  \vspace{2cm}
  %\vskip 5em
  %\doclicenseThis
  \end{titlepage}
}
\makeatother

\makeatletter
\def \makeheaderfilesforbidden
{
  \noindent
  \begin{tabularx}{\textwidth}{|X X X X|}
    \hline
  \multicolumn{1}{|>{\raggedright}m{1cm}|}
  {\vskip 2mm \includegraphics[height=1cm]{assets/42ai_logo.pdf}} &
  \multicolumn{2}{>{\centering}m{12cm}}{\small Exercise : \@exnumber } &
  \multicolumn{1}{ >{\raggedleft}p{1.5cm}|}
%%              {\scriptsize points : \@exscore} \\ \hline
              {} \\ \hline

  \multicolumn{4}{|>{\centering}m{15cm}|}
              {\small \@extitle} \\ \hline

  \multicolumn{4}{|>{\raggedright}m{15cm}|}
              {\small Turn-in directory : \ttfamily
                $ex\@exnumber/$ }
              \\ \hline
  \multicolumn{4}{|>{\raggedright}m{15cm}|}
              {\small Files to turn in : \ttfamily \@exfiles }
              \\ \hline

  \multicolumn{4}{|>{\raggedright}m{15cm}|}
              {\small Forbidden functions : \ttfamily \@exforbidden }
              \\ \hline

%%  \multicolumn{4}{|>{\raggedright}m{15cm}|}
%%              {\small Remarks : \ttfamily \@exnotes }
%%              \\ \hline
\end{tabularx}
%% \exnotes
\exrules
\exmake
\exauthorize{None}
\exforbidden{None}
\extitle{}
\exnumber{}
}
\makeatother

%%  Syntactic highlights
\makeatletter
\newenvironment{pythoncode}{%
  \VerbatimEnvironment
  \usemintedstyle{emacs}
  \minted@resetoptions
  \setkeys{minted@opt}{bgcolor=black,formatcom=\color{lightgrey},fontsize=\scriptsize}
  \begin{figure}[ht!]
    \centering
    \begin{minipage}{16cm}
      \begin{VerbatimOut}{\jobname.pyg}}
{%[
      \end{VerbatimOut}
      \minted@pygmentize{c}
      \DeleteFile{\jobname.pyg}
    \end{minipage}
\end{figure}}
\makeatother

\usemintedstyle{native}

\begin{document}

% =============================================================================%
%                     =====================================                    %

\title{Machine Learning Bootcamp - Module 01}
\subtitle{Univariate Linear Regression}
\author{
  Maxime Choulika (cmaxime), Pierre Peigné (ppeigne), Matthieu David (mdavid), Amir Mahla (amahla), Mathieu Perez (maperez)
}

\summary
{
  Today you will implement a crucial method to improve your model's performance: \textbf{gradient descent}.
  Then you will discover the notion of \textbf{normalization}.
}

\maketitle
%******************************************************************************%
%                                                                              %
%                        Section usefull ressources                            %
%                          for ML Modules                                      %
%                                                                              %
%******************************************************************************%


\chapter*{Notions covered and learning resources}

\section*{What notions will be covered by this module?}

\begin{itemize}
    \item Multivariate linear hypothesis
    \item Multivariate linear gradient descent
    \item Polynomial models
    \item Training set, test set, overfitting
\end{itemize}

\section*{Useful Resources}

You are recommended to use the following material: \href{https://www.coursera.org/learn/machine-learning}{Machine Learning MOOC - Stanford}\\
\newline
This series of videos is available at no cost: simply log in, select "Enroll for Free", and click "Audit" at the bottom of the pop-up window.\\
\newline
The following sections of the course are particularly relevant to today's exercises: 

\subsection*{Week 2: Regression with multiple input variables}

\subsubsection*{Multiple linear regression}
\begin{itemize}
  \item Multiple features
  \item Gradient descent for multiple linear regression
\end{itemize}

\subsubsection*{Gradient descent in practice}
\begin{itemize}
  \item Feature scaling part 1
  \item Feature scaling part 2
  \item Checking gradient descent for convergence
  \item Choosing the learning rate
  \item Feature engineering
  \item Polynomial regression
\end{itemize}
\emph{All videos above are available also on this \href{https://youtube.com/playlist?list=PLkDaE6sCZn6FNC6YRfRQc_FbeQrF8BwGI&feature=shared}{Andrew Ng's YouTube playlist}, videos 21 and from 24 to 30}
%******************************************************************************%
%                                                                              %
%                        Common Instructions                                   %
%                          for Python Projects                                 %
%                                                                              %
%******************************************************************************%

\chapter{Common Instructions}
\begin{itemize}
  \item The version of Python recommended to use is 3.7. You can
  check your Python's version with the following command: \texttt{python -V}
  
  \item The norm: during this bootcamp, it is recommended to follow the
  \href{https://www.python.org/dev/peps/pep-0008/}{PEP 8 standards}, though it is not mandatory.
  You can install \href{https://pypi.org/project/pycodestyle}{pycodestyle} or 
  \href{https://black.readthedocs.io/en/stable/}{Black}, which are convenient 
  packages to check your code.
  
  \item The function \texttt{eval} is never allowed.
  
  \item The exercises are ordered from the easiest to the hardest.
  
  \item Your exercises are going to be evaluated by someone else,
  so make sure that your variable names and function names are appropriate and civil.

  \item Your manual is the internet.

  \item If you're planning on using an AI assistant such as a LLM, make sure it is helpful 
  for you to \textbf{learn and practice}, not to provide you with hands-on solution ! Own your tool, don't let it own you.
  
  \item If you are a student from 42, you can access our Discord server 
  on \href{https://discord.com/channels/887850395697807362/887850396314398720}{42 student's associations portal} and ask your
  questions to your peers in the dedicated Bootcamp channel. 

  \item You can learn more about 42 Artificial Intelligence by visiting \href{https://42-ai.github.io}{our website}.

  \item If you find any issue or mistake in the subject please create an issue on 
  \href{https://github.com/42-AI/bootcamp_machine-learning/issues}{42AI repository on Github}.
  
  \item We encourage you to create test programs for your
  project even though this work \textbf{won't have to be
  submitted and won't be graded}. It will give you a chance
  to easily test your work and your peers’ work. You will find
  those tests especially useful during your defence. Indeed,
  during defence, you are free to use your tests and/or the
  tests of the peer you are evaluating.

\end{itemize}
\newpage
\tableofcontents
\startexercices

%                     =====================================                    %
% =============================================================================%


%******************************************************************************%
%                                                                              %
%                                   Exercises                                  %
%                                                                              %
%******************************************************************************%

% ============================================== %
% ===========================(start ex 00)       %
\chapter{Exercise 00}
%******************************************************************************%
%                                                                              %
%                                 Interlude                                    %
%                         for Machine Learning module                          %
%                                                                              %
%******************************************************************************%

% =============================== %
\section*{Interlude}
% =============================== %
\subsection*{Classification: The Art of Labelling Things}
% ******************************* %
Over the last three modules you have implemented your first machine learning algorithm.\\
\\
You are now familiar the three-steps cycle we follow when we build \textbf{learning algorithms}:
\\
\begin{figure}[!h]
    \centering
    \includegraphics[scale=0.25]{assets/Default.png}
    %\caption{The Learning Cycle}
\end{figure}
\\
The first algorithm you discovered, \textbf{Multivariate Linear Regression}, can now be used to predict a numerical value, based on several features.
This algorithm uses gradient descent to optimize its loss function.\\
\\
Now let's introduce your first \textbf{classification algorithm}: the notorious \textbf{Logistic Regression.}
\hint{regression vs classification; discrete vs continuous values}
\newpage
\noindent{\textbf{Logistic regression} performs a \textit{classification task}, which means that you are not predicting a numerical value (like price, age, grades...) 
but \textbf{categories}, or \textbf{labels} (like dog, cat, sick/healty...)}.
\\
\warn{
    Don't be confused by the word \textit{'regression'} in \textbf{Logistic Regression}.
    It really is a \textit{classification task}! The name is a bit tricky but you will quickly get used to it.
    Once again: \textbf{Logistic Regression is a classification algorithm} which assigns a label/category/class to a given example.
}
\info{
    In this module we will use the following terms interchangeably: \textbf{class}, \textbf{category}, and \textbf{label}.
    They all refer to the \textit{groups} to which each training example can be assigned to, in a classification task.
}

% =============================== %
\subsection*{Predict I: Introducing the Sigmoid Function}
% ******************************* %

\begin{figure}[!h]
    \centering
    \includegraphics[scale=0.25]{assets/Predict.png}
    %\caption{The Learning Cycle - Predict}
\end{figure}

% =============================== %
\subsubsection*{Formulating a Hypothesis}
% ******************************* %
Remember that a hypothesis, denoted $h(\theta)$, is an equation that combines a set of \textbf{features} (that characterizes an example) with \textbf{parameters} in order to output a \textbf{prediction}.\\
\\
Remember the hypothesis we used in linear regression?\\
$$
h(\theta) = \theta_0 + \theta_{1} x_{1}^{(i)} + \dots + \theta_{n} x_{n}^{(i)} = \theta \cdot x'^{(i)}
$$
\newline
It worked fine to predict continuous values, but could we also use it to tell, for example, 
if a patient is sick or not?
That's a yes-or-no question, so the output from the hypothesis function should reflect that.\\
\\
To get started, we will assign each class a numerical value: sick patients will be 
assigned a value of 1, and healthy patients will be assigned a value of 0.\\
The goal will be to build a hypothesis that outputs a probability that a patient is sick as a float number in the range of [0, 1].\\
\\
The good news is that we can keep the linear equation we already worked with!\\
\\
All we need to do is squash its output through another function that is bounded between 0 and 1.\\
\\
That's the purpose of the \textbf{Sigmoid function} and your next assignment is to implement it!

\newpage
\extitle{Linear Gradient - Iterative Version}
\turnindir{ex00}
\exnumber{00}
\exfiles{gradient.py}
\exforbidden{None}
\makeheaderfilesforbidden

% ================================== %
\section*{Objective}
% ---------------------------------- %
Understand and manipulate the notion of gradient and gradient descent in machine learning.\\
\newline
You must write a function that computes the \textbf{\textit{gradient}} of the loss function.
It must compute a partial derivative with respect to each theta parameter separately, and return the vector gradient.\\
\newline
The partial derivatives can be calculated with the following formulas:  
$$
\nabla(J)_0 = \frac{1}{m}\sum_{i=1}^{m}(h_{\theta}(x^{(i)}) - y^{(i)})
$$

$$
\nabla(J)_1 = \frac{1}{m}\sum_{i=1}^{m}(h_{\theta}(x^{(i)}) - y^{(i)})x^{(i)}
$$
Where:
\begin{itemize}
  \item $\nabla(J)$ is the gradient vector of size $2 \times 1$, (this strange symbol : $\nabla$ is called nabla)
  \item $x$ is a vector of dimension $m$
  \item $y$ is a vector of dimension $m$
  \item $x^{(i)}$ is the i$^\text{th}$ component of vector $x$
  \item $y^{(i)}$ is the i$^\text{th}$ component of vector $y$
  \item $\nabla(J)_j$ is the j$^\text{th}$ component of $\nabla(J)$
  \item $h_{\theta}(x^{(i)})$ corresponds to the model's prediction of $y^{(i)}$
\end{itemize}

% ================================== %
\section*{Hypothesis Notation}
% ---------------------------------- %
$h_{\theta}(x^{(i)})$ is the same as what we previously noted $\hat{y}^{(i)}$.  
The two notations are equivalent.
They represent the model's prediction (or estimation) of the ${y}^{(i)}$ value.
If you follow Andrew Ng's course material on Coursera, you will see him using the former notation.
\newline
As a reminder:
$h_{\theta}(x^{(i)}) = \theta_0 + \theta_1x^{(i)}$

% ================================== %
\section*{Instructions}
% ---------------------------------- %

In the \texttt{gradient.py} file create the following function as per the instructions given below:
\newline
\begin{minted}[bgcolor=darcula-back,formatcom=\color{lightgrey},fontsize=\scriptsize]{python}
  def simple_gradient(x, y, theta):
    """Computes a gradient vector from three non-empty numpy.arrays, with a for-loop.
       The three arrays must have compatible shapes.
    Args:
      x: has to be an numpy.array, a vector of shape m * 1.
      y: has to be an numpy.array, a vector of shape m * 1.
      theta: has to be an numpy.array, a 2 * 1 vector.
    Return:
      The gradient as a numpy.array, a vector of shape 2 * 1.
      None if x, y, or theta are empty numpy.array.
      None if x, y and theta do not have compatible shapes.
      None if x, y or theta is not of the expected type.
    Raises:
      This function should not raise any Exception.
    """
    ... Your code ...
\end{minted}

% ================================== %
\section*{Examples}
% ---------------------------------- %

\begin{minted}[bgcolor=darcula-back,formatcom=\color{lightgrey},fontsize=\scriptsize]{python}
import numpy as np
x = np.array([12.4956442, 21.5007972, 31.5527382, 48.9145838, 57.5088733]).reshape((-1, 1))
y = np.array([37.4013816, 36.1473236, 45.7655287, 46.6793434, 59.5585554]).reshape((-1, 1))

# Example 0:
theta1 = np.array([2, 0.7]).reshape((-1, 1))
simple_gradient(x, y, theta1)
# Output:
array([[-19.0342574], [-586.66875564]])

# Example 1:
theta2 = np.array([1, -0.4]).reshape((-1, 1))
simple_gradient(x, y, theta2)
# Output:
array([[-57.86823748], [-2230.12297889]])
\end{minted}
% ===========================(fin ex 00)         %
% ============================================== %
\newpage
% ============================================== %
% ===========================(start ex 01)       %
\chapter{Exercise 01}
%******************************************************************************%
%                                                                              %
%                                 Interlude                                    %
%                         for Machine Learning module                          %
%                                                                              %
%******************************************************************************%

% =============================== %
\section*{Linear Algebra Tricks part II}
% ******************************* %

If you tried to run your code on a very large dataset, you would find that it sometimes takes a (very) long time to execute!
That's because it doesn't use the power of Python libraries that are optimized for matrix operations.\\
\newline
Remember the linear algebra trick from the previous module? Let's use it again!  
If you concatenate a column of $1$'s to the left of the $x$ vector, you get what we called matrix $X'$.   
$$
X' = \begin{bmatrix} 1 & x^{(1)} \\ \vdots & \vdots \\ 1 & x^{(m)}\end{bmatrix}
$$
This transformation is very convenient because we can rewrite each $1$ as $x_0^{(i)}$, and each $x^{(i)}$ as $x_1^{(i)}$.
So now the $X'$ matrix looks like this:
$$
X' = \begin{bmatrix} x_0^{(1)} & x_1^{(1)} \\ \vdots & \vdots \\ x_0^{(m)} & x_1^{(m)}\end{bmatrix}
$$
Notice that each $x^{(i)}$ example becomes a vector made of $(x^{(i)}_0, x^{(i)}_1)$.  
The $0$ and $1$ indices on the $x$ features correspond to the indices of the $\theta$ parameters with which they will be multiplied.\\
\newline
Why does this matter?
Well, if we take the equation from the previous exercise:

$$
\nabla(J)_0 = \frac{1}{m}\sum_{i=1}^{m}(h_{\theta}(x^{(i)}) - y^{(i)})
$$
We can multiply it by $1$ without changing its value:
$$
\nabla(J)_0 = \frac{1}{m}\sum_{i=1}^{m}(h_{\theta}(x^{(i)}) - y^{(i)}) \cdot 1
$$
And rewrite $1$ as $x_0^{(i)}$:
$$
\nabla(J)_0 = \frac{1}{m}\sum_{i=1}^{m}(h_{\theta}(x^{(i)}) - y^{(i)})x_{0}^{(i)}
$$
This means that the equation for $\nabla(J)_0$ is now similar to the equation we had for $\nabla(J)_1$, so they can both be captured by ONE \textbf{generic equation}:
$$
\begin{matrix}
\nabla(J)_j = \frac{1}{m}\sum_{i=1}^{m}(h_{\theta}(x^{(i)}) - y^{(i)})x_{j}^{(i)} & & \text{ for j = 0, 1}    
\end{matrix}
$$
And as you probably suspected, a generic equation opens the door to vectorization...

% =============================== %
\subsection*{Vectorizing the Gradient Calculation}
% ******************************* %
Now it's time to learn how to calculate the entire gradient in one short, pretty, linear algebra equation!  
\begin{itemize}
    \item First, we'll use the $X'$ matrix and our vectorized hypothesis equation $h_{\theta}(x)=X'\theta$
    $$
    \begin{matrix}
    \nabla(J)_j = \frac{1}{m} (X'\theta - y)X'_{j} & & \text{ for j = 0, 1}
    \end{matrix}
    $$
    
    \item Second, we need to tweak the equation a bit so that it directly returns a $\nabla(J)$ vector containing both $\nabla(J)_0$ and $\nabla(J)_1$.
    
    $$
    \nabla(J) = \frac{1}{m} {X'}^T(X'\theta - y)    
    $$
\end{itemize}
If the equation does not seems obvious, play a bit with your vectors, on paper and in your code, until you get it.\\

% =============================== %
\subsubsection*{Notation Remark}
% ******************************* %
${X'}^T$: You might wonder what the $^T$ is for.
It means the $X'$ matrix must be \textbf{transposed}.\\
\newline
Transposing a matrix flips it on its diagonal so that its rows become its columns and \textit{vice-versa}.
Here we need to make sure that matrix dimensions are appropriate and allow for multiplication, and to multiply the right items together.
\newpage
\extitle{Linear Gradient - Vectorized Version}
\turnindir{ex01}
\exnumber{01}
\exfiles{vec\_gradient.py}
\exforbidden{None}
\makeheaderfilesforbidden

% ================================= %
\section*{Objective}
% --------------------------------- %
Understand and experiment with the notions of \textbf{gradient} and \textbf{gradient descent} in machine learning.\\
\newline
You must implement the following formula as a function:

$$
\nabla(J) = \frac{1}{m} {X'}^T(X'\theta - y)
$$  
Where:
\begin{itemize}
  \item $\nabla(J)$ is a vector of dimension $2 \times 1$
  \item $X'$ is a \textbf{matrix} of dimensions $(m \times 2)$
  \item ${X'}^T$ is the transpose of $X'$. Its dimensions are $(2 \times m)$
  \item $y$ is a vector of dimension $m$
  \item $\theta$ is a vector of dimension $2 \times 1$ 
\end{itemize}
Be careful:
\begin{itemize}
  \item the $x$ you will get as an input is an $m$ vector,
  \item $\theta$ is a $2 \times 1$ vector. You have to transform $x$ to fit the dimension of $\theta$!
\end{itemize}
\newpage
% ================================= %
\section*{Instructions}
% --------------------------------- %
In the \texttt{vec\_gradient.py} file create the following function as per the instructions given below:
\newline
\par
\begin{minted}[bgcolor=darcula-back,formatcom=\color{lightgrey},fontsize=\scriptsize]{python}
def simple_gradient(x, y, theta):
    """Computes a gradient vector from three non-empty numpy.arrays, without any for loop.
       The three arrays must have compatible shapes.
    Args:
      x: has to be a numpy.array, a vector of shape m * 1.
      y: has to be a numpy.array, a vector of shape m * 1.
      theta: has to be a numpy.array, a 2 * 1 vector.
    Return:
      The gradient as a numpy.ndarray, a vector of dimension 2 * 1.
      None if x, y, or theta is an empty numpy.ndarray.
      None if x, y and theta do not have compatible dimensions.
    Raises:
      This function should not raise any Exception.
    """
    ... Your code ...
\end{minted}

% ================================= %
\section*{Examples}
% --------------------------------- %

\begin{minted}[bgcolor=darcula-back,formatcom=\color{lightgrey},fontsize=\scriptsize]{python}
import numpy as np
x = np.array([12.4956442, 21.5007972, 31.5527382, 48.9145838, 57.5088733]).reshape((-1, 1))
y = np.array([37.4013816, 36.1473236, 45.7655287, 46.6793434, 59.5585554]).reshape((-1, 1))

# Example 0:
theta1 = np.array([2, 0.7]).reshape((-1, 1))
gradient(x, y, theta1)
# Output:
array([[-19.0342...], [-586.6687...]])

# Example 1:
theta2 = np.array([1, -0.4]).reshape((-1, 1))
gradient(x, y, theta2)
# Output:
array([[-57.8682...], [-2230.1229...]])
\end{minted}
% ===========================(fin ex 01)         %
% ============================================== %
\newpage
% ============================================== %
% ===========================(start ex 02)       %
\input{exercises/m06ex02.tex}
% ===========================(fin ex 02)         %
% ============================================== %
\newpage
% ============================================== %
% ===========================(start ex 03)       %
\chapter{Exercise 03}
\extitle{Linear Regression with Class}
\turnindir{ex03}
\exnumber{03}
\exfiles{my\_linear\_regression.py}
\exforbidden{any functions from sklearn}
\makeheaderfilesforbidden

% ================================= %
\section*{Objective}
% --------------------------------- %
Write a class that contains all the necessary methods to perform a linear regression.
% ================================= %
\section*{Instructions}
% --------------------------------- %
In this exercise, you will not learn anything new but don't worry, it's for your own good.\\
\newline
You are expected to write your own \texttt{MyLinearRegression} class which looks similar to the class available in Scikit-learn:
\texttt{sklearn.linear\_model.LinearRegression}\\
\newline
\par
\begin{minted}[bgcolor=darcula-back,formatcom=\color{lightgrey},fontsize=\scriptsize]{python}
class MyLinearRegression():
	"""
	Description:
		My personnal linear regression class to fit like a boss.
	"""
	def __init__(self, thetas, alpha=0.001, max_iter=1000):
				self.alpha = alpha
				self.max_iter = max_iter
				self.thetas = thetas

	#... other methods ...
\end{minted}
\newpage
You will add the following methods:
\begin{itemize}
  \item \texttt{fit\_(self, x, y)},
  \item \texttt{predict\_(self, x)},
  \item \texttt{loss\_elem\_(self, y, y\_hat)},
  \item \texttt{loss\_(self, y, y\_hat)}.
\end{itemize}
You have already implemented these functions, you just need a few adjustments so that they all work well within your \texttt{MyLinearRegression} class.

% ================================= %
\section*{Examples}
% --------------------------------- %
\begin{minted}[bgcolor=darcula-back,formatcom=\color{lightgrey},fontsize=\scriptsize]{python}
import numpy as np
from my_linear_regression import MyLinearRegression as MyLR
x = np.array([[12.4956442], [21.5007972], [31.5527382], [48.9145838], [57.5088733]])
y = np.array([[37.4013816], [36.1473236], [45.7655287], [46.6793434], [59.5585554]])

lr1 = MyLR(np.array([[2], [0.7]]))

# Example 0.0:
y_hat = lr1.predict_(x)
# Output:
array([[10.74695094],
		[17.05055804],
		[24.08691674],
		[36.24020866],
		[42.25621131]])

# Example 0.1:
lr1.loss_elem_(y, y_hat)
# Output:
array([[710.45867381],
		[364.68645485],
		[469.96221651],
		[108.97553412],
		[299.37111101]])

# Example 0.2:
lr1.loss_(y, y_hat)
# Output:
195.34539903032385

# Example 1.0:
lr2 = MyLR(np.array([[1], [1]]), 5e-8, 1500000)
lr2.fit_(x, y)
lr2.thetas
# Output:
array([[1.40709365],
		[1.1150909 ]])

# Example 1.1:
y_hat = lr2.predict_(x)
# Output:
array([[15.3408728 ],
		[25.38243697],
		[36.59126492],
		[55.95130097],
		[65.53471499]])

# Example 1.2:
lr2.loss_elem_(y, y_hat)
# Output:
array([[486.66604863],
		[115.88278416],
		[ 84.16711596],
		[ 85.96919719],
		[ 35.71448348]])

# Example 1.3:
lr2.loss_(y, y_hat)
# Output:
80.83996294128525
\end{minted}
% ===========================(fin ex 03)         %
% ============================================== %
\newpage
% ============================================== %
% ===========================(start ex 04)       %
\chapter{Exercise 04}
\extitle{Practicing Linear Regression}
\turnindir{ex04}
\exnumber{04}
\exfiles{linear\_model.py, are\_blue\_pills\_magics.csv}
\exforbidden{sklearn}
\makeheaderfilesforbidden


% ================================= %
\section*{Objective}
% --------------------------------- %
Evaluate a linear regression model on a very small dataset, with a given hypothesis function $h$.
Manipulate the loss function $J$, plot it, and briefly analyze the plot.

% ================================= %
\section*{Instructions}
% --------------------------------- %
You can find in the \texttt{resources} folder a tiny dataset called \texttt{are\_blue\_pills\_magics.csv} which gives you the driving performance of space pilots as a function of the quantity of the "blue pills" they took before the test.
\newline
You have a description of the data in the file named \texttt{are\_blue\_pills\_magics.txt}.
As your hypothesis function $h$, you will choose:
$$
h_{\theta}(x) = \theta_0 + \theta_1x
$$
Where $x$ is the variable, and $\theta_0$ and $\theta_1$ are the coefficients of the hypothesis.
The hypothesis is a function of $x$.
\info{You are strongly encouraged to use the class you have implemented in the previous exercise}
\newpage
\noindent{Your program must:}
\begin{itemize}
  \item Read the dataset from the csv file
  \item Perform a linear regression
\end{itemize}
Then you will model the data and plot 2 different graphs:
\begin{itemize}
  \item A graph with the data and the hypothesis you get for the spacecraft piloting score versus the quantity of "blue pills" (see figure \ref{best fit for score vs micrograms})
  \begin{figure}[!h]
    \centering
    \includegraphics[scale=0.6]{assets/ex04_score_vs_bluepills.png}
    \caption{Space driving score as a function of the quantity of blue pill (in micrograms). In blue the real values and in green the predicted values.}
    \label{best fit for score vs micrograms}
  \end{figure}
  
  \item The loss function $J(\theta)$ in function of the $\theta$ values (see figure \ref{loss function qs function of theta1 and theta0}),
  \begin{figure}[!h]
    \centering
    \includegraphics[scale=0.6]{assets/ex04_J_vs_t1.png}
    \caption{Evolution of the loss function $J$ as a function of $\theta_1$ for different values of $\theta_0$.}
    \label{loss function qs function of theta1 and theta0}
  \end{figure}
   
  \item You will calculate the MSE of the hypothesis you chose (you know how to do it already).
\end{itemize}

% ================================= %
\section*{Examples}
% ================================= %
\begin{minted}[bgcolor=darcula-back,formatcom=\color{lightgrey},fontsize=\scriptsize]{python}
import pandas as pd
import numpy as np
from sklearn.metrics import mean_squared_error
from my_linear_regression import MyLinearRegression as MyLR

data = pd.read_csv("are_blue_pills_magic.csv")
Xpill = np.array(data['Micrograms']).reshape(-1,1)
Yscore = np.array(data['Score']).reshape(-1,1)

linear_model1 = MyLR(np.array([[89.0], [-8]]))
linear_model2 = MyLR(np.array([[89.0], [-6]]))
Y_model1 = linear_model1.predict_(Xpill)
Y_model2 = linear_model2.predict_(Xpill)

print(MyLR.mse_(Yscore, Y_model1))
# 57.60304285714282
print(mean_squared_error(Yscore, Y_model1))
# 57.603042857142825
print(MyLR.mse_(Yscore, Y_model2))
# 232.16344285714285
print(mean_squared_error(Yscore, Y_model2))
# 232.16344285714285
\end{minted}
\par
\info{
  Here, the use of scikit learn is to ensure that our code is performing as expected. The use of scikit learn is forbidden in the code you will turn-in.
}
\hint{
There is no method named \texttt{.mse\_} in sklearn's LinearRegression class, but there is also a method named \texttt{.score}.
The \texttt{.score} method corresponds to the $R^2$ score.
The metric MSE is available in the \texttt{sklearn.metrics} module.
}
% ===========================(fin ex 04)         %
% ============================================== %
\newpage
% ============================================== %
% ===========================(start ex 05)       %
\chapter{Exercise 05}
%******************************************************************************%
%                                                                              %
%                                 Interlude                                    %
%                         for Machine Learning module                          %
%                                                                              %
%******************************************************************************%

% ============================================== %
\section*{Interlude}
% ============================================== %
\subsection*{Vectorized Logistic Gradient}
% ---------------------------------------------- %

Given the previous logistic gradient formula, it's quite easy to produce a vectorized version of it.
Actually, you almost already implemented it on module02!\\
\\
As with the previous exercise, \textbf{the only thing you have to change is your hypothesis} 
in order to calculate your logistic gradient.\\

$$
\begin{matrix}
\nabla(J)_0 &  = &\cfrac{1}{m}\sum_{i=1}^{m}(h_{\theta}(x^{(i)}) - y^{(i)}) & \\
\nabla(J)_j & = &\cfrac{1}{m}\sum_{i=1}^{m}(h_{\theta}(x^{(i)}) - y^{(i)})x_{j}^{(i)} & \text{ for j = 1, ..., n}    
\end{matrix}
$$

% ============================================== %
\subsection*{Vectorized Version}
% ---------------------------------------------- %

Can be vectorized the same way you did before:

$$
\nabla(J) = \cfrac{1}{m} X'^T(h_\theta(X) - y)
$$  

\newpage
\extitle{Normalization I: Z-score Standardization}
\turnindir{ex05}
\exnumber{05}
\exfiles{z\_score.py}
\exforbidden{None}
\makeheaderfilesforbidden



% ================================= %
\section*{Objective}
% --------------------------------- %
Introduction to standardization/normalization methods.\\
\\
You must implement the following formula as a function:
$$
\begin{matrix}
 x'^{(i)} = \frac{x^{(i)} - \frac{1}{m} \sum_{i = 1}^{m} x^{(i)}}{\sqrt{\frac{1}{m - 1} \sum_{i = 1}^{m} (x^{(i)} - \frac{1}{m} \sum_{i = 1}^{m} x^{(i)})^{2}}} & &\text{ for $i$ in $1, ..., m$} 
\end{matrix}
$$
Where:
\begin{itemize}
  \item $x$ is a vector of dimension $m$
  \item $x^{(i)}$ is the i$^\text{th}$ component of the $x$ vector
  \item $x'$ is the normalized version of the $x$ vector
\end{itemize}
\noindent{The equation is much easier to understand in the following form:}
$$
\begin{matrix}
x'^{(i)} = \frac{x^{(i)} - \mu}{\sigma} & &\text{ for $i$ in $1, ..., m$}
\end{matrix}
$$
This should remind you of something from \textbf{TinyStatistician}... doesn't it?!
\\
Ok, let's do a quick recap !
\begin{itemize}
  \item $\mu$ is the mean of $x$
  \item $\sigma$ is the standard deviation of $x$
\end{itemize}

% ================================= %
\section*{Instructions}
% --------------------------------- %
\noindent{In the \texttt{zscore.py} file, write the \texttt{zscore} function as per the instructions given below:}
\\
\begin{minted}[bgcolor=darcula-back,formatcom=\color{lightgrey},fontsize=\scriptsize]{python}
def zscore(x):
	"""Computes the normalized version of a non-empty numpy.ndarray using the z-score standardization.
	Args:
		x: has to be an numpy.ndarray, a vector.
	Returns:
		x' as a numpy.ndarray. 
		None if x is a non-empty numpy.ndarray or not a numpy.ndarray.
	Raises:
		This function shouldn't raise any Exception.
	"""
	... Your code ...
\end{minted}


% ================================= %
\section*{Examples}
% --------------------------------- %
\begin{minted}[bgcolor=darcula-back,formatcom=\color{lightgrey},fontsize=\scriptsize]{python}
# Example 1:
X = numpy.array([0, 15, -9, 7, 12, 3, -21])
zscore(X)
# Output:
array([-0.08620324,  1.2068453 , -0.86203236,  0.51721942,  0.94823559,
		0.17240647, -1.89647119])

# Example 2:
Y = np.array([2, 14, -13, 5, 12, 4, -19]).reshape((-1, 1))
zscore(Y)
# Output:
array([ 0.11267619,  1.16432067, -1.20187941,  0.37558731,  0.98904659,
		0.28795027, -1.72770165])
\end{minted}

% ===========================(fin ex 05)         %
% ============================================== %
\newpage
% ============================================== %
% ===========================(start ex 06)       %
\chapter{Exercise 06}
\extitle{Normalization II: Min-max Standardization}
\turnindir{ex06}
\exnumber{06}
\exfiles{minmax.py}
\exforbidden{None}
\makeheaderfilesforbidden


% ================================= %
\section*{Objective}
% --------------------------------- %
Introduction to standardization/normalization methods.
Implement another normalization method.\\
\\
You must implement the following formula as a function: 

$$
\begin{matrix}
  x'^{(i)} = \frac{x^{(i)} - min(x)}{max(x) - min(x)} & & \text{ for $i = 1, ..., m$}
\end{matrix}
$$
Where:
\begin{itemize}
  \item $x$ is a vector of dimension $m$
  \item $x^{(i)}$ is the i$^\text{th}$ component of vector $x$
  \item $min(x)$ is the minimum value found among the components of vector $x$
  \item $max(x)$ is the maximum value found among the components of vector $x$
\end{itemize}
You will notice that this min-max standardization doesn't scale the values to the $[-1,1]$ range.
What do you think the final range will be?
\newpage
% ================================= %
\section*{Instructions}
% --------------------------------- %
In the \texttt{minmax.py} file, create the \texttt{minmax} function as per the instructions given below:\\
\begin{minted}[bgcolor=darcula-back,formatcom=\color{lightgrey},fontsize=\scriptsize]{python}
def minmax(x):
	"""Computes the normalized version of a non-empty numpy.ndarray using the min-max standardization.
	Args:
		x: has to be an numpy.ndarray, a vector.
	Returns:
		x' as a numpy.ndarray. 
		None if x is a non-empty numpy.ndarray or not a numpy.ndarray.
	Raises:
		This function shouldn't raise any Exception.
	"""
    ... Your code ...
\end{minted}

% ================================= %
\section*{Examples}
% --------------------------------- %
\begin{minted}[bgcolor=darcula-back,formatcom=\color{lightgrey},fontsize=\scriptsize]{python}
# Example 1:
X = np.array([0, 15, -9, 7, 12, 3, -21]).reshape((-1, 1))
minmax(X)
# Output:
array([0.58333333, 1.        , 0.33333333, 0.77777778, 0.91666667,
		0.66666667, 0.        ])

# Example 2:
Y = np.array([2, 14, -13, 5, 12, 4, -19]).reshape((-1, 1))
minmax(Y)
# Output:
array([0.63636364, 1.        , 0.18181818, 0.72727273, 0.93939394,
		0.6969697 , 0.        ])
\end{minted}
% ===========================(fin ex 06)         %
% ============================================== %
\newpage
% ============================================== %
% ========================== Closing Chapter     %
\chapter{Conclusion - What you have learnt}

You are now done with module01, congratulations!

Based on all the notions and problems tackled today, you should be able to discuss and answer the following questions:
\begin{enumerate}
  \item What is a hypothesis and what is its goal?  
  \item What is the loss function and what does it represent?   
  \item What is Linear Gradient Descent and what does it do?  
  (hint: you have to talk about J, its gradient and the theta parameters...)  
  \item What happens if you choose a learning rate that is too large?
  \item What happens if you choose a very small learning rate, but still a sufficient number of cycles?
  \item Can you explain MSE and what it measures?
\end{enumerate}

% ========================= end Closing Chapter  %
% ============================================== %
\newpage
\section*{Contact}
% --------------------------------- %
You can contact 42AI by email: \href{mailto:contact@42ai.fr}{contact@42ai.fr}\\
\newline
Thank you for attending 42AI's Machine Learning Bootcamp !

% ================================= %
\section*{Acknowledgements}
% --------------------------------- %
The Python \& ML bootcamps are the result of a collective effort. We would like to thank:\\
\begin{itemize}
  \item Maxime Choulika (cmaxime),
  \item Pierre Peigné (ppeigne),
  \item Matthieu David (mdavid),
  \item Quentin Feuillade--Montixi (qfeuilla, quentin@42ai.fr)
  \item Mathieu Perez (maperez, mathieu.perez@42ai.fr)
\end{itemize}
who supervised the creation and enhancements of the present transcription.\\
\begin{itemize}
  \item Louis Develle (ldevelle, louis@42ai.fr)
  \item Owen Roberts (oroberts)
  \item Augustin Lopez (aulopez)
  \item Luc Lenotre (llenotre)
  \item Amric Trudel (amric@42ai.fr)
  \item Benjamin Carlier (bcarlier@student.42.fr)
  \item Pablo Clement (pclement@student.42.fr)
  \item Amir Mahla (amahla, amahla@42ai.fr)
\end{itemize}
for your investment in the creation and development of these modules.\\
\begin{itemize}
    \item All prior participants who took a moment to provide their feedbacks, and help us improve these bootcamps !
  \end{itemize}

\vfill
\doclicenseThis


\end{document}
