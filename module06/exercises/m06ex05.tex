\chapter{Exercise 05}
%******************************************************************************%
%                                                                              %
%                                 Interlude                                    %
%                         for Machine Learning module                          %
%                                                                              %
%******************************************************************************%

\section*{Interlude - Normalization}

The values inside the $x$ vector can vary quite a lot in magnitude,
depending on the type of data you are working with.
For example, if your dataset contains distances between planets in km, the numbers will be huge.
On the other hand, if you are working with planet masses expressed as a fraction of the solar system's total mass, the numbers will be very small (between 0 and 1).
Both cases may slow down convergence in Gradient Descent (or even sometimes prevent convergence at all).
To avoid that kind of situation, normalization is a very effective way to proceed.


The idea behind this technique is straightforward: \textbf{scaling the data}.  


With normalization, you can transform your $x$ vector into a new $x'$ vector whose values range between $[-1, 1]$ more or less. Doing this allows you to see much more easily how a training example compares to the other ones:
\begin{itemize}
    \item If an $x'$ value is close to $1$, you know it's among the largest in the dataset
    \item If an $x'$ value is close to $0$, you know it's close to the median
    \item If an $x'$ value is close to $-1$, you know it's among the smallest
\end{itemize}

So with the upcoming normalization techniques, you'll be able to map your data to two different value ranges: $[0, 1]$ or $[-1, 1]$. Your algorithm will like it and thank you for it.  

\newpage
\extitle{Normalization I: Z-score Standardization}
\turnindir{ex05}
\exnumber{05}
\exfiles{z\_score.py}
\exforbidden{None}
\makeheaderfilesforbidden



% ================================= %
\section*{Objective}
% --------------------------------- %
Introduction to standardization/normalization methods.\
You must implement the following formula as a function:
$$
\begin{matrix}
 x'^{(i)} = \frac{x^{(i)} - \frac{1}{m} \sum_{i = 1}^{m} x^{(i)}}{\sqrt{\frac{1}{m - 1} \sum_{i = 1}^{m} (x^{(i)} - \frac{1}{m} \sum_{i = 1}^{m} x^{(i)})^{2}}} & &\text{ for $i$ in $1, ..., m$} 
\end{matrix}
$$

Where:
\begin{itemize}
  \item $x$ is a vector of dimension $m$,
  \item $x^{(i)}$ is the i$^\text{th}$ component of the $x$ vector,
  \item $x'$ is the normalized version of the $x$ vector.
\end{itemize}

The equation is much easier to understand in the following form:

$$
\begin{matrix}
x'^{(i)} = \frac{x^{(i)} - \mu}{\sigma} & &\text{ for $i$ in $1, ..., m$}
\end{matrix}
$$

This should remind you something from \textbf{TinyStatistician}...

Nope?

Ok let's do a quick recap:
\begin{itemize}
  \item $\mu$ is the mean of $x$,
  \item $\sigma$ is the standard deviation of $x$.
\end{itemize}

% ================================= %
\section*{Instructions}
% --------------------------------- %
In the \texttt{zscore.py} file, write the \texttt{zscore} function as per the instructions given below:

\begin{minted}[bgcolor=darcula-back,formatcom=\color{lightgrey},fontsize=\scriptsize]{python}
def zscore(x):
	"""Computes the normalized version of a non-empty numpy.ndarray using the z-score standardization.
	Args:
		x: has to be an numpy.ndarray, a vector.
	Returns:
		x' as a numpy.ndarray. 
		None if x is a non-empty numpy.ndarray or not a numpy.ndarray.
	Raises:
		This function shouldn't raise any Exception.
	"""
	... Your code ...
\end{minted}


% ================================= %
\section*{Examples}
% --------------------------------- %
\begin{minted}[bgcolor=darcula-back,formatcom=\color{lightgrey},fontsize=\scriptsize]{python}
# Example 1:
X = numpy.array([0, 15, -9, 7, 12, 3, -21])
zscore(X)
# Output:
array([-0.08620324,  1.2068453 , -0.86203236,  0.51721942,  0.94823559,
		0.17240647, -1.89647119])

# Example 2:
Y = np.array([2, 14, -13, 5, 12, 4, -19]).reshape((-1, 1))
zscore(Y)
# Output:
array([ 0.11267619,  1.16432067, -1.20187941,  0.37558731,  0.98904659,
		0.28795027, -1.72770165])
\end{minted}
