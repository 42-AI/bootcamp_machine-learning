\chapter{Exercise 01}
\extitle{TinyStatistician}
\turnindir{ex01}
\exnumber{01}
\exfiles{TinyStatistician.py}
\exforbidden{Any functions which calculates mean, median, quartiles, percentiles, variance or standard deviation}
\makeheaderfilesforbidden

% ================================= %
\section*{Objective}
% --------------------------------- %
These exercises are key assignments from the last bootcamp. If you haven't completed them yet, now is the time, as they will be essential to your success in this module !

% ================================= %
\section*{Instructions}
% --------------------------------- %
Create a class named \texttt{TinyStatistician} with the following methods.\\
\newline
All methods take a list or a numpy.array as first parameter.
You have to protect your functions against input errors.

\begin{itemize}
  \item \texttt{mean(x)}: computes the mean of a given non-empty list or array $x$, using a for-loop.\\
        The method returns the mean as a float, otherwise None if $x$ is an empty list or array,
        or a non-expected type object.\\
        This method should not raise any Exception.\\
        \newline
        Given a vector $x$ of dimension m * 1, the mathematical formula of its mean is:
        $$
        \bar{x} = \frac{\sum_{i = 1}^{m}{x_i}}{m}
        $$

  \item \texttt{median(x)}: computes the median, which is also the 50th percentile, of a given non-empty list or array $x$ .\\
        This method returns the median as a float,
        otherwise None if $x$ is an empty list or array or a non-expected type object.\\
        This method should not raise any Exception.\\

  \item \texttt{quartile(x)}: computes the 1$^\text{st}$ and 3$^\text{rd}$ quartiles,
        also called the 25$^\text{th}$ percentile and the 75$^\text{th}$ percentile, of a given non-empty list or array $x$.\\
        This method returns the quartiles as a list of 2 floats,
        otherwise None if $x$ is an empty list or array or a non-expected type object.\\
        This method should not raise any Exception.\\

  \item \texttt{percentile(x, p)}: computes the expected percentile of a given non-empty list or array $x$.\\
        This method returns the percentile as a float,
        otherwise None if $x$ is an empty list or array or a non-expected type object.\\
        The second parameter is the demanded percentile.\\
        This method should not raise any Exception.\\

  \item \texttt{var(x)}: computes the sample variance of a given non-empty list or array $x$.\\
        This method returns the sample variance as a float,
        otherwise None if $x$ is an empty list or array or a non-expected type object.\\
        This method should not raise any Exception.\\
        \newline
        Given a vector $x$ of dimension m * 1 representing a population sample, the mathematical formula of its variance is:
        $$
        \sigma^2 = \frac{\sum_{i = 1}^{m}{(x_i - \bar{x})^2}}{m - 1} = \frac{\sum_{i = 1}^{m}{[x_i - (\frac{1}{m}\sum_{j = 1}^{m}{x_j}})]^2}{m - 1}
        $$

  \item \texttt{std(x)}: computes the sample standard deviation of a given non-empty list or array $x$.\\
        The method returns the sample standard deviation as a float,
        otherwise None if $x$ is an empty list or array or a non-expected type object.\\
        This method should not raise any Exception.\\
        \newline
        Given a vector $x$ of dimension m * 1, the mathematical formula of the sample's standard deviation is:
        $$
        \sigma = \sqrt{\frac{\sum_{i = 1}^{m}{(x_i - \bar{x})^2}}{m - 1}} = \sqrt{\frac{\sum_{i = 1}^{m}{[x_i - (\frac{1}{m}\sum_{j = 1}^{m}{x_j}})]^2}{m - 1}}
        $$
\end{itemize}

% ================================= %
\section*{Examples}
% --------------------------------- %

\begin{minted}[bgcolor=darcula-back,formatcom=\color{lightgrey},fontsize=\scriptsize]{python}  
  a = [1, 42, 300, 10, 59]
  TinyStatistician().mean(a)
  # Output:
  82.4

  TinyStatistician().median(a)
  # Output:
  42.0

  TinyStatistician().quartile(a)
  # Output:
  [10.0, 59.0]

  TinyStatistician().percentile(a, 10)
  # Output:
  4.6

  TinyStatistician().percentile(a, 15)
  # Output:
  6.4

  TinyStatistician().percentile(a, 20)
  # Output:
  8.2

  TinyStatistician().var(a)
  # Output:
  15349.3

  TinyStatistician().std(a)
  # Output:
  123.89229193133849
\end{minted}

\info{
  numpy uses a different definition of percentile, it does linear interpolation between the two closest list element to the percentile.
  Make sure to understand the difference between the population and the sample definition for the statistic metrics.
}
