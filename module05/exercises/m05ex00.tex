\chapter{Exercise 00}
\extitle{The Matrix}
\turnindir{ex00}
\exnumber{00}
\exfiles{matrix.py, test.py}
\exforbidden{Numpy}
\makeheaderfilesforbidden

% ================================== %
\section*{Objective}
% ---------------------------------- %
Manipulation and understanding of basic matrix operations.\\
In this exercise, you have to create a \texttt{Matrix} and a \texttt{Vector} class.
The goal is to have matrices and be able to perform both matrix-matrix operation
and matrix-vector operations with them.

% ================================== %
\section*{Instructions}
% ---------------------------------- %
You will provide a testing file to prove that your classes works as expected.\\

Your \texttt{Matrix} class must have the following 2 attributes: 
\begin{itemize}
  \item \texttt{data}: list of lists
  \item \texttt{shape}: the dimensions of the matrix as a tuple (rows, columns)
\end{itemize}

You should be able to initialize the object with either:
\begin{itemize}
  \item the elements of the matrix as a list of lists: \texttt{Matrix([[1.0, 2.0], [3.0, 4.0]])}
  \item a shape: \texttt{Matrix((3, 3))} (the matrix will be filled with zeros by default)
\end{itemize}

You will implement all the following built-in functions (called \texttt{magic/special methods}) for your \texttt{Matrix} class:

\begin{minted}[bgcolor=darcula-back,formatcom=\color{lightgrey},fontsize=\scriptsize]{python}
    # add : only matrices of same dimensions.
    __add__
    __radd__
    # sub : only matrices of same dimensions.
    __sub__
    __rsub__
    # div : only scalars.
    __truediv__
    __rtruediv__
    # mul : scalars, vectors and matrices , can have errors with vectors and matrices, 
    # returns a Vector if we perform Matrix * Vector mutliplication.
    __mul__
    __rmul__
    __str__
    __repr__
\end{minted}

You will also implement: 
\begin{itemize}
  \item a \texttt{.T()} method which returns the transpose of the matrix (see examples below),
\end{itemize}


Then, you must create a \texttt{Vector} class that inherit from Matrix.
At initialization, you must check that a column or a row vector is passed as the data argument.
If not, you must send an error message :

\begin{minted}[bgcolor=darcula-back,formatcom=\color{lightgrey},fontsize=\scriptsize]{python}
  v1 = Vector([[1, 2, 3]]) # create a row vector
  v2 = Vector([[1], [2], [3]]) # create a column vector
  v3 = Vector([[1, 2], [3, 4]]) # return an error
\end{minted}

\par
For \texttt{Vector}, you must implement:
\begin{itemize}
  \item a \texttt{.dot(self, v: Vector)} method which returns the dot product between the current vector and v. If shapes don't match, you must properly display errors.
\end{itemize}

\warn{
  Caution: when you do operations between Vector, it must return a Vector and not a Matrix 
}
\hint{
  type(self)
}

\section*{Examples}

\begin{minted}[bgcolor=darcula-back,formatcom=\color{lightgrey},fontsize=\scriptsize]{python}
m1 = Matrix([[0.0, 1.0], [2.0, 3.0], [4.0, 5.0]])
m1.shape
# Output:
(3, 2)

m1.T()
# Output:
Matrix([[0., 2., 4.], [1., 3., 5.]])

m1.T().shape
# Output:
(2, 3)
\end{minted}
\begin{minted}[bgcolor=darcula-back,formatcom=\color{lightgrey},fontsize=\scriptsize]{python}
m1 = Matrix([[0., 2., 4.], [1., 3., 5.]])
m1.shape
# Output:
(2, 3)

m1.T()
# Output:
Matrix([[0.0, 1.0], [2.0, 3.0], [4.0, 5.0]])

m1.T().shape
# Output:
(3, 2)
\end{minted}
\begin{minted}[bgcolor=darcula-back,formatcom=\color{lightgrey},fontsize=\scriptsize]{python}
m1 = Matrix([[0.0, 1.0, 2.0, 3.0], 
             [0.0, 2.0, 4.0, 6.0]])

m2 = Matrix([[0.0, 1.0],
             [2.0, 3.0],
             [4.0, 5.0],
             [6.0, 7.0]])

m1 * m2
# Output:
Matrix([[28., 34.], [56., 68.]])
\end{minted}
\begin{minted}[bgcolor=darcula-back,formatcom=\color{lightgrey},fontsize=\scriptsize]{python}
m1 = Matrix([[0.0, 1.0, 2.0],
             [0.0, 2.0, 4.0]])
v1 = Vector([[1], [2], [3]])

m1 * v1
# Output:
Matrix([[8], [16]])
# Or: Vector([[8], [16]
\end{minted}
\begin{minted}[bgcolor=darcula-back,formatcom=\color{lightgrey},fontsize=\scriptsize]{python}
v1 = Vector([[1], [2], [3]])
v2 = Vector([[2], [4], [8]])

v1 + v2
# Output:
Vector([[3],[6],[11]])
\end{minted}

\newpage

% ================================== %
\section*{Mathematical notions}
% ---------------------------------- %
% ----------------------- %
\subsection*{Matrix - vector operations}
% ----------------------- %
\begin{itemize}
  \item Multiplication between a $(m \times n)$ matrix and a vector of dimension $n$
\end{itemize}

\begin{equation*}
  X y = 
  \begin{bmatrix}
    x^{(1)}_{1} & \dots& x^{(1)}_n \\ 
    \vdots & \ddots & \vdots \\ 
    x^{(m)}_1 & \dots & x^{(m)}_n
  \end{bmatrix}  
  \cdot
  \begin{bmatrix} 
    y_1 \\
    \vdots \\ 
    y_n 
  \end{bmatrix} 
  = 
  \begin{bmatrix}
    x^{(1)} \cdot y \\
    \vdots \\
    x^{(m)} \cdot y
  \end{bmatrix}
\end{equation*}


In other words:

$$
X y =
\begin{bmatrix}
  \sum_{i = 1}^{n} x_{i}^{(1)} \cdot y_i \\
  \vdots \\
 \sum_{i = 1}^{n} x_{i}^{(m)} \cdot y_i
\end{bmatrix}
$$

% ----------------------- %
\subsection*{Matrix - matrix operations}
% ----------------------- %
\begin{itemize}
  \item Addition between two matrices of same dimension $(m \times n)$,
\end{itemize}

$$
X + Y = 
\begin{bmatrix}
  x_{1}^{(1)} & \dots & x_{n}^{(1)} \\
  \vdots & \ddots & \vdots \\ 
  x_{1}^{(m)} & \dots & x_{n}^{(m)} 
\end{bmatrix} +  
\begin{bmatrix}
  y_{1}^{(1)} & \dots & y_{n}^{(1)}  \\
  \vdots & \ddots & \vdots \\
  y_{1}^{(m)} & \dots & y_{n}^{(m)} 
\end{bmatrix} = 
\begin{bmatrix}
  x_{1}^{(1)} + y_{1}^{(1)}  & \dots & x_{n}^{(1)} + y_{n}^{(1)}  \\
  \vdots & \ddots & \vdots \\
  x_{1}^{(m)} + y_{1}^{(m)} & \dots & x_{n}^{(m)} + y_{n}^{(m)}
\end{bmatrix}
$$

\begin{itemize}
  \item Substraction between two matrices of same dimension $(m \times n)$,
\end{itemize}

$$
X - Y = 
\begin{bmatrix} 
x_{1}^{(1)} & \dots & x_{n}^{(1)}  \\ 
\vdots & \ddots & \vdots \\ 
x_{1}^{(m)} & \dots & x_{n}^{(m)} 
\end{bmatrix} - 
\begin{bmatrix} 
y_{1}^{(1)} & \dots & y_{n}^{(1)}  \\ 
\vdots & \ddots & \vdots \\ 
y_{1}^{(m)} & \dots & y_{n}^{(m)} 
\end{bmatrix} = 
\begin{bmatrix} 
x_{1}^{(1)} - y_{1}^{(1)}  & \dots & x_{n}^{(1)} - y_{n}^{(1)}  \\ 
\vdots & \ddots & \vdots \\ 
x_{1}^{(m)} - y_{1}^{(m)} & \dots & x_{n}^{(m)} - y_{n}^{(m)}
\end{bmatrix}
$$

\begin{itemize}
  \item Multiplication or division between one matrix $(m \times n)$ and one scalar,
\end{itemize}

$$
\alpha X = 
\alpha \begin{bmatrix} 
x_{1}^{(1)} & \dots & x_{n}^{(1)}  \\ 
\vdots & \ddots & \vdots \\ 
x_{1}^{(m)} & \dots & x_{n}^{(m)} 
\end{bmatrix} 
= 
\begin{bmatrix} 
\alpha x_{1}^{(1)}  & \dots & \alpha x_{n}^{(1)}  \\ 
\vdots & \ddots & \vdots \\ 
\alpha x_{1}^{(m)} & \dots & \alpha x_{n}^{(m)}
\end{bmatrix}
$$

\newpage

\begin{itemize}
  \item Mutiplication between two matrices of compatible dimension: $(m \times n)$ and $(n \times p)$,
\end{itemize}

$$
X  Y = 
\begin{bmatrix} 
x_{1}^{(1)} & \dots & x_{n}^{(1)}  \\ 
\vdots & \ddots & \vdots \\ 
x_{1}^{(m)} & \dots & x_{n}^{(m)} 
\end{bmatrix}  
\begin{bmatrix} 
y_{1}^{(1)} & \dots & y_{p}^{(1)}  \\ 
\vdots & \ddots & \vdots \\ 
y_{1}^{(n)} & \dots & y_{p}^{(n)} 
\end{bmatrix} = 
\begin{bmatrix} 
x^{(1)} \cdot y_1  & \dots & x^{(1)} \cdot y_{p} \\ 
\vdots & \ddots & \vdots \\ 
x^{(m)} \cdot y_1 & \dots & x^{(m)} \cdot y_{p}
\end{bmatrix}
$$

In other words:

$$
X Y = 
\begin{bmatrix} 
\sum_{i = 1}^{n} x_{i}^{(1)} \cdot y_{1}^{(i)} & \dots & \sum_{i=1}^{n} x_{i}^{(1)} \cdot y_{p}^{(i)} \\
\vdots & \ddots & \vdots \\ 
\sum_{i = 1}^{n} x_{i}^{(m)} \cdot y_{1}^{(i)} & \dots & \sum_{i=1}^{n} x_{i}^{(m)} \cdot y_{p}^{(i)} \\
\end{bmatrix}
$$  

Don't forget to handle all kinds of errors properly!\textit{You are expected to demonstrate the errors management between
Matrix object and Vector object and scalar plus the functionality of the accepted operations}. 