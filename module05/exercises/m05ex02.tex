\chapter{Exercise 02}
%******************************************************************************%
%                                                                              %
%                                 Interlude                                    %
%                         for Machine Learning module                          %
%                                                                              %
%******************************************************************************%

% =============================== %
\section*{Interlude - Evaluate}
% ------------------------------- %

\begin{figure}[!h]
    \centering
    \includegraphics[scale=0.2]{assets/Evaluate.png}
    %\caption{The Learning Cycle: Evaluate}
\end{figure}

% =============================== %
\section*{Back to the Loss Function}
% ------------------------------- %
How is our model doing?\\
To evaluate our model, remember that we have already used a \textbf{metric} called the \textbf{loss function} (also known as \textbf{cost function}).
The loss function is basically just a measure of how wrong the model is, in all of its predictions.\\
\newline
Two modules ago, we defined the loss function as the average of the squared distances between each prediction and its expected value (distances represented by the dotted lines in the figure below):

\begin{figure}[!h]
    \centering
    \includegraphics[scale=0.5]{assets/bad_pred_with_distance.png}
    \caption{Distances between predicted and expected values}
\end{figure}
\newpage
\noindent{The formula was the following:}

$$
J(\theta) = \frac{1}{2m}\sum_{i=1}^{m}(\hat{y}^{(i)} - y^{(i)})^2
$$
\\
And its vectorized form:

$$
\begin{matrix}
J(\theta) = \frac{1}{2m}(\hat{y} - y)\cdot(\hat{y}- y)
\end{matrix}
$$
\\
\textit{So, now that we moved to multivariate linear regression, what does it change?}\\
\newline
You may have noticed that variables such as $x_j$ and $\theta_j$ are not in the equation.
Indeed, the loss function only uses the predictions ($\hat{y}$) and the expected values ($y$), 
so the inner workings of the model do not have an impact on its evaluation metric.\\
\\
This means we can use the exact same loss function as we did before!

\newpage
\extitle{Simple Prediction}
\turnindir{ex02}
\exnumber{02}
\exfiles{prediction.py}
\exforbidden{any functions which performs prediction}
\makeheaderfilesforbidden

% ================================= %
\section*{Objective}
% --------------------------------- %
Understand and manipulate the notion of hypothesis in machine learning.

You must implement the following formula as a function:  
$$
\begin{matrix}
\hat{y}^{(i)} = \theta_0 + \theta_1 x^{(i)} & &\text{ for i = 1, ..., m}
\end{matrix}
$$  

Where:
\begin{itemize}
  \item $x$ is a vector of dimension $m$, the vector of examples/features (without the $y$ values)
  \item $\hat{y}$ is a vector of dimension m * 1, the vector of predicted values
  \item $\theta$ is a vector of dimension 2 * 1, the vector of parameters
  \item $y^{(i)}$ is the $i^{th}$ component of vector $y$
  \item $x^{(i)}$ is the $i^{th}$ component of vector $x$
\end{itemize}

% ================================= %
\section*{Instructions}
% --------------------------------- %
In the prediction.py file, write the following function as per the instructions given below:

\begin{minted}[bgcolor=darcula-back,formatcom=\color{lightgrey},fontsize=\scriptsize]{python}
def simple_predict(x, theta):
    """Computes the vector of prediction y_hat from two non-empty numpy.ndarray.
    Args:
      x: has to be an numpy.ndarray, a one-dimensional array of size m.
      theta: has to be an numpy.ndarray, a one-dimensional array of size 2.
    Returns:
      y_hat as a numpy.ndarray, a one-dimensional array of size m.
      None if x or theta are empty numpy.ndarray.
      None if x or theta dimensions are not appropriate.
    Raises:
      This function should not raise any Exception.
    """
    ... Your code ...
\end{minted}

% ================================= %
\section*{Examples}
% --------------------------------- %
\begin{minted}[bgcolor=darcula-back,formatcom=\color{lightgrey},fontsize=\scriptsize]{python}
import numpy as np
x = np.arange(1,6)

# Example 1:
theta1 = np.array([5, 0])
simple_predict(x, theta1)
# Ouput:
array([5., 5., 5., 5., 5.])
# Do you understand why y_hat contains only 5s here?  


# Example 2:
theta2 = np.array([0, 1])
simple_predict(x, theta2)
# Output:
array([1., 2., 3., 4., 5.])
# Do you understand why y_hat == x here?  


# Example 3:
theta3 = np.array([5, 3])
simple_predict(x, theta3)
# Output:
array([ 8., 11., 14., 17., 20.])


# Example 4:
theta4 = np.array([-3, 1])
simple_predict(x, theta4)
# Output:
array([-2., -1.,  0.,  1.,  2.])  
\end{minted}