\chapter{Exercise 09}
\extitle{Other loss functions}
\turnindir{ex09}
\exnumber{09}
\exfiles{other\_losses.py}
\exforbidden{None}
\makeheaderfilesforbidden

Deepen the notion of loss function in machine learning.

You certainly had a lot of fun implementing your loss function.
Remember we told you it was \textbf{one among many possible ways of measuring the loss}.
Now, you will get to implement other metrics.  You already know about one of them: \textbf{MSE}.  
There are several more which are quite common: \textbf{RMSE}, \textbf{MAE} and \textbf{R2score}.

\newpage

% ================================= %
\section*{Objective}
% --------------------------------- %
You must implement the following formulas as functions:

$$
MSE(y, \hat{y}) = \frac{1}{m}\sum_{i=1}^{m}(\hat{y}^{(i)} - y^{(i)})^2
$$

$$
RMSE(y, \hat{y}) = \sqrt{\frac{1}{m}\sum_{i=1}^{m}(\hat{y}^{(i)} - y^{(i)})^2}
$$

$$
MAE(y, \hat{y}) = \frac{1}{m}\sum_{i=1}^{m}|{\hat{y}^{(i)} - y^{(i)}}|
$$

$$
R^2(y, \hat{y}) = 1 - \frac{\sum_{i=1}^{m}(\hat{y}^{(i)} - y^{(i)})^2}{\sum_{i=1}^{m}({y}^{(i)} - \bar{y})^2}
$$

Where:
\begin{itemize}
  \item $y$ is a vector of dimension $m$,
  \item $\hat{y}$ is a vector of dimension $m$,
  \item $y^{(i)}$ is the $i^{th}$ component of vector $y$,
  \item $\hat{y}^{(i)}$ is the $i^{th}$ component of $\hat{y}$,
  \item $\bar{y}$ is the mean of the $y$ vector
\end{itemize}

\newpage

% ================================= %
\section*{Instructions}
% --------------------------------- %
In the \texttt{other\_losses.py} file, create the following functions: MSE, RMSE, MAE, R2score, as per the instructions given below:

\begin{minted}[bgcolor=darcula-back,formatcom=\color{lightgrey},fontsize=\scriptsize]{python}
def mse_(y, y_hat):
	"""
	Description:
		Calculate the MSE between the predicted output and the real output.
	Args:
        y: has to be a numpy.array, a two-dimensional array of shape m * 1.
        y_hat: has to be a numpy.array, a two-dimensional vector of shape m * 1.		
	Returns:
		mse: has to be a float.
		None if there is a matching dimension problem.
	Raises:
		This function should not raise any Exceptions.
	"""
		... your code here ...
\end{minted}
\begin{minted}[bgcolor=darcula-back,formatcom=\color{lightgrey},fontsize=\scriptsize]{python}
def rmse_(y, y_hat):
	"""
	Description:
		Calculate the RMSE between the predicted output and the real output.
	Args:
	      y: has to be a numpy.array, a two-dimensional array of shape m * 1.
        y_hat: has to be a numpy.array, a two-dimensional array of shape m * 1.		
	Returns:
		rmse: has to be a float.
		None if there is a matching dimension problem.
	Raises:
		This function should not raise any Exceptions.
	"""
		... your code here ...
\end{minted}
\begin{minted}[bgcolor=darcula-back,formatcom=\color{lightgrey},fontsize=\scriptsize]{python}
def mae_(y, y_hat):
	"""
	Description:
		Calculate the MAE between the predicted output and the real output.
	Args:
        y: has to be a numpy.array, a two-dimensional array of shape m * 1.
        y_hat: has to be a numpy.array, a two-dimensional array of shape m * 1.		
	Returns:
		mae: has to be a float.
		None if there is a matching dimension problem.
	Raises:
		This function should not raise any Exceptions.
	"""
		... your code here ...
\end{minted}
\begin{minted}[bgcolor=darcula-back,formatcom=\color{lightgrey},fontsize=\scriptsize]{python}
def r2score_(y, y_hat):
	"""
	Description:
		Calculate the R2score between the predicted output and the output.
	Args:
        y: has to be a numpy.array, a two-dimensional array of shape m * 1.
        y_hat: has to be a numpy.array, a two-dimensional array of shape m * 1.		
	Returns:
		r2score: has to be a float.
		None if there is a matching dimension problem.
	Raises:
		This function should not raise any Exceptions.
	"""
		... your code here ...
\end{minted}

\hint{
  You might consider implementing four more methods, similar to what you did for the loss function in exercise 07:
\begin{itemize}
  \item \texttt{mse\_elem()},
  \item \texttt{rmse\_elem()},
  \item \texttt{mae\_elem()},
  \item \texttt{r2score\_elem()}.
\end{itemize}
}

% ================================= %
\section*{Examples}
% --------------------------------- %
\begin{minted}[bgcolor=darcula-back,formatcom=\color{lightgrey},fontsize=\scriptsize]{python}
import numpy as np
from sklearn.metrics import mean_squared_error, mean_absolute_error, r2_score
from math import sqrt

# Example 1:
x = np.array([[0], [15], [-9], [7], [12], [3], [-21]])
y = np.array([[2], [14], [-13], [5], [12], [4], [-19]])

# Mean squared error
## your implementation
mse_(x,y)
## Output:
4.285714285714286
## sklearn implementation
mean_squared_error(x,y)
## Output:
4.285714285714286

# Root mean squared error
## your implementation
rmse_(x,y)
## Output:
2.0701966780270626
## sklearn implementation not available: take the square root of MSE
sqrt(mean_squared_error(x,y))
## Output:
2.0701966780270626

# Mean absolute error
## your implementation
mae_(x,y)
# Output:
1.7142857142857142
## sklearn implementation
mean_absolute_error(x,y)
# Output:
1.7142857142857142

# R2-score
## your implementation
r2score_(x,y)
## Output:
0.9681721733858745
## sklearn implementation
r2_score(x,y)
## Output:
0.9681721733858745
\end{minted}