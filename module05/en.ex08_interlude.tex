%******************************************************************************%
%                                                                              %
%                                 Interlude                                    %
%                         for Machine Learning module                          %
%                                                                              %
%******************************************************************************%

\section*{Interlude - Fifty Shades of Linear Algebra}

In the last exercise, we implemented the loss function in two subfunctions.
It worked, but it's not very pretty.
What if we could do it all in one step, with linear algebra?   

As we did with the hypothesis, we can use a vectorized equation to improve the calculations of the loss function.

So now let's look at how squaring and averaging can be performed (more or less) in a single matrix multiplication!

$$
J(\theta) = \frac{1}{2m}\sum_{i=1}^{m}(\hat{y}^{(i)} - y^{(i)})^2
$$
$$
J(\theta) = \frac{1}{2m}\sum_{i=1}^{m}[(\hat{y}^{(i)} - y^{(i)}) (\hat{y}^{(i)} - y^{(i)})]
$$

Now, if we apply the definition of the dot product:

$$
J(\theta) = \frac{1}{2m}(\hat{y} - y) \cdot(\hat{y}- y)
$$