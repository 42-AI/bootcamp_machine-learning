% vim: set ts=4 sw=4 tw=80 noexpandtab:

\documentclass{42-en}

%******************************************************************************%
%                                                                              %
%                                   Prologue                                   %
%                                                                              %
%******************************************************************************%
\usepackage[
    type={CC},
    modifier={by-nc-sa},
    version={4.0},
]{doclicense}
\usepackage{amsmath} % The amsmath package provides commands to typeset matrices with different delimiters. 
\usepackage{epigraph}
\usepackage[parfill]{parskip}
\setlength\epigraphwidth{.95\textwidth}
%****************************************************************%
%                  Re/definition of commands                     %
%****************************************************************%

\newcommand{\ailogo}[1]{\def \@ailogo {#1}}\ailogo{assets/42ai_logo.pdf}

%%  Redefine \maketitle
\makeatletter
\def \maketitle {
  \begin{titlepage}
    \begin{center}
	%\begin{figure}[t]
	  %\includegraphics[height=8cm]{\@ailogo}
	  \includegraphics[height=8cm]{assets/42ai_logo.pdf}
	%\end{figure}
      \vskip 5em
      {\huge \@title}
      \vskip 2em
      {\LARGE \@subtitle}
      \vskip 4em
    \end{center}
    %\begin{center}
	  %\@author
    %\end{center}
	%\vskip 5em
  \vfill
  \begin{center}
    \emph{\summarytitle : \@summary}
  \end{center}
  \vspace{2cm}
  %\vskip 5em
  %\doclicenseThis
  \end{titlepage}
}
\makeatother

\makeatletter
\def \makeheaderfilesforbidden
{
  \noindent
  \begin{tabularx}{\textwidth}{|X X  X X|}
    \hline
  \multicolumn{1}{|>{\raggedright}m{1cm}|}
  {\vskip 2mm \includegraphics[height=1cm]{assets/42ai_logo.pdf}} &
  \multicolumn{2}{>{\centering}m{12cm}}{\small Exercise : \@exnumber } &
  \multicolumn{1}{ >{\raggedleft}p{1.5cm}|}
%%              {\scriptsize points : \@exscore} \\ \hline
              {} \\ \hline

  \multicolumn{4}{|>{\centering}m{15cm}|}
              {\small \@extitle} \\ \hline

  \multicolumn{4}{|>{\raggedright}m{15cm}|}
              {\small Turn-in directory : \ttfamily
                $ex\@exnumber/$ }
              \\ \hline
  \multicolumn{4}{|>{\raggedright}m{15cm}|}
              {\small Files to turn in : \ttfamily \@exfiles }
              \\ \hline

  \multicolumn{4}{|>{\raggedright}m{15cm}|}
              {\small Forbidden functions : \ttfamily \@exforbidden }
              \\ \hline

%%  \multicolumn{4}{|>{\raggedright}m{15cm}|}
%%              {\small Remarks : \ttfamily \@exnotes }
%%              \\ \hline
\end{tabularx}
%% \exnotes
\exrules
\exmake
\exauthorize{None}
\exforbidden{None}
\extitle{}
\exnumber{}
}
\makeatother

%%  Syntactic highlights
\makeatletter
\newenvironment{pythoncode}{%
  \VerbatimEnvironment
  \usemintedstyle{emacs}
  \minted@resetoptions
  \setkeys{minted@opt}{bgcolor=black,formatcom=\color{lightgrey},fontsize=\scriptsize}
  \begin{figure}[ht!]
    \centering
    \begin{minipage}{16cm}
      \begin{VerbatimOut}{\jobname.pyg}}
{%[
      \end{VerbatimOut}
      \minted@pygmentize{c}
      \DeleteFile{\jobname.pyg}
    \end{minipage}
\end{figure}}
\makeatother

\usemintedstyle{native}

\begin{document}

% =============================================================================%
%                     =====================================                    %

\title{Machine Learning Bootcamp - Module 00}
\subtitle{Stepping into Machine Learning}
\author{
  Maxime Choulika (cmaxime), Pierre Peigné (ppeigne), Matthieu David (mdavid),
   Amir Mahla (amahla), Mathieu Perez (maperez)
}

\summary
{
You will start by reviewing some linear algebra and statistics.
Then, you will implement your first model and learn how to evaluate its performances.
}

\maketitle
%******************************************************************************%
%                                                                              %
%                        Section usefull ressources                            %
%                          for ML Modules                                      %
%                                                                              %
%******************************************************************************%


\chapter*{Notions and ressources}

\section*{Notions of the module}
\begin{itemize}
  \item Regularization
  \item Overfitting
  \item Regularized loss function
  \item Regularized gradient descent
  \item Regularized linear regression
  \item Regularized logistic regression
\end{itemize}

\section*{Useful Ressources}

You are recommended to use the following material: \href{https://www.coursera.org/learn/machine-learning}{Machine Learning MOOC - Stanford}\\
\newline
This series of videos is available at no cost: simply log in, select "Enroll for Free", and click "Audit" at the bottom of the pop-up window.\\
\newline
The following sections of the course are particularly relevant to today's exercises: 

\subsection*{Week 3: Classification}

\subsubsection*{Classification with logistic regression (already seen in module 03)}
\begin{itemize}
  \item Motivations
  \item Logistic regression
  \item Decision boundary
\end{itemize}

\subsubsection*{Cost function for logistic regression (already seen in module 03)}
\begin{itemize}
  \item Cost function for logistic regression
  \item Simplified Cost Function for Logistic Regression
\end{itemize}

\subsubsection*{Gradient descent for logistic regression (already seen in module 03)}
\begin{itemize}
  \item Gradient Descent Implementation
\end{itemize}

\subsubsection*{The problem of overfitting (New !!!)}
\begin{itemize}
  \item The problem of overfitting
  \item Addressing overfitting
  \item Cost function with regularization
  \item Regularized linear regression
  \item Regularized logistic regression  
\end{itemize}

\noindent{\emph{All videos above are available also on this
 \href{https://youtube.com/playlist?list=PLkDaE6sCZn6FNC6YRfRQc_FbeQrF8BwGI&feature=shared}
 {Andrew Ng's YouTube playlist}, videos 31 to 36 (already seen in module 03) and 37 to 41 (new !!!).}}
%******************************************************************************%
%                                                                              %
%                        Common Instructions                                   %
%                          for Python Projects                                 %
%                                                                              %
%******************************************************************************%

\chapter{Common Instructions}
\begin{itemize}
  \item The version of Python recommended to use is 3.7. You can
  check your Python's version with the following command: \texttt{python -V}
  
  \item The norm: during this bootcamp, it is recommended to follow the
  \href{https://www.python.org/dev/peps/pep-0008/}{PEP 8 standards}, though it is not mandatory.
  You can install \href{https://pypi.org/project/pycodestyle}{pycodestyle} or 
  \href{https://black.readthedocs.io/en/stable/}{Black}, which are convenient 
  packages to check your code.
  
  \item The function \texttt{eval} is never allowed.
  
  \item The exercises are ordered from the easiest to the hardest.
  
  \item Your exercises are going to be evaluated by someone else,
  so make sure that your variable names and function names are appropriate and civil.

  \item Your manual is the internet.

  \item If you're planning on using an AI assistant such as a LLM, make sure it is helpful 
  for you to \textbf{learn and practice}, not to provide you with hands-on solution ! Own your tool, don't let it own you.
  
  \item If you are a student from 42, you can access our Discord server 
  on \href{https://discord.com/channels/887850395697807362/887850396314398720}{42 student's associations portal} and ask your
  questions to your peers in the dedicated Bootcamp channel. 

  \item You can learn more about 42 Artificial Intelligence by visiting \href{https://42-ai.github.io}{our website}.

  \item If you find any issue or mistake in the subject please create an issue on 
  \href{https://github.com/42-AI/bootcamp_machine-learning/issues}{42AI repository on Github}.
  
  \item We encourage you to create test programs for your
  project even though this work \textbf{won't have to be
  submitted and won't be graded}. It will give you a chance
  to easily test your work and your peers’ work. You will find
  those tests especially useful during your defence. Indeed,
  during defence, you are free to use your tests and/or the
  tests of the peer you are evaluating.

\end{itemize}
\newpage
\tableofcontents
\startexercices

%                     =====================================                    %
% =============================================================================%


%******************************************************************************%
%                                                                              %
%                                   Exercises                                  %
%                                                                              %
%******************************************************************************%

% ============================================== %
% ===========================(start ex 00)       %
\chapter{Exercise 00}
\extitle{The Matrix}
\turnindir{ex00}
\exnumber{00}
\exfiles{matrix.py, test.py}
\exforbidden{Numpy}
\makeheaderfilesforbidden

% ================================== %
\section*{Objective}
% ---------------------------------- %
Basic understanding and manipulation of elementary matrix operations.\\
\\
In this exercise, you have to create a \texttt{Matrix} and a \texttt{Vector} class.\\
\\
The goal is to have matrices and to be able to perform both matrix-matrix operations
and matrix-vector operations with them.

% ================================== %
\section*{Instructions}
% ---------------------------------- %
You will provide a test file to prove that your classes work as expected.\\

\subsection*{Matrix class}

Your \texttt{Matrix} class must have the 2 following attributes: 
\begin{itemize}
  \item \texttt{data}: list of lists
  \item \texttt{shape}: the dimensions of the matrix as a tuple (rows, columns)
\end{itemize}

You should be able to initialize the object with either:
\begin{itemize}
  \item the elements of the matrix as a list of lists: \texttt{Matrix([[1.0, 2.0], [3.0, 4.0]])}
  \item a shape: \texttt{Matrix((3, 3))} (the matrix will be filled with zeros by default)
\end{itemize}

You will implement all of the following built-in functions (called \texttt{magic/special methods}) for your \texttt{Matrix} class:

\begin{minted}[bgcolor=darcula-back,formatcom=\color{lightgrey},fontsize=\scriptsize]{python}
    # add : only matrices of same dimensions.
    __add__
    __radd__
    # sub : only matrices of same dimensions.
    __sub__
    __rsub__
    # div : only scalars.
    __truediv__
    __rtruediv__
    # mul : scalars, vectors and matrices , can have errors with vectors and matrices, 
    # returns a Vector if we perform Matrix * Vector mutliplication.
    __mul__
    __rmul__
    __str__
    __repr__
\end{minted}

You will also implement: 
\begin{itemize}
  \item a \texttt{.T()} method which returns the transpose of the matrix (see examples below)
\end{itemize}

\subsection*{Vector class}

Then, you must create a \texttt{Vector} class that inherits from the Matrix class.\\

At initialization, you must check that a column or a row vector is passed as the data argument.
If not, you must send an error message :

\begin{minted}[bgcolor=darcula-back,formatcom=\color{lightgrey},fontsize=\scriptsize]{python}
  v1 = Vector([[1, 2, 3]]) # create a row vector
  v2 = Vector([[1], [2], [3]]) # create a column vector
  v3 = Vector([[1, 2], [3, 4]]) # return an error
\end{minted}

\par
For \texttt{Vector}, you must implement:
\begin{itemize}
  \item a \texttt{.dot(self, v: Vector)} method which returns the dot product between the current vector and v. If shapes don't match, you must properly handle errors.
\end{itemize}

\warn{
  Caution: when you do operations between Vector, it must return a Vector and not a Matrix 
}
\hint{
  type(self)
}

\section*{Examples}

\begin{minted}[bgcolor=darcula-back,formatcom=\color{lightgrey},fontsize=\scriptsize]{python}
m1 = Matrix([[0.0, 1.0], [2.0, 3.0], [4.0, 5.0]])
m1.shape
# Output:
(3, 2)

m1.T()
# Output:
Matrix([[0., 2., 4.], [1., 3., 5.]])

m1.T().shape
# Output:
(2, 3)
\end{minted}
\begin{minted}[bgcolor=darcula-back,formatcom=\color{lightgrey},fontsize=\scriptsize]{python}
m1 = Matrix([[0., 2., 4.], [1., 3., 5.]])
m1.shape
# Output:
(2, 3)

m1.T()
# Output:
Matrix([[0.0, 1.0], [2.0, 3.0], [4.0, 5.0]])

m1.T().shape
# Output:
(3, 2)
\end{minted}
\begin{minted}[bgcolor=darcula-back,formatcom=\color{lightgrey},fontsize=\scriptsize]{python}
m1 = Matrix([[0.0, 1.0, 2.0, 3.0], 
             [0.0, 2.0, 4.0, 6.0]])

m2 = Matrix([[0.0, 1.0],
             [2.0, 3.0],
             [4.0, 5.0],
             [6.0, 7.0]])

m1 * m2
# Output:
Matrix([[28., 34.], [56., 68.]])
\end{minted}
\begin{minted}[bgcolor=darcula-back,formatcom=\color{lightgrey},fontsize=\scriptsize]{python}
m1 = Matrix([[0.0, 1.0, 2.0],
             [0.0, 2.0, 4.0]])
v1 = Vector([[1], [2], [3]])

m1 * v1
# Output:
Matrix([[8], [16]])
# Or: Vector([[8], [16]
\end{minted}
\begin{minted}[bgcolor=darcula-back,formatcom=\color{lightgrey},fontsize=\scriptsize]{python}
v1 = Vector([[1], [2], [3]])
v2 = Vector([[2], [4], [8]])

v1 + v2
# Output:
Vector([[3],[6],[11]])
\end{minted}

\newpage

% ================================== %
\section*{Mathematical notions}
% ---------------------------------- %
% ----------------------- %
\subsection*{Matrix - vector operations}
% ----------------------- %
\begin{itemize}
  \item Multiplication between a $(m \times n)$ matrix and a vector of dimension $n$
\end{itemize}

\begin{equation*}
  X y = 
  \begin{bmatrix}
    x^{(1)}_{1} & \dots& x^{(1)}_n \\ 
    \vdots & \ddots & \vdots \\ 
    x^{(m)}_1 & \dots & x^{(m)}_n
  \end{bmatrix}  
  \cdot
  \begin{bmatrix} 
    y_1 \\
    \vdots \\ 
    y_n 
  \end{bmatrix} 
  = 
  \begin{bmatrix}
    x^{(1)} \cdot y \\
    \vdots \\
    x^{(m)} \cdot y
  \end{bmatrix}
\end{equation*}


In other words:

$$
X y =
\begin{bmatrix}
  \sum_{i = 1}^{n} x_{i}^{(1)} \cdot y_i \\
  \vdots \\
 \sum_{i = 1}^{n} x_{i}^{(m)} \cdot y_i
\end{bmatrix}
$$

% ----------------------- %
\subsection*{Matrix - matrix operations}
% ----------------------- %
\begin{itemize}
  \item Addition between two matrices of same dimension $(m \times n)$,
\end{itemize}

$$
X + Y = 
\begin{bmatrix}
  x_{1}^{(1)} & \dots & x_{n}^{(1)} \\
  \vdots & \ddots & \vdots \\ 
  x_{1}^{(m)} & \dots & x_{n}^{(m)} 
\end{bmatrix} +  
\begin{bmatrix}
  y_{1}^{(1)} & \dots & y_{n}^{(1)}  \\
  \vdots & \ddots & \vdots \\
  y_{1}^{(m)} & \dots & y_{n}^{(m)} 
\end{bmatrix} = 
\begin{bmatrix}
  x_{1}^{(1)} + y_{1}^{(1)}  & \dots & x_{n}^{(1)} + y_{n}^{(1)}  \\
  \vdots & \ddots & \vdots \\
  x_{1}^{(m)} + y_{1}^{(m)} & \dots & x_{n}^{(m)} + y_{n}^{(m)}
\end{bmatrix}
$$

\begin{itemize}
  \item Substraction between two matrices of same dimension $(m \times n)$,
\end{itemize}

$$
X - Y = 
\begin{bmatrix} 
x_{1}^{(1)} & \dots & x_{n}^{(1)}  \\ 
\vdots & \ddots & \vdots \\ 
x_{1}^{(m)} & \dots & x_{n}^{(m)} 
\end{bmatrix} - 
\begin{bmatrix} 
y_{1}^{(1)} & \dots & y_{n}^{(1)}  \\ 
\vdots & \ddots & \vdots \\ 
y_{1}^{(m)} & \dots & y_{n}^{(m)} 
\end{bmatrix} = 
\begin{bmatrix} 
x_{1}^{(1)} - y_{1}^{(1)}  & \dots & x_{n}^{(1)} - y_{n}^{(1)}  \\ 
\vdots & \ddots & \vdots \\ 
x_{1}^{(m)} - y_{1}^{(m)} & \dots & x_{n}^{(m)} - y_{n}^{(m)}
\end{bmatrix}
$$

\begin{itemize}
  \item Multiplication or division between one matrix $(m \times n)$ and one scalar,
\end{itemize}

$$
\alpha X = 
\alpha \begin{bmatrix} 
x_{1}^{(1)} & \dots & x_{n}^{(1)}  \\ 
\vdots & \ddots & \vdots \\ 
x_{1}^{(m)} & \dots & x_{n}^{(m)} 
\end{bmatrix} 
= 
\begin{bmatrix} 
\alpha x_{1}^{(1)}  & \dots & \alpha x_{n}^{(1)}  \\ 
\vdots & \ddots & \vdots \\ 
\alpha x_{1}^{(m)} & \dots & \alpha x_{n}^{(m)}
\end{bmatrix}
$$

\newpage

\begin{itemize}
  \item Mutiplication between two matrices of compatible dimension: $(m \times n)$ and $(n \times p)$,
\end{itemize}

$$
X  Y = 
\begin{bmatrix} 
x_{1}^{(1)} & \dots & x_{n}^{(1)}  \\ 
\vdots & \ddots & \vdots \\ 
x_{1}^{(m)} & \dots & x_{n}^{(m)} 
\end{bmatrix}  
\begin{bmatrix} 
y_{1}^{(1)} & \dots & y_{p}^{(1)}  \\ 
\vdots & \ddots & \vdots \\ 
y_{1}^{(n)} & \dots & y_{p}^{(n)} 
\end{bmatrix} = 
\begin{bmatrix} 
x^{(1)} \cdot y_1  & \dots & x^{(1)} \cdot y_{p} \\ 
\vdots & \ddots & \vdots \\ 
x^{(m)} \cdot y_1 & \dots & x^{(m)} \cdot y_{p}
\end{bmatrix}
$$

In other words:

$$
X Y = 
\begin{bmatrix} 
\sum_{i = 1}^{n} x_{i}^{(1)} \cdot y_{1}^{(i)} & \dots & \sum_{i=1}^{n} x_{i}^{(1)} \cdot y_{p}^{(i)} \\
\vdots & \ddots & \vdots \\ 
\sum_{i = 1}^{n} x_{i}^{(m)} \cdot y_{1}^{(i)} & \dots & \sum_{i=1}^{n} x_{i}^{(m)} \cdot y_{p}^{(i)} \\
\end{bmatrix}
$$  

% ===========================(fin ex 00)         %
% ============================================== %
\newpage
% ============================================== %
% ===========================(start ex 01)       %
\chapter{Exercise 01}
\extitle{TinyStatistician}
\turnindir{ex01}
\exnumber{01}
\exfiles{TinyStatistician.py}
\exforbidden{Any functions which calculates mean, median, quartiles, percentiles, variance or standard deviation}
\makeheaderfilesforbidden

% ================================= %
\section*{Objective}
% --------------------------------- %
These exercises are key assignments from the last bootcamp. If you haven't completed them yet, now is the time, as they will be essential to your success in this module !

% ================================= %
\section*{Instructions}
% --------------------------------- %
Create a class named \texttt{TinyStatistician} with the following methods.\\
\newline
All methods take a list or a numpy.array as first parameter.
You have to protect your functions against input errors.

\begin{itemize}
  \item \texttt{mean(x)}: computes the mean of a given non-empty list or array $x$, using a for-loop.\\
        The method returns the mean as a float, otherwise None if $x$ is an empty list or array,
        or a non-expected type object.\\
        This method should not raise any Exception.\\
        \newline
        Given a vector $x$ of dimension m * 1, the mathematical formula of its mean is:
        $$
        \bar{x} = \frac{\sum_{i = 1}^{m}{x_i}}{m}
        $$

  \item \texttt{median(x)}: computes the median, which is also the 50th percentile, of a given non-empty list or array $x$ .\\
        This method returns the median as a float,
        otherwise None if $x$ is an empty list or array or a non-expected type object.\\
        This method should not raise any Exception.\\

  \item \texttt{quartile(x)}: computes the 1$^\text{st}$ and 3$^\text{rd}$ quartiles,
        also called the 25$^\text{th}$ percentile and the 75$^\text{th}$ percentile, of a given non-empty list or array $x$.\\
        This method returns the quartiles as a list of 2 floats,
        otherwise None if $x$ is an empty list or array or a non-expected type object.\\
        This method should not raise any Exception.\\

  \item \texttt{percentile(x, p)}: computes the expected percentile of a given non-empty list or array $x$.\\
        This method returns the percentile as a float,
        otherwise None if $x$ is an empty list or array or a non-expected type object.\\
        The second parameter is the demanded percentile.\\
        This method should not raise any Exception.\\

  \item \texttt{var(x)}: computes the sample variance of a given non-empty list or array $x$.\\
        This method returns the sample variance as a float,
        otherwise None if $x$ is an empty list or array or a non-expected type object.\\
        This method should not raise any Exception.\\
        \newline
        Given a vector $x$ of dimension m * 1 representing a population sample, the mathematical formula of its variance is:
        $$
        \sigma^2 = \frac{\sum_{i = 1}^{m}{(x_i - \bar{x})^2}}{m - 1} = \frac{\sum_{i = 1}^{m}{[x_i - (\frac{1}{m}\sum_{j = 1}^{m}{x_j}})]^2}{m - 1}
        $$

  \item \texttt{std(x)}: computes the sample standard deviation of a given non-empty list or array $x$.\\
        The method returns the sample standard deviation as a float,
        otherwise None if $x$ is an empty list or array or a non-expected type object.\\
        This method should not raise any Exception.\\
        \newline
        Given a vector $x$ of dimension m * 1, the mathematical formula of the sample's standard deviation is:
        $$
        \sigma = \sqrt{\frac{\sum_{i = 1}^{m}{(x_i - \bar{x})^2}}{m - 1}} = \sqrt{\frac{\sum_{i = 1}^{m}{[x_i - (\frac{1}{m}\sum_{j = 1}^{m}{x_j}})]^2}{m - 1}}
        $$
\end{itemize}

% ================================= %
\section*{Examples}
% --------------------------------- %

\begin{minted}[bgcolor=darcula-back,formatcom=\color{lightgrey},fontsize=\scriptsize]{python}  
  a = [1, 42, 300, 10, 59]
  TinyStatistician().mean(a)
  # Output:
  82.4

  TinyStatistician().median(a)
  # Output:
  42.0

  TinyStatistician().quartile(a)
  # Output:
  [10.0, 59.0]

  TinyStatistician().percentile(a, 10)
  # Output:
  4.6

  TinyStatistician().percentile(a, 15)
  # Output:
  6.4

  TinyStatistician().percentile(a, 20)
  # Output:
  8.2

  TinyStatistician().var(a)
  # Output:
  12279.439999999999

  TinyStatistician().std(a)
  # Output:
  110.81263465868862
\end{minted}

\info{
  Numpy uses a different definition of percentile, it does linear interpolation between the two closest list element to the percentile.
  Make sure to understand the difference between the population and the sample definition for the statistic metrics.
}

% ===========================(fin ex 01)         %
% ============================================== %
\newpage
% ============================================== %
% ===========================(start ex 02)       %
\chapter{Exercise 02}
%******************************************************************************%
%                                                                              %
%                                 Interlude                                    %
%                         for Machine Learning module                          %
%                                                                              %
%******************************************************************************%

% =============================== %
\section*{Interlude - Evaluate}
% ------------------------------- %

\begin{figure}[!h]
    \centering
    \includegraphics[scale=0.2]{assets/Evaluate.png}
    %\caption{The Learning Cycle: Evaluate}
\end{figure}

% =============================== %
\section*{Back to the Loss Function}
% ------------------------------- %
How is our model doing?\\
To evaluate our model, remember that we have already used a \textbf{metric} called the \textbf{loss function} (also known as \textbf{cost function}).
The loss function is basically just a measure of how wrong the model is, in all of its predictions.\\
\newline
Two modules ago, we defined the loss function as the average of the squared distances between each prediction and its expected value (distances represented by the dotted lines in the figure below):

\begin{figure}[!h]
    \centering
    \includegraphics[scale=0.5]{assets/bad_pred_with_distance.png}
    \caption{Distances between predicted and expected values}
\end{figure}
\newpage
\noindent{The formula was the following:}

$$
J(\theta) = \frac{1}{2m}\sum_{i=1}^{m}(\hat{y}^{(i)} - y^{(i)})^2
$$
\\
And its vectorized form:

$$
\begin{matrix}
J(\theta) = \frac{1}{2m}(\hat{y} - y)\cdot(\hat{y}- y)
\end{matrix}
$$
\\
\textit{So, now that we moved to multivariate linear regression, what does it change?}\\
\newline
You may have noticed that variables such as $x_j$ and $\theta_j$ are not in the equation.
Indeed, the loss function only uses the predictions ($\hat{y}$) and the expected values ($y$), 
so the inner workings of the model do not have an impact on its evaluation metric.\\
\\
This means we can use the exact same loss function as we did before!

\newpage
\extitle{Simple Prediction}
\turnindir{ex02}
\exnumber{02}
\exfiles{prediction.py}
\exforbidden{any functions which performs prediction}
\makeheaderfilesforbidden

% ================================= %
\section*{Objective}
% --------------------------------- %
Understand and manipulate the notion of hypothesis in machine learning.

You must implement the following formula as a function:  
$$
\begin{matrix}
\hat{y}^{(i)} = \theta_0 + \theta_1 x^{(i)} & &\text{ for i = 1, ..., m}
\end{matrix}
$$  

Where:
\begin{itemize}
  \item $x$ is a vector of dimension $m$, the vector of examples/features (without the $y$ values)
  \item $\hat{y}$ is a vector of dimension m * 1, the vector of predicted values
  \item $\theta$ is a vector of dimension 2 * 1, the vector of parameters
  \item $y^{(i)}$ is the $i^{th}$ component of vector $y$
  \item $x^{(i)}$ is the $i^{th}$ component of vector $x$
\end{itemize}

% ================================= %
\section*{Instructions}
% --------------------------------- %
In the prediction.py file, write the following function as per the instructions given below:

\begin{minted}[bgcolor=darcula-back,formatcom=\color{lightgrey},fontsize=\scriptsize]{python}
def simple_predict(x, theta):
    """Computes the vector of prediction y_hat from two non-empty numpy.ndarray.
    Args:
      x: has to be an numpy.ndarray, a one-dimensional array of size m.
      theta: has to be an numpy.ndarray, a one-dimensional array of size 2.
    Returns:
      y_hat as a numpy.ndarray, a one-dimensional array of size m.
      None if x or theta are empty numpy.ndarray.
      None if x or theta dimensions are not appropriate.
    Raises:
      This function should not raise any Exception.
    """
    ... Your code ...
\end{minted}

% ================================= %
\section*{Examples}
% --------------------------------- %
\begin{minted}[bgcolor=darcula-back,formatcom=\color{lightgrey},fontsize=\scriptsize]{python}
import numpy as np
x = np.arange(1,6)

# Example 1:
theta1 = np.array([5, 0])
simple_predict(x, theta1)
# Ouput:
array([5., 5., 5., 5., 5.])
# Do you understand why y_hat contains only 5s here?  


# Example 2:
theta2 = np.array([0, 1])
simple_predict(x, theta2)
# Output:
array([1., 2., 3., 4., 5.])
# Do you understand why y_hat == x here?  


# Example 3:
theta3 = np.array([5, 3])
simple_predict(x, theta3)
# Output:
array([ 8., 11., 14., 17., 20.])


# Example 4:
theta4 = np.array([-3, 1])
simple_predict(x, theta4)
# Output:
array([-2., -1.,  0.,  1.,  2.])  
\end{minted}
% ===========================(fin ex 02)         %
% ============================================== %
\newpage
% ============================================== %
% ===========================(start ex 03)       %
\chapter{Exercise 03}
%******************************************************************************%
%                                                                              %
%                                 Interlude                                    %
%                         for Machine Learning module                          %
%                                                                              %
%******************************************************************************%

% =============================== %
\section*{Interlude - Improve with the Gradient}
% ******************************* %

\begin{figure}[!h]
    \centering
    \includegraphics[scale=0.2]{assets/Improve.png}
    %\caption{The Learning Cycle: Improve}
\end{figure}

% =============================== %
\section*{Multivariate Gradient}
% ******************************* %
From our multivariate linear hypothesis we can derive our multivariate gradient.
It looks a lot like the one we saw during the previous module, but instead of having just two components, the gradient now has as many as there are parameters.
This means that now we need to calculate $\nabla(J)_0,\nabla(J)_1,\dots,\nabla(J)_n$.\\
\newline
If we take the univariate equations we used during the previous module and replace the formula for $\nabla(J)_1$ by a more general $\nabla(J)_j$, we get the following:

$$
\begin{matrix}
\nabla(J)_0 &  = &\frac{1}{m}\sum_{i=1}^{m}(h_{\theta}(x^{(i)}) - y^{(i)}) & \\
\nabla(J)_j & = &\frac{1}{m}\sum_{i=1}^{m}(h_{\theta}(x^{(i)}) - y^{(i)})x_{j}^{(i)} & \text{ for j = 1, ..., n}    
\end{matrix}
$$
Where:
\begin{itemize}
    \item $\nabla(J)$ is a vector of dimension $(n + 1)$, the gradient vector
    \item $\nabla(J)_j$ is the j$^\text{th}$ component of $\nabla(J)$, the partial derivative of $J$ with respect to $\theta_j$
    \item $y$ is a vector of dimension $m$, the vector of expected values
    \item $y^{(i)}$ is a scalar, the i$^\text{th}$ component of vector $y$
    \item $x^{(i)}$ is the feature vector of the i$^\text{th}$ example
    \item $x^{(i)}_j$ is a scalar, the j$^\text{th}$ feature value of the i$^\text{th}$ example
    \item $h_{\theta}(x^{(i)})$ is a scalar, the model's estimation of $y^{(i)}$. (It can also be denoted $\hat{y}^{(i)}$)
\end{itemize}

% =============================== %
\section*{Vectorized Form}
% ******************************* %
As usual, we can use some linear algebra magic to get a more compact (and computationally efficient) formula.
First we can use our convention that each training example has an extra $x_0 = 1$ feature, and replace the gradient formulas above by one single equation that is valid for all $j$ components:

$$
\begin{matrix}
\nabla(J)_j & = &\frac{1}{m}\sum_{i=1}^{m}(h_{\theta}(x^{(i)}) - y^{(i)})x_{j}^{(i)} & \text{ for j = 0, ..., n}
\end{matrix}
$$
And this generic equation can then be rewritten in a vectorized form:

$$
\nabla(J) = \frac{1}{m} {X'}^T(X'\theta - y)
$$  
Where:  
\begin{itemize}
    \item $\nabla(J)$ is the gradient vector of dimension $(n + 1)$
    \item $X'$ is a matrix of dimensions $(m \times (n + 1))$, the design matrix onto which a column of $1$'s was added as the first column
    \item ${X'}^T$ means the matrix has been transposed
    \item $\theta$ is a vector of dimension $(n + 1)$: the parameter vector 
    \item $y$ is a vector of dimension $m$: the vector of expected values
\end{itemize}
The vectorized equation can output the entire gradient vector all at once, in one calculation!\\
\newline
So if you understand the linear algebra operations, you can forget about the equations we presented at the top of the page and simply use the vectorized one.

\newpage
\extitle{Add Intercept}
\turnindir{ex03}
\exnumber{03}
\exfiles{tools.py}
\exforbidden{None}
\makeheaderfilesforbidden

% ================================= %
\section*{Objective}
% --------------------------------- %

Understand and manipulate the notion of hypothesis in machine learning.
\\
You must implement a function which adds an extra column of $1$'s on the left side of a given vector or matrix.

% ================================= %
\section*{Instructions}
% --------------------------------- %
In the tools.py file create the following function as per the instructions given below:

\begin{minted}[bgcolor=darcula-back,formatcom=\color{lightgrey},fontsize=\scriptsize]{python}
def add_intercept(x):
    """Adds a column of 1's to the non-empty numpy.array x.
    Args:
      x: has to be a numpy.array. x can be a one-dimensional (m * 1) or two-dimensional (m * n) array.
    Returns:
      X, a numpy.array of dimension m * (n + 1).
      None if x is not a numpy.array.
      None if x is an empty numpy.array.
    Raises:
      This function should not raise any Exception.
    """
    ... Your code ...
\end{minted}

% ================================= %
\section*{Examples}
% --------------------------------- %

\begin{minted}[bgcolor=darcula-back,formatcom=\color{lightgrey},fontsize=\scriptsize]{python}
import numpy as np

# Example 1:
x = np.arange(1,6)
add_intercept(x)
# Output:
array([[1., 1.],
       [1., 2.],
       [1., 3.],
       [1., 4.],
       [1., 5.]])


# Example 2:
y = np.arange(1,10).reshape((3,3))
add_intercept(y)
# Output:
array([[1., 1., 2., 3.],
       [1., 4., 5., 6.],
       [1., 7., 8., 9.]])
\end{minted}


% ===========================(fin ex 03)         %
% ============================================== %
\newpage
% ============================================== %
% ===========================(start ex 04)       %
\chapter{Exercise 04}
\extitle{Prediction}
\turnindir{ex04}
\exnumber{04}
\exfiles{prediction.py}
\exforbidden{None}
\makeheaderfilesforbidden


% ================================= %
\section*{Objective}
% --------------------------------- %
Understand and manipulate the notion of hypothesis in machine learning.

You must implement the following formula as a function: 

$$
\begin{matrix}
\hat{y}^{(i)} = \theta_0 + \theta_1 x^{(i)} & &\text{ for i = 1, ..., m}
\end{matrix}
$$  

Where:
\begin{itemize}
  \item $\hat{y}^{(i)}$ is the $i^{th}$ component of vector $\hat{y}$
  \item $\hat{y}$ is a vector of dimension $m$, the vector of predicted values
  \item $\theta$ is a vector of dimension $2 \times 1$, the vector of parameters
  \item $x^{(i)}$ is the $i^{th}$ component of vector $x$  
  \item $x$ is a vector of dimension $m$, the vector of examples
\end{itemize}

But this time you have to do it with the linear algebra trick!

$$
\hat{y} = X' \cdot \theta = 
\begin{bmatrix} 
1 & x^{(1)} \\ 
\vdots & \vdots \\ 
1 & x^{(m)} 
\end{bmatrix} 
\cdot
\begin{bmatrix}
\theta_0 \\ 
\theta_1 
\end{bmatrix} 
 = \begin{bmatrix} 
\theta_0 + \theta_1x^{(1)} \\ 
\vdots \\ 
\theta_0 + \theta_1x^{(m)} 
\end{bmatrix} 
$$

\warn{
\begin{itemize}
  \item the argument $x$ is an $m$ vector
  \item $\theta$ is a $2 \times 1$ vector. 
\end{itemize}
}

You have to transform $x$ into $X'$ to fit the dimension of $\theta$!


% ================================= %
\section*{Instructions}
% --------------------------------- %
In the prediction.py file create the following function as per the instructions given below:
\newline
\begin{minted}[bgcolor=darcula-back,formatcom=\color{lightgrey},fontsize=\scriptsize]{python}
def predict_(x, theta):
    """Computes the vector of prediction y_hat from two non-empty numpy.array.
    Args:
      x: has to be an numpy.array, a one-dimensional array of size m.
      theta: has to be an numpy.array, a two-dimensional array of shape 2 * 1.
    Returns:
      y_hat as a numpy.array, a two-dimensional array of shape m * 1.
      None if x and/or theta are not numpy.array.
      None if x or theta are empty numpy.array.
      None if x or theta dimensions are not appropriate.
    Raises:
      This function should not raise any Exceptions.
    """
    ... Your code ...
\end{minted}

\section*{Examples}
\begin{minted}[bgcolor=darcula-back,formatcom=\color{lightgrey},fontsize=\scriptsize]{python}
import numpy as np
x = np.arange(1,6)

# Example 1:
theta1 = np.array([[5], [0]])
predict_(x, theta1)
# Ouput:
array([[5.], [5.], [5.], [5.], [5.]])
# Do you remember why y_hat contains only 5's here?

# Example 2:
theta2 = np.array([[0], [1]])
predict_(x, theta2)
# Output:
array([[1.], [2.], [3.], [4.], [5.]])
# Do you remember why y_hat == x here?

# Example 3:
theta3 = np.array([[5], [3]])
predict_(x, theta3)
# Output:
array([[ 8.], [11.], [14.], [17.], [20.]])


# Example 4:
theta4 = np.array([[-3], [1]])
predict_(x, theta4)
# Output:
array([[-2.], [-1.], [ 0.], [ 1.], [ 2.]])
\end{minted}

% ===========================(fin ex 04)         %
% ============================================== %
\newpage
% ============================================== %
% ===========================(start ex 05)       %
\chapter{Exercise 05}
\extitle{Let’s Make Nice Plots}
\turnindir{ex05}
\exnumber{05}
\exfiles{plot.py}
\exforbidden{None}
\makeheaderfilesforbidden

\info{
For your information, the task we are performing here is called \textbf{regression}.
It means that we are trying to predict a continuous numerical attribute for all examples (like a price, for instance).
Later in the bootcamp, you will see that we can predict other things such as categories.
}

% ================================= %
\section*{Objective}
% --------------------------------- %
You must implement a function to plot the data and the prediction line (or regression line).\\
\newline
You will plot the data points (with their x and y values), and the prediction line that represents your hypothesis ($h_{\theta}$).
\newpage
% ================================= %
\section*{Instructions}
% --------------------------------- %
In the plot.py file, create the following function as per the instructions given below:

\begin{minted}[bgcolor=darcula-back,formatcom=\color{lightgrey},fontsize=\scriptsize]{python}
def plot(x, y, theta):
    """Plot the data and prediction line from three non-empty numpy.array.
    Args:
      x: has to be an numpy.array, a one-dimensional array of size m.
      y: has to be an numpy.array, a one-dimensional array of size m.
      theta: has to be an numpy.array, a two-dimensional array of shape 2 * 1.
    Returns:
        Nothing.
    Raises:
      This function should not raise any Exceptions.
    """
    ... Your code ...
\end{minted}

% ================================= %
\section*{Examples}
% --------------------------------- %

\begin{minted}[bgcolor=darcula-back,formatcom=\color{lightgrey},fontsize=\scriptsize]{python}
import numpy as np
x = np.arange(1,6)
y = np.array([3.74013816, 3.61473236, 4.57655287, 4.66793434, 5.95585554])

# Example 1:
theta1 = np.array([[4.5],[-0.2]])
plot(x, y, theta1)
# Output:
\end{minted}

\begin{figure}[H]
  \centering
  \includegraphics[scale=0.6]{assets/plot1.png}
\end{figure}

\newpage

\begin{minted}[bgcolor=darcula-back,formatcom=\color{lightgrey},fontsize=\scriptsize]{python}
# Example 2:
theta2 = np.array([[-1.5],[2]])
plot(x, y, theta2)
# Output:
\end{minted}

\begin{figure}[H]
  \centering
  \includegraphics[scale=0.6]{assets/plot2.png}
  \caption{Example 2}
\end{figure}

\begin{minted}[bgcolor=darcula-back,formatcom=\color{lightgrey},fontsize=\scriptsize]{python}
# Example 3:
theta3 = np.array([[3],[0.3]])
plot(x, y, theta3)
# Output:
\end{minted}

\begin{figure}[H]
  \centering
  \includegraphics[scale=0.6]{assets/plot3.png}
  \caption{Example 3}
\end{figure}


% ===========================(fin ex 05)         %
% ============================================== %
\newpage
% ============================================== %
% ===========================(start ex 06)       %
\chapter{Exercise 06}
%******************************************************************************%
%                                                                              %
%                                 Interlude                                    %
%                         for Machine Learning module                          %
%                                                                              %
%******************************************************************************%

% ============================================== %
\section*{Interlude}
% ============================================== %
\subsection*{Linear Regression to the Next Level: Ridge Regression}
% ---------------------------------------------- %

Until now we only talked about L$_2$ regularization and its implication on the calculation of the loss function and gradient for both linear and logistic regression.

Now it's time to use proper terminology:
When we apply L$_2$ regularization on a linear regression model, the new model is called a \textbf{Ridge Regression} model.
Besides that brand-new name, Ridge regression is nothing more than linear regression regularized with L$_2$.

We suggest you watch this nice explanation \href{https://www.youtube.com/watch?v=Q81RR3yKn30}{very nice explanation of Ridge Regularization}.
By the way, this Youtube channel, \texttt{\textit{StatQuest}}, is very good to help you understand the gist of a lot of machine learning concepts.
You will not waste your time watching its statistics and machine learning playlists!

\newpage
\extitle{Loss function}
\turnindir{ex06}
\exnumber{06}
\exfiles{loss.py}
\exforbidden{None}
\makeheaderfilesforbidden


% ================================= %
\section*{Objective}
% --------------------------------- %
Understand and experiment with the \textbf{loss function} in machine learning.

You must implement the following formula as a function (and another one very close to it):

$$
J(\theta) = \frac{1}{2m}\sum_{i=1}^{m}(\hat{y}^{(i)} - y^{(i)})^2
$$

Where:
\begin{itemize}
  \item $\hat{y}$ is a vector of dimension $m\times 1$, the vector of predicted values
  \item $y$ is a vector of dimension $m\times 1$, the vector of expected values
  \item $\hat{y}^{(i)}$ is the ith component of vector $\hat{y}$
  \item $y^{(i)}$ is the ith component of vector $y$
\end{itemize}

\newpage

% ================================= %
\section*{Instructions}
% --------------------------------- %
The implementation of the loss function has been split in two functions:
\begin{itemize}
  \item \texttt{loss\_elem\_()}, which computes the squared distances for all examples ($\hat{y}^{(i)} - y^{(i)})^2$,
  \item \texttt{loss\_()}, which averages the squared distances of all examples (the $J_(\theta)$ above).
\end{itemize}

In the loss.py file create the following functions as per the instructions given below:
\par
\begin{minted}[bgcolor=darcula-back,formatcom=\color{lightgrey},fontsize=\scriptsize]{python}
def loss_elem_(y, y_hat):
	"""
	Description:
		Calculates all the elements (y_pred - y)^2 of the loss function.
	Args:
      y: has to be an numpy.array, a two-dimensional array of shape m * 1.
      y_hat: has to be an numpy.array, a two-dimensional array of shape m * 1.
	Returns:
		J_elem: numpy.array, a array of dimension (number of the training examples, 1).
		None if there is a dimension matching problem.
		None if any argument is not of the expected type.
	Raises:
		This function should not raise any Exception.
	"""
	... your code here ...

def loss_(y, y_hat):
	"""
	Description:
		Calculates the value of loss function.
	Args:
      y: has to be an numpy.array, a two-dimensional array of shape m * 1.
      y_hat: has to be an numpy.array, a two-dimensional array of shape m * 1.
	Returns:
		J_value : has to be a float.
		None if there is a dimension matching problem.
		None if any argument is not of the expected type.
	Raises:
		This function should not raise any Exception.
	"""
	... your code here ...
\end{minted}

% ================================= %
\section*{Examples}
% --------------------------------- %
\begin{minted}[bgcolor=darcula-back,formatcom=\color{lightgrey},fontsize=\scriptsize]{python}
import numpy as np

x1 = np.array([[0.], [1.], [2.], [3.], [4.]])
theta1 = np.array([[2.], [4.]])
y_hat1 = predict_(x1, theta1)
y1 = np.array([[2.], [7.], [12.], [17.], [22.]])

# Example 1:
loss_elem_(y1, y_hat1)

# Output:
array([[0.], [1], [4], [9], [16]])

# Example 2:
loss_(y1, y_hat1)

# Output:
3.0

x2 = np.array([0, 15, -9, 7, 12, 3, -21]).reshape(-1, 1)
theta2 = np.array(np.array([[0.], [1.]]))
y_hat2 = predict_(x2, theta2)
y2 = np.array([2, 14, -13, 5, 12, 4, -19]).reshape(-1, 1)

# Example 3:
loss_(y2, y_hat2)

# Output:
2.142857142857143

# Example 4:
loss_(y2, y2)

# Output:
0.0
\end{minted}

\info{
This loss function is very close to the one called \textbf{"Mean Squared Error"}, which is frequently mentioned in Machine Learning resources.
The difference is in the denominator as you can see in the formula of the $MSE = \frac{1}{m}\sum_{i=1}^{m}(\hat{y}^{(i)} - y^{(i)})^2$.\
\newline
Except for the division by $2m$ instead of $m$, these functions are rigourously identical: $J(\theta) = \frac{MSE}{2}$.\
\newline
MSE is called like that because it represents the mean of the errors (i.e.: the differences between the predicted values and the true values), squared.\
\newline
You might wonder why we choose to divide by two instead of simply using the MSE?  
  \textbf{(It's a good question, by the way.)}
  \begin{itemize}
    \item First, it does not change the overall model evaluation: if all performance measures are divided by two, we can still compare different models and their performance ranking will remain the same.
    \item Second, it will be very convenient when we calculate the gradient tommorow. Be patient, and trust us ;)
  \end{itemize}
}
  
% ===========================(fin ex 06)         %
% ============================================== %
\newpage
% ============================================== %
% ===========================(start ex 07)       %
\chapter{Exercise 07}
\extitle{Vectorized loss function}
\turnindir{ex07}
\exnumber{07}
\exfiles{vec\_loss.py}
\exforbidden{None}
\makeheaderfilesforbidden
  
% ================================= %
\section*{Objective}
% --------------------------------- %
Understand and experiment with the \textbf{loss function} in machine learning.
  
You must implement the following formula as a function:  
$$
\begin{matrix}
  J(\theta) &  = & \frac{1}{2m}(\hat{y} - y) \cdot(\hat{y}- y)
\end{matrix}
$$

Where:
\begin{itemize}
  \item $\hat{y}$ is a vector of dimension $m$, the vector of predicted values
  \item $y$ is a vector of dimension $m$, the vector of expected values
\end{itemize}

\newpage

% ================================= %
\section*{Instructions}
% --------------------------------- %
In the \texttt{vec\_loss.py} file, create the following function as per the instructions given below:

\begin{minted}[bgcolor=darcula-back,formatcom=\color{lightgrey},fontsize=\scriptsize]{python}
def loss_(y, y_hat):
    """Computes the half mean-squared-error of two non-empty numpy.arrays, without any for loop.
    The two arrays must have the same dimensions.
    Args:
      y: has to be an numpy.array, a one-dimensional array of size m.
      y_hat: has to be an numpy.array, a one-dimensional array of size m.
    Returns:
      The half mean-squared-error of the two vectors as a float.
      None if y or y_hat are empty numpy.array.
      None if y and y_hat does not share the same dimensions.
    Raises:
      This function should not raise any Exceptions.
    """
    ... Your code ...
\end{minted}


% ================================= %
\section*{Examples}
% --------------------------------- %
\begin{minted}[bgcolor=darcula-back,formatcom=\color{lightgrey},fontsize=\scriptsize]{python}
import numpy as np
X = np.array([0, 15, -9, 7, 12, 3, -21])
Y = np.array([2, 14, -13, 5, 12, 4, -19])

# Example 1:
loss_(X, Y)
# Output:
2.142857142857143

# Example 2:
loss_(X, X)
# Output:
0.0
\end{minted}
% ===========================(fin ex 07)         %
% ============================================== %
\newpage
% ============================================== %
% ===========================(start ex 08)       %
\chapter{Exercise 08}
%******************************************************************************%
%                                                                              %
%                                 Interlude                                    %
%                         for Machine Learning module                          %
%                                                                              %
%******************************************************************************%

% ============================================== %
\section*{Interlude}
% ============================================== %
\subsection*{More Evaluation Metrics!}
% ---------------------------------------------- %
Once your classifier is trained, evaluating its performances is key.\\
\\
You already know about \textit{cross-entropy}, as you have implemented it as your \textit{loss function}.
But when it comes to classification, there are more informative metrics we can use besides the loss function.
Each metric focuses on different error types.  
But what is an error type?\\
\\
A single classification prediction is either right or wrong, nothing in between.
Either an object is assigned to the right class, or to the wrong class.\\
When calculating performance scores for a multiclass classifier, we like to compute a 
separate score for each class that your classifier learned to discriminate (in a one-vs-all manner).\\
\\
In other words, for a given \textit{Class A}, we want a score that can answer the question: "how good is 
the model at assigning \textit{A} objects to \textit{Class A}, and at NOT assigning 
\textit{non-A} objects to \textit{Class A}?" \\
\\
You may not realize it yet, but this question involves measuring two 
very different error types, and this distinction is crucial.\\
\newpage
% ============================================== %
\subsection*{Error Types}
% ---------------------------------------------- %
With respect to a given \textit{Class A}, classification errors fall in two categories:
\begin{itemize}
    \item \textbf{False positive:} when a \textit{non-A} object is assigned to \textit{Class A}.\\
      For example:
      \begin{itemize}
          \item Pulling the fire alarm when there is no fire.
          \item Considering that someone is sick when she isn't.
          \item Identifying a face in an image when in fact it was a Teddy Bear.
      \end{itemize}
    
    \item \textbf{False negative:} when an \textit{A} object is assigned to another class than \textit{Class A}.\\
      For example:
      \begin{itemize}
          \item Not pulling the fire alarm when there is a fire.
          \item Considering that someone is not sick when she is.
          \item Failing to recognize a face in an image that does contain one.
      \end{itemize}
\end{itemize}
\bigskip
\noindent{It turns out that it's really hard to minimize both error types at the same time.}\\
\\
At some point you'll need to decide which one is the most critical, depending on your use case.\\
\\
For example, if you want to detect cancer, of course it's not good if your model 
erroneously diagnoses cancer on a few healthy patients (\textbf{false positives}), 
but you absolutely want to avoid failing at diagnosing cancer on affected patients (\textbf{false negatives}) 
and let them go on with their lives while developing a potentially dangerous cancer.\\

% ============================================== %
\subsection*{Metrics}
% ---------------------------------------------- %
A metric is computed on a set of predictions along with the corresponding set of actual categories.
The metric you choose will focus more or less on those two error types.
If we come back to the \textbf{Class A} classifier:
\begin{itemize}
    \item \textbf{Accuracy}: tells you the percentage of predictions that are accurate (i.e. the correct class was predicted).
          Accuracy doesn't give information about either error type.
    \item \textbf{Precision}: tells you how much you can trust your model when it says that an object belongs to \textit{Class A}.
          More precisely, it is the percentage of the objects assigned to \textit{Class A} that really were \textit{A} objects.
          You use precision when you want to control for \textbf{False positives}.
    \item \textbf{Recall}: tells you how much you can trust that your model is able to recognize ALL \textit{Class A} objects.
          It is the percentage of all \textbf{A} objects that were properly classified by the model as \textit{Class A}.
          You use recall when you want to control for \textbf{False negatives}.
    \item \textbf{F1 score}: combines precision and recall in one single measure.
          You use the F1 score when you want to control both \textbf{False positives} and \textbf{False negatives}.
\end{itemize}

\newpage
\extitle{Lets Make Nice Plots Again}
\turnindir{ex08}
\exnumber{08}
\exfiles{plot.py}
\exforbidden{None}
\makeheaderfilesforbidden


% ================================= %
\section*{Objective}
% --------------------------------- %
You must implement a function which plots the data, the prediction line, and the loss.\\
\newline
You will plot the $x$ and $y$ coordinates of all data points as well as the prediction line generated by your theta parameters.\\
\newline
Your function must also display the overall loss ($J$) in the title, and draw small lines marking the distance between each data point and its predicted value.

% ================================= %
\section*{Instructions}
% --------------------------------- %
In the plot.py file create the following function as per the instructions given below:\\
\newline
\begin{minted}[bgcolor=darcula-back,formatcom=\color{lightgrey},fontsize=\scriptsize]{python}
def plot_with_loss(x, y, theta):
"""Plot the data and prediction line from three non-empty numpy.ndarray.
    Args:
      x: has to be an numpy.ndarray, one-dimensional array of size m.
      y: has to be an numpy.ndarray, one-dimensional array of size m.
      theta: has to be an numpy.ndarray, one-dimensional array of size 2.
    Returns:
        Nothing.
    Raises:
      This function should not raise any Exception.
    """
    ... Your code ...
\end{minted}

\newpage

% ================================= %
\section*{Examples}
% --------------------------------- %
\begin{minted}[bgcolor=darcula-back,formatcom=\color{lightgrey},fontsize=\scriptsize]{python}
import numpy as np
x = np.arange(1,6)
y = np.array([11.52434424, 10.62589482, 13.14755699, 18.60682298, 14.14329568])

# Example 1:
theta1= np.array([18,-1])
plot_with_loss(x, y, theta1)
# Output:
\end{minted}

\begin{figure}[H]
  \centering
  \includegraphics[scale=0.65]{assets/plotcost1.png}
  \caption{Example 1}
\end{figure}

\begin{minted}[bgcolor=darcula-back,formatcom=\color{lightgrey},fontsize=\scriptsize]{python}
# Example 2:
theta2 = np.array([14, 0])
plot_with_loss(x, y, theta2)
# Output:
\end{minted}

\begin{figure}[H]
  \centering
  \includegraphics[scale=0.65]{assets/plotcost2.png}
  \caption{Example 2}
\end{figure}

\newpage

\begin{minted}[bgcolor=darcula-back,formatcom=\color{lightgrey},fontsize=\scriptsize]{python}
# Example 3:
theta3 = np.array([12, 0.8])
plot_with_loss(x, y, theta3)
# Output:
\end{minted}

\begin{figure}[H]
  \centering
  \includegraphics[scale=0.65]{assets/plotcost3.png}
  \caption{Example 3}
\end{figure}
% ===========================(fin ex 08)         %
% ============================================== %
\newpage
% ============================================== %
% ===========================(start ex 09)       %
\chapter{Exercise 09}
\extitle{Other loss functions}
\turnindir{ex09}
\exnumber{09}
\exfiles{other\_losses.py}
\exforbidden{None}
\makeheaderfilesforbidden

Deepen the notion of loss function in machine learning.

You certainly had a lot of fun implementing your loss function!
Remember we told you it was \textbf{one among many possible ways of measuring the loss}.\\
\newline
Now, you will get to implement other metrics.  You already know about one of them: \textbf{MSE}.  
There are several more which are quite common: \textbf{RMSE}, \textbf{MAE} and \textbf{R2score}.

\newpage

% ================================= %
\section*{Objective}
% --------------------------------- %
You must implement the following formulas as functions:

$$
MSE(y, \hat{y}) = \frac{1}{m}\sum_{i=1}^{m}(\hat{y}^{(i)} - y^{(i)})^2
$$

$$
RMSE(y, \hat{y}) = \sqrt{\frac{1}{m}\sum_{i=1}^{m}(\hat{y}^{(i)} - y^{(i)})^2}
$$

$$
MAE(y, \hat{y}) = \frac{1}{m}\sum_{i=1}^{m}|{\hat{y}^{(i)} - y^{(i)}}|
$$

$$
R^2(y, \hat{y}) = 1 - \frac{\sum_{i=1}^{m}(\hat{y}^{(i)} - y^{(i)})^2}{\sum_{i=1}^{m}({y}^{(i)} - \bar{y})^2}
$$

Where:
\begin{itemize}
  \item $y$ is a vector of dimension $m$
  \item $\hat{y}$ is a vector of dimension $m$
  \item $y^{(i)}$ is the $i^{th}$ component of vector $y$
  \item $\hat{y}^{(i)}$ is the $i^{th}$ component of $\hat{y}$
  \item $\bar{y}$ is the mean of the $y$ vector
\end{itemize}

\newpage

% ================================= %
\section*{Instructions}
% --------------------------------- %
In the \texttt{other\_losses.py} file, create the following functions as per the instructions given below:

\begin{minted}[bgcolor=darcula-back,formatcom=\color{lightgrey},fontsize=\scriptsize]{python}
def mse_(y, y_hat):
	"""
	Description:
		Calculate the MSE between the predicted output and the real output.
	Args:
        y: has to be a numpy.array, a two-dimensional array of shape m * 1.
        y_hat: has to be a numpy.array, a two-dimensional vector of shape m * 1.		
	Returns:
		mse: has to be a float.
		None if there is a matching dimension problem.
	Raises:
		This function should not raise any Exceptions.
	"""
		... your code here ...
\end{minted}
\begin{minted}[bgcolor=darcula-back,formatcom=\color{lightgrey},fontsize=\scriptsize]{python}
def rmse_(y, y_hat):
	"""
	Description:
		Calculate the RMSE between the predicted output and the real output.
	Args:
	      y: has to be a numpy.array, a two-dimensional array of shape m * 1.
        y_hat: has to be a numpy.array, a two-dimensional array of shape m * 1.		
	Returns:
		rmse: has to be a float.
		None if there is a matching dimension problem.
	Raises:
		This function should not raise any Exceptions.
	"""
		... your code here ...
\end{minted}
\begin{minted}[bgcolor=darcula-back,formatcom=\color{lightgrey},fontsize=\scriptsize]{python}
def mae_(y, y_hat):
	"""
	Description:
		Calculate the MAE between the predicted output and the real output.
	Args:
        y: has to be a numpy.array, a two-dimensional array of shape m * 1.
        y_hat: has to be a numpy.array, a two-dimensional array of shape m * 1.		
	Returns:
		mae: has to be a float.
		None if there is a matching dimension problem.
	Raises:
		This function should not raise any Exceptions.
	"""
		... your code here ...
\end{minted}
\begin{minted}[bgcolor=darcula-back,formatcom=\color{lightgrey},fontsize=\scriptsize]{python}
def r2score_(y, y_hat):
	"""
	Description:
		Calculate the R2score between the predicted output and the output.
	Args:
        y: has to be a numpy.array, a two-dimensional array of shape m * 1.
        y_hat: has to be a numpy.array, a two-dimensional array of shape m * 1.		
	Returns:
		r2score: has to be a float.
		None if there is a matching dimension problem.
	Raises:
		This function should not raise any Exceptions.
	"""
		... your code here ...
\end{minted}

\hint{
  You might consider implementing four more methods, similar to what you did for the loss function in exercise 07:
\begin{itemize}
  \item \texttt{mse\_elem()},
  \item \texttt{rmse\_elem()},
  \item \texttt{mae\_elem()},
  \item \texttt{r2score\_elem()}.
\end{itemize}
}

% ================================= %
\section*{Examples}
% --------------------------------- %
\begin{minted}[bgcolor=darcula-back,formatcom=\color{lightgrey},fontsize=\scriptsize]{python}
import numpy as np
from sklearn.metrics import mean_squared_error, mean_absolute_error, r2_score
from math import sqrt

# Example 1:
x = np.array([[0], [15], [-9], [7], [12], [3], [-21]])
y = np.array([[2], [14], [-13], [5], [12], [4], [-19]])

# Mean-squared-error
## your implementation
mse_(x,y)
## Output:
4.285714285714286
## sklearn implementation
mean_squared_error(x,y)
## Output:
4.285714285714286

# Root mean-squared-error
## your implementation
rmse_(x,y)
## Output:
2.0701966780270626
## sklearn implementation not available: take the square root of MSE
sqrt(mean_squared_error(x,y))
## Output:
2.0701966780270626

# Mean absolute error
## your implementation
mae_(x,y)
# Output:
1.7142857142857142
## sklearn implementation
mean_absolute_error(x,y)
# Output:
1.7142857142857142

# R2-score
## your implementation
r2score_(x,y)
## Output:
0.9681721733858745
## sklearn implementation
r2_score(x,y)
## Output:
0.9681721733858745
\end{minted}
% ===========================(fin ex 09)         %
% ============================================== %
\newpage
% ============================================== %
% Closing Chapter %
\chapter{Conclusion - What you have learnt}

This first series of exercises is finished, well done!

Based on all the notions and problems tackled today, you should be able to discuss and answer the following questions:
\begin{enumerate}
  \item Why do we concatenate a column of ones to the left of the $x$ vector when we use the linear algebra trick?   
  
  \item Why does the loss function square the distances between the data points and their predicted values?
  
  \item What does the loss function's output represent?
  
  \item Towards which value do we want the loss function to tend? What would that mean? 
  
  \item Do you understand why matrix multiplications are not commutative?
\end{enumerate}

These questions are an opportunity for discussion among your peers, or to simply reflect on your own acquired understanding during this day ! 

% ============================================== %
\newpage
% ================================= %

\section*{Contact}
% --------------------------------- %
You can contact 42AI by email: \href{mailto:contact@42ai.fr}{contact@42ai.fr}\\
\newline
Thank you for attending 42AI's Machine Learning Bootcamp !

% ================================= %
\section*{Acknowledgements}
% --------------------------------- %
The Python \& ML bootcamps are the result of a collective effort. We would like to thank:\\
\begin{itemize}
  \item Maxime Choulika (cmaxime),
  \item Pierre Peigné (ppeigne),
  \item Matthieu David (mdavid),
  \item Quentin Feuillade--Montixi (qfeuilla, quentin@42ai.fr)
  \item Mathieu Perez (maperez, mathieu.perez@42ai.fr)
\end{itemize}
who supervised the creation and enhancements of the present transcription.\\
\begin{itemize}
  \item Louis Develle (ldevelle, louis@42ai.fr)
  \item Owen Roberts (oroberts)
  \item Augustin Lopez (aulopez)
  \item Luc Lenotre (llenotre)
  \item Amric Trudel (amric@42ai.fr)
  \item Benjamin Carlier (bcarlier@student.42.fr)
  \item Pablo Clement (pclement@student.42.fr)
  \item Amir Mahla (amahla, amahla@42ai.fr)
\end{itemize}
for your investment in the creation and development of these modules.\\
\begin{itemize}
    \item All prior participants who took a moment to provide their feedbacks, and help us improve these bootcamps !
  \end{itemize}

\vfill
\doclicenseThis


\end{document}
