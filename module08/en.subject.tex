% vim: set ts=4 sw=4 tw=80 noexpandtab:

\documentclass{42-en}

%******************************************************************************%
%                                                                              %
%                                   Prologue                                   %
%                                                                              %
%******************************************************************************%
\usepackage[
    type={CC},
    modifier={by-nc-sa},
    version={4.0},
]{doclicense}
\usepackage{amsmath} % The amsmath package provides commands to typeset matrices with different delimiters. 
\usepackage{epigraph}
\setlength\epigraphwidth{.95\textwidth}
\usepackage{multirow}
\usepackage{cancel}
%****************************************************************%
%                  Re/definition of commands                     %
%****************************************************************%

\newcommand{\ailogo}[1]{\def \@ailogo {#1}}\ailogo{assets/42ai_logo.pdf}

%%  Redefine \maketitle
\makeatletter
\def \maketitle {
  \begin{titlepage}
    \begin{center}
	%\begin{figure}[t]
	  %\includegraphics[height=8cm]{\@ailogo}
	  \includegraphics[height=8cm]{assets/42ai_logo.pdf}
	%\end{figure}
      \vskip 5em
      {\huge \@title}
      \vskip 2em
      {\LARGE \@subtitle}
      \vskip 4em
    \end{center}
    %\begin{center}
	  %\@author
    %\end{center}
	%\vskip 5em
  \vfill
  \begin{center}
    \emph{\summarytitle : \@summary}
  \end{center}
  \vspace{2cm}
  %\vskip 5em
  %\doclicenseThis
  \end{titlepage}
}
\makeatother

\makeatletter
\def \makeheaderfilesforbidden
{
  \noindent
  \begin{tabularx}{\textwidth}{|X X  X X|}
    \hline
  \multicolumn{1}{|>{\raggedright}m{1cm}|}
  {\vskip 2mm \includegraphics[height=1cm]{assets/42ai_logo.pdf}} &
  \multicolumn{2}{>{\centering}m{12cm}}{\small Exercise : \@exnumber } &
  \multicolumn{1}{ >{\raggedleft}p{1.5cm}|}
%%              {\scriptsize points : \@exscore} \\ \hline
              {} \\ \hline

  \multicolumn{4}{|>{\centering}m{15cm}|}
              {\small \@extitle} \\ \hline

  \multicolumn{4}{|>{\raggedright}m{15cm}|}
              {\small Turn-in directory : \ttfamily
                $ex\@exnumber/$ }
              \\ \hline
  \multicolumn{4}{|>{\raggedright}m{15cm}|}
              {\small Files to turn in : \ttfamily \@exfiles }
              \\ \hline

  \multicolumn{4}{|>{\raggedright}m{15cm}|}
              {\small Forbidden functions : \ttfamily \@exforbidden }
              \\ \hline

%%  \multicolumn{4}{|>{\raggedright}m{15cm}|}
%%              {\small Remarks : \ttfamily \@exnotes }
%%              \\ \hline
\end{tabularx}
%% \exnotes
\exrules
\exmake
\exauthorize{None}
\exforbidden{None}
\extitle{}
\exnumber{}
}
\makeatother

%%  Syntactic highlights
\makeatletter
\newenvironment{pythoncode}{%
  \VerbatimEnvironment
  \usemintedstyle{emacs}
  \minted@resetoptions
  \setkeys{minted@opt}{bgcolor=black,formatcom=\color{lightgrey},fontsize=\scriptsize}
  \begin{figure}[ht!]
    \centering
    \begin{minipage}{16cm}
      \begin{VerbatimOut}{\jobname.pyg}}
{%[
      \end{VerbatimOut}
      \minted@pygmentize{c}
      \DeleteFile{\jobname.pyg}
    \end{minipage}
\end{figure}}
\makeatother
%****************************************************************%
%                 END Re/definition of commands                  %
%****************************************************************%
\usemintedstyle{native}
\begin{document}

% =============================================================================%
%                     =====================================   

\title{Machine Learning Bootcamp - Module 03}
\subtitle{Logistic Regression}
\author{
  Maxime Choulika (cmaxime), Pierre Peigné (ppeigne), Matthieu David (mdavid), Amir Mahla (amahla), Mathieu Perez (maperez)
}

\summary
{
  Discover your first classification algorithm: logistic regression.
  You will learn about its loss function, gradient descent and some metrics to evaluate its performance.
}

\maketitle
%******************************************************************************%
%                                                                              %
%                        Section usefull ressources                            %
%                          for ML Modules                                      %
%                                                                              %
%******************************************************************************%


\chapter*{Notions and ressources}

\section*{Notions of the module}
\begin{itemize}
  \item Regularization
  \item Overfitting
  \item Regularized loss function
  \item Regularized gradient descent
  \item Regularized linear regression
  \item Regularized logistic regression
\end{itemize}

\section*{Useful Ressources}

You are recommended to use the following material: \href{https://www.coursera.org/learn/machine-learning}{Machine Learning MOOC - Stanford}\\
\newline
This series of videos is available at no cost: simply log in, select "Enroll for Free", and click "Audit" at the bottom of the pop-up window.\\
\newline
The following sections of the course are particularly relevant to today's exercises: 

\subsection*{Week 3: Classification}

\subsubsection*{Classification with logistic regression (already seen in module 03)}
\begin{itemize}
  \item Motivations
  \item Logistic regression
  \item Decision boundary
\end{itemize}

\subsubsection*{Cost function for logistic regression (already seen in module 03)}
\begin{itemize}
  \item Cost function for logistic regression
  \item Simplified Cost Function for Logistic Regression
\end{itemize}

\subsubsection*{Gradient descent for logistic regression (already seen in module 03)}
\begin{itemize}
  \item Gradient Descent Implementation
\end{itemize}

\subsubsection*{The problem of overfitting (New !!!)}
\begin{itemize}
  \item The problem of overfitting
  \item Addressing overfitting
  \item Cost function with regularization
  \item Regularized linear regression
  \item Regularized logistic regression  
\end{itemize}

\noindent{\emph{All videos above are available also on this
 \href{https://youtube.com/playlist?list=PLkDaE6sCZn6FNC6YRfRQc_FbeQrF8BwGI&feature=shared}
 {Andrew Ng's YouTube playlist}, videos 31 to 36 (already seen in module 03) and 37 to 41 (new !!!).}}
%******************************************************************************%
%                                                                              %
%                        Common Instructions                                   %
%                          for Python Projects                                 %
%                                                                              %
%******************************************************************************%

\chapter{Common Instructions}
\begin{itemize}
  \item The version of Python recommended to use is 3.7. You can
  check your Python's version with the following command: \texttt{python -V}
  
  \item The norm: during this bootcamp, it is recommended to follow the
  \href{https://www.python.org/dev/peps/pep-0008/}{PEP 8 standards}, though it is not mandatory.
  You can install \href{https://pypi.org/project/pycodestyle}{pycodestyle} or 
  \href{https://black.readthedocs.io/en/stable/}{Black}, which are convenient 
  packages to check your code.
  
  \item The function \texttt{eval} is never allowed.
  
  \item The exercises are ordered from the easiest to the hardest.
  
  \item Your exercises are going to be evaluated by someone else,
  so make sure that your variable names and function names are appropriate and civil.

  \item Your manual is the internet.

  \item If you're planning on using an AI assistant such as a LLM, make sure it is helpful 
  for you to \textbf{learn and practice}, not to provide you with hands-on solution ! Own your tool, don't let it own you.
  
  \item If you are a student from 42, you can access our Discord server 
  on \href{https://discord.com/channels/887850395697807362/887850396314398720}{42 student's associations portal} and ask your
  questions to your peers in the dedicated Bootcamp channel. 

  \item You can learn more about 42 Artificial Intelligence by visiting \href{https://42-ai.github.io}{our website}.

  \item If you find any issue or mistake in the subject please create an issue on 
  \href{https://github.com/42-AI/bootcamp_machine-learning/issues}{42AI repository on Github}.
  
  \item We encourage you to create test programs for your
  project even though this work \textbf{won't have to be
  submitted and won't be graded}. It will give you a chance
  to easily test your work and your peers’ work. You will find
  those tests especially useful during your defence. Indeed,
  during defence, you are free to use your tests and/or the
  tests of the peer you are evaluating.

\end{itemize}
\newpage
\tableofcontents
\startexercices

%                     =====================================                    %
% =============================================================================%


%******************************************************************************%
%                                                                              %
%                                   Exercises                                  %
%                                                                              %
%******************************************************************************%

% ============================================== %
% ===========================(start ex 00)       %
\chapter{Exercise 00}
%******************************************************************************%
%                                                                              %
%                                 Interlude                                    %
%                         for Machine Learning module                          %
%                                                                              %
%******************************************************************************%

% =============================== %
\section*{Interlude}
% =============================== %
\subsection*{Classification: The Art of Labelling Things}
% ******************************* %
Over the last three modules you have implemented your first machine learning algorithm.\\
\\
You are now familiar the three-steps cycle we follow when we build \textbf{learning algorithms}:
\\
\begin{figure}[!h]
    \centering
    \includegraphics[scale=0.25]{assets/Default.png}
    %\caption{The Learning Cycle}
\end{figure}
\\
The first algorithm you discovered, \textbf{Multivariate Linear Regression}, can now be used to predict a numerical value, based on several features.
This algorithm uses gradient descent to optimize its loss function.\\
\\
Now let's introduce your first \textbf{classification algorithm}: the notorious \textbf{Logistic Regression.}
\hint{regression vs classification; discrete vs continuous values}
\newpage
\noindent{\textbf{Logistic regression} performs a \textit{classification task}, which means that you are not predicting a numerical value (like price, age, grades...) 
but \textbf{categories}, or \textbf{labels} (like dog, cat, sick/healty...)}.
\\
\warn{
    Don't be confused by the word \textit{'regression'} in \textbf{Logistic Regression}.
    It really is a \textit{classification task}! The name is a bit tricky but you will quickly get used to it.
    Once again: \textbf{Logistic Regression is a classification algorithm} which assigns a label/category/class to a given example.
}
\info{
    In this module we will use the following terms interchangeably: \textbf{class}, \textbf{category}, and \textbf{label}.
    They all refer to the \textit{groups} to which each training example can be assigned to, in a classification task.
}

% =============================== %
\subsection*{Predict I: Introducing the Sigmoid Function}
% ******************************* %

\begin{figure}[!h]
    \centering
    \includegraphics[scale=0.25]{assets/Predict.png}
    %\caption{The Learning Cycle - Predict}
\end{figure}

% =============================== %
\subsubsection*{Formulating a Hypothesis}
% ******************************* %
Remember that a hypothesis, denoted $h(\theta)$, is an equation that combines a set of \textbf{features} (that characterizes an example) with \textbf{parameters} in order to output a \textbf{prediction}.\\
\\
Remember the hypothesis we used in linear regression?\\
$$
h(\theta) = \theta_0 + \theta_{1} x_{1}^{(i)} + \dots + \theta_{n} x_{n}^{(i)} = \theta \cdot x'^{(i)}
$$
\newline
It worked fine to predict continuous values, but could we also use it to tell, for example, 
if a patient is sick or not?
That's a yes-or-no question, so the output from the hypothesis function should reflect that.\\
\\
To get started, we will assign each class a numerical value: sick patients will be 
assigned a value of 1, and healthy patients will be assigned a value of 0.\\
The goal will be to build a hypothesis that outputs a probability that a patient is sick as a float number in the range of [0, 1].\\
\\
The good news is that we can keep the linear equation we already worked with!\\
\\
All we need to do is squash its output through another function that is bounded between 0 and 1.\\
\\
That's the purpose of the \textbf{Sigmoid function} and your next assignment is to implement it!

\newpage
\extitle{Sigmoid}
\turnindir{ex00}
\exnumber{00}
\exfiles{sigmoid.py}
\exforbidden{None}
\makeheaderfilesforbidden

% ================================== %
\section*{Objective}
% ---------------------------------- %
Introduction to the hypothesis in the context of logistic regression.\\
\\
You must implement the sigmoid function, given by the following formula:

$$
\text{sigmoid}(x) = \cfrac{1} {1 + e^{-x}}
$$
Where:
\begin{itemize}
  \item $x$ is a scalar or a vector,
  \item $e$ is the contracted form for the exponential function. It is also a mathematical constant, named Euler's number.
\end{itemize}
This function is also known as \textbf{Standard logistic sigmoid function}.
This explains the name \textit{logistic regression}.\\
\\
The sigmoid function transforms an input into a probability value, i.e. a value between 0 and 1.  
This probability value will then be used to classify the inputs.
\\
% ================================== %
\section*{Instructions}
% ---------------------------------- %
In the \texttt{sigmoid.py} file, write the following function as per the instructions below:\\
\\
\par
\begin{minted}[bgcolor=darcula-back,formatcom=\color{lightgrey},fontsize=\scriptsize]{python}
def sigmoid_(x):
    """
    Compute the sigmoid of a vector.
    Args:
        x: has to be a numpy.ndarray of shape (m, 1).
    Returns: 
        The sigmoid value as a numpy.ndarray of shape (m, 1).
        None if x is an empty numpy.ndarray.
    Raises:
        This function should not raise any Exception.
    """
    ... Your code ...
\end{minted}

% ================================== %
\section*{Examples}
% ---------------------------------- %

\begin{minted}[bgcolor=darcula-back,formatcom=\color{lightgrey},fontsize=\scriptsize]{python}
# Example 1:
x = np.array([[-4]])
sigmoid_(x)
# Output:
array([[0.01798620996209156]])

# Example 2:
x = np.array([[2]])
sigmoid_(x)
# Output:
array([[0.8807970779778823]])

# Example 3:
x = np.array([[-4], [2], [0]])
sigmoid_(x)
# Output:
array([[0.01798620996209156], [0.8807970779778823], [0.5]])
\end{minted}


\info{
  Our sigmoid formula is a special case of the logistic function below, with $L = 1$, $k = 1$ and $x_0 = 0$:
  $$
  f(x) = \cfrac{L}{1 + e^{-k(x-x_0)}}
  $$
}
% ===========================(fin ex 00)         %
% ============================================== %
\newpage
% ===========================(start ex 01)       %
\chapter{Exercise 01}
%******************************************************************************%
%                                                                              %
%                                 Interlude                                    %
%                         for Machine Learning module                          %
%                                                                              %
%******************************************************************************%

% =============================== %
\section*{Linear Algebra Tricks part II}
% ******************************* %

If you tried to run your code on a very large dataset, you would find that it sometimes takes a (very) long time to execute!
That's because it doesn't use the power of Python libraries that are optimized for matrix operations.\\
\newline
Remember the linear algebra trick from the previous module? Let's use it again!  
If you concatenate a column of $1$'s to the left of the $x$ vector, you get what we called matrix $X'$.   
$$
X' = \begin{bmatrix} 1 & x^{(1)} \\ \vdots & \vdots \\ 1 & x^{(m)}\end{bmatrix}
$$
This transformation is very convenient because we can rewrite each $1$ as $x_0^{(i)}$, and each $x^{(i)}$ as $x_1^{(i)}$.
So now the $X'$ matrix looks like this:
$$
X' = \begin{bmatrix} x_0^{(1)} & x_1^{(1)} \\ \vdots & \vdots \\ x_0^{(m)} & x_1^{(m)}\end{bmatrix}
$$
Notice that each $x^{(i)}$ example becomes a vector made of $(x^{(i)}_0, x^{(i)}_1)$.  
The $0$ and $1$ indices on the $x$ features correspond to the indices of the $\theta$ parameters with which they will be multiplied.\\
\newline
Why does this matter?
Well, if we take the equation from the previous exercise:

$$
\nabla(J)_0 = \frac{1}{m}\sum_{i=1}^{m}(h_{\theta}(x^{(i)}) - y^{(i)})
$$
We can multiply it by $1$ without changing its value:
$$
\nabla(J)_0 = \frac{1}{m}\sum_{i=1}^{m}(h_{\theta}(x^{(i)}) - y^{(i)}) \cdot 1
$$
And rewrite $1$ as $x_0^{(i)}$:
$$
\nabla(J)_0 = \frac{1}{m}\sum_{i=1}^{m}(h_{\theta}(x^{(i)}) - y^{(i)})x_{0}^{(i)}
$$
This means that the equation for $\nabla(J)_0$ is now similar to the equation we had for $\nabla(J)_1$, so they can both be captured by ONE \textbf{generic equation}:
$$
\begin{matrix}
\nabla(J)_j = \frac{1}{m}\sum_{i=1}^{m}(h_{\theta}(x^{(i)}) - y^{(i)})x_{j}^{(i)} & & \text{ for j = 0, 1}    
\end{matrix}
$$
And as you probably suspected, a generic equation opens the door to vectorization...

% =============================== %
\subsection*{Vectorizing the Gradient Calculation}
% ******************************* %
Now it's time to learn how to calculate the entire gradient in one short, pretty, linear algebra equation!  
\begin{itemize}
    \item First, we'll use the $X'$ matrix and our vectorized hypothesis equation $h_{\theta}(x)=X'\theta$
    $$
    \begin{matrix}
    \nabla(J)_j = \frac{1}{m} (X'\theta - y)X'_{j} & & \text{ for j = 0, 1}
    \end{matrix}
    $$
    
    \item Second, we need to tweak the equation a bit so that it directly returns a $\nabla(J)$ vector containing both $\nabla(J)_0$ and $\nabla(J)_1$.
    
    $$
    \nabla(J) = \frac{1}{m} {X'}^T(X'\theta - y)    
    $$
\end{itemize}
If the equation does not seems obvious, play a bit with your vectors, on paper and in your code, until you get it.\\

% =============================== %
\subsubsection*{Notation Remark}
% ******************************* %
${X'}^T$: You might wonder what the $^T$ is for.
It means the $X'$ matrix must be \textbf{transposed}.\\
\newline
Transposing a matrix flips it on its diagonal so that its rows become its columns and \textit{vice-versa}.
Here we need to make sure that matrix dimensions are appropriate and allow for multiplication, and to multiply the right items together.
\newpage
\extitle{Logistic Hypothesis}
\turnindir{ex01}
\exnumber{01}
\exfiles{log\_pred.py}
\exforbidden{None}
\makeheaderfilesforbidden

% ================================= %
\section*{Objective}
% --------------------------------- %
Introduction to the hypothesis notion in the context of logistic regression.\\
\\
You must implement the following formula as a function:\\

$$
\begin{matrix}
\hat{y} & = & \text{sigmoid}(X' \cdot \theta) & = & \cfrac{1} {1 + e^{-X' \cdot \theta}}    
\end{matrix}
$$
Where:
\begin{itemize}
  \item $X$ is a matrix of dimensions $(m \times n)$, the design matrix
  \item $X'$ is a matrix of dimensions $(m \times (n + 1))$, 
  the design matrix onto which a column of $1$'s is added as a first column
  \item $\hat{y}$ is a vector of dimension $m$, the vector of predicted values
  \item $\theta$ is a vector of dimension $(n + 1)$, the vector of parameters
\end{itemize}
Be careful: 
\begin{itemize}
  \item the $x$ your function will get as an input corresponds to $X$, the $(m \times n)$ matrix.
        Not $X'$.
  \item $\theta$ is a vector of dimension $(n + 1)$
\end{itemize}
\newpage
% ================================= %
\section*{Instructions}
% --------------------------------- %
In the \texttt{log\_pred.py} file, write the following function as per the instructions below:\\
\par
\begin{minted}[bgcolor=darcula-back,formatcom=\color{lightgrey},fontsize=\scriptsize]{python}
def logistic_predict_(x, theta):
    """Computes the vector of prediction y_hat from two non-empty numpy.ndarray.
    Args:
      x: has to be an numpy.ndarray, a vector of dimension m * n.
      theta: has to be an numpy.ndarray, a vector of dimension (n + 1) * 1.
    Returns:
      y_hat as a numpy.ndarray, a vector of dimension m * 1.
      None if x or theta are empty numpy.ndarray.
      None if x or theta dimensions are not appropriate.
    Raises:
      This function should not raise any Exception.
    """
    ... Your code ...
\end{minted}

% ================================= %
\section*{Examples}
% --------------------------------- %

\begin{minted}[bgcolor=darcula-back,formatcom=\color{lightgrey},fontsize=\scriptsize]{python}
# Example 1
x = np.array([4]).reshape((-1, 1))
theta = np.array([[2], [0.5]])
logistic_predict_(x, theta)
# Output: 
array([[0.98201379]])

# Example 1
x2 = np.array([[4], [7.16], [3.2], [9.37], [0.56]])
theta2 = np.array([[2], [0.5]]) 
logistic_predict_(x2, theta2)
# Output: 
array([[0.98201379],
       [0.99624161],
       [0.97340301],
       [0.99875204],
       [0.90720705]])

# Example 3
x3 = np.array([[0, 2, 3, 4], [2, 4, 5, 5], [1, 3, 2, 7]])
theta3 = np.array([[-2.4], [-1.5], [0.3], [-1.4], [0.7]])
logistic_predict_(x3, theta3)
# Output: 
array([[0.03916572],
       [0.00045262],
       [0.2890505 ]])
\end{minted}
% ===========================(fin ex 01)         %
% ============================================== %
\newpage
% ============================================== %
% ===========================(start ex 02)       %
\chapter{Exercise 02}
%******************************************************************************%
%                                                                              %
%                                 Interlude                                    %
%                         for Machine Learning module                          %
%                                                                              %
%******************************************************************************%

% =============================== %
\section*{Interlude - Evaluate}
% ------------------------------- %

\begin{figure}[!h]
    \centering
    \includegraphics[scale=0.2]{assets/Evaluate.png}
    %\caption{The Learning Cycle: Evaluate}
\end{figure}

% =============================== %
\section*{Back to the Loss Function}
% ------------------------------- %
How is our model doing?\\
To evaluate our model, remember that we have already used a \textbf{metric} called the \textbf{loss function} (also known as \textbf{cost function}).
The loss function is basically just a measure of how wrong the model is, in all of its predictions.\\
\newline
Two modules ago, we defined the loss function as the average of the squared distances between each prediction and its expected value (distances represented by the dotted lines in the figure below):

\begin{figure}[!h]
    \centering
    \includegraphics[scale=0.5]{assets/bad_pred_with_distance.png}
    \caption{Distances between predicted and expected values}
\end{figure}
\newpage
\noindent{The formula was the following:}

$$
J(\theta) = \frac{1}{2m}\sum_{i=1}^{m}(\hat{y}^{(i)} - y^{(i)})^2
$$
\\
And its vectorized form:

$$
\begin{matrix}
J(\theta) = \frac{1}{2m}(\hat{y} - y)\cdot(\hat{y}- y)
\end{matrix}
$$
\\
\textit{So, now that we moved to multivariate linear regression, what does it change?}\\
\newline
You may have noticed that variables such as $x_j$ and $\theta_j$ are not in the equation.
Indeed, the loss function only uses the predictions ($\hat{y}$) and the expected values ($y$), 
so the inner workings of the model do not have an impact on its evaluation metric.\\
\\
This means we can use the exact same loss function as we did before!

\newpage
\extitle{Logistic Loss Function}
\turnindir{ex02}
\exnumber{02}
\exfiles{log\_loss.py}
\exforbidden{None}
\makeheaderfilesforbidden

% ================================= %
\section*{Objective}
% --------------------------------- %
Understanding and manipulation of the loss function in the context of logistic regression.\\
\\
You must implement the following formula as a function:  

$$
J( \theta) = -\cfrac{1} {m} \lbrack \sum_{i = 1}^{m} y^{(i)}\log(\hat{y}^{(i)})) + (1 - y^{(i)})\log(1 - \hat{y}^{(i)})\rbrack
$$
Where:
\begin{itemize}
  \item $\hat{y}$ is a vector of dimension $m$, the vector of predicted values
  \item $\hat{y}^{(i)}$ is the $i^{th}$ component of the $\hat{y}$ vector
  \item $y$ is a vector of dimension $m$, the vector of expected values
  \item $y^{(i)}$ is the $i^{th}$ component of the $y$ vector
\end{itemize}

% ================================= %
\section*{Instructions}
% --------------------------------- %
In the \texttt{log\_loss.py} file, write the following function as per the instructions below:
\\
\par
\begin{minted}[bgcolor=darcula-back,formatcom=\color{lightgrey},fontsize=\scriptsize]{python}
def log_loss_(y, y_hat, eps=1e-15):
    """
    Computes the logistic loss value.
    Args:
        y: has to be an numpy.ndarray, a vector of shape m * 1.
        y_hat: has to be an numpy.ndarray, a vector of shape m * 1.
        eps: has to be a float, epsilon (default=1e-15)
    Returns:
        The logistic loss value as a float.
        None on any error.
    Raises:
        This function should not raise any Exception.
    """
    ... Your code ...
\end{minted}

\hint{
  The logarithmic function isn't defined in $0$.
  This means that if $y^{(i)} = 0$ you will get an error when you try to compute $log(y^{(i)})$.
  The purpose of the \texttt{eps} argument is to avoid $log(0)$ errors.
  It is a very small residual value we add to \texttt{y}, also referred to as `epsilon`.
}

% ================================= %
\section*{Examples}
% --------------------------------- %
\begin{minted}[bgcolor=darcula-back,formatcom=\color{lightgrey},fontsize=\scriptsize]{python}
# Example 1:
y1 = np.array([1]).reshape((-1, 1))
x1 = np.array([4]).reshape((-1, 1))
theta1 = np.array([[2], [0.5]])
y_hat1 = logistic_predict_(x1, theta1)
log_loss_(y1, y_hat1)
# Output:
0.01814992791780973

# Example 2:
y2 = np.array([[1], [0], [1], [0], [1]])
x2 = np.array([[4], [7.16], [3.2], [9.37], [0.56]])
theta2 = np.array([[2], [0.5]])
y_hat2 = logistic_predict_(x2, theta2)
log_loss_(y2, y_hat2)
# Output:
2.4825011602474483

# Example 3:
y3 = np.array([[0], [1], [1]])
x3 = np.array([[0, 2, 3, 4], [2, 4, 5, 5], [1, 3, 2, 7]])
theta3 = np.array([[-2.4], [-1.5], [0.3], [-1.4], [0.7]])
y_hat3 = logistic_predict_(x3, theta3)
log_loss_(y3, y_hat3)
# Output:
2.9938533108607053
\end{minted}

\info{
  This function is called \textbf{Cross-Entropy loss}, or \textbf{logistic loss}.
  For more information you can look at \href{https://en.wikipedia.org/wiki/Cross_entropy\#Cross-entropy\_error\_function\_and\_logistic\_regression}{this section}
  of the Cross entropy Wikipedia article.
}
% ===========================(fin ex 02)         %
% ============================================== %
\newpage
% ============================================== %
% ===========================(start ex 03)       %
\chapter{Exercise 03}
\extitle{Vectorized Logistic Loss Function}
%******************************************************************************%
%                                                                              %
%                                 Interlude                                    %
%                         for Machine Learning module                          %
%                                                                              %
%******************************************************************************%

% =============================== %
\section*{Interlude - Improve with the Gradient}
% ******************************* %

\begin{figure}[!h]
    \centering
    \includegraphics[scale=0.2]{assets/Improve.png}
    %\caption{The Learning Cycle: Improve}
\end{figure}

% =============================== %
\section*{Multivariate Gradient}
% ******************************* %
From our multivariate linear hypothesis we can derive our multivariate gradient.
It looks a lot like the one we saw during the previous module, but instead of having just two components, the gradient now has as many as there are parameters.
This means that now we need to calculate $\nabla(J)_0,\nabla(J)_1,\dots,\nabla(J)_n$.\\
\newline
If we take the univariate equations we used during the previous module and replace the formula for $\nabla(J)_1$ by a more general $\nabla(J)_j$, we get the following:

$$
\begin{matrix}
\nabla(J)_0 &  = &\frac{1}{m}\sum_{i=1}^{m}(h_{\theta}(x^{(i)}) - y^{(i)}) & \\
\nabla(J)_j & = &\frac{1}{m}\sum_{i=1}^{m}(h_{\theta}(x^{(i)}) - y^{(i)})x_{j}^{(i)} & \text{ for j = 1, ..., n}    
\end{matrix}
$$
Where:
\begin{itemize}
    \item $\nabla(J)$ is a vector of dimension $(n + 1)$, the gradient vector
    \item $\nabla(J)_j$ is the j$^\text{th}$ component of $\nabla(J)$, the partial derivative of $J$ with respect to $\theta_j$
    \item $y$ is a vector of dimension $m$, the vector of expected values
    \item $y^{(i)}$ is a scalar, the i$^\text{th}$ component of vector $y$
    \item $x^{(i)}$ is the feature vector of the i$^\text{th}$ example
    \item $x^{(i)}_j$ is a scalar, the j$^\text{th}$ feature value of the i$^\text{th}$ example
    \item $h_{\theta}(x^{(i)})$ is a scalar, the model's estimation of $y^{(i)}$. (It can also be denoted $\hat{y}^{(i)}$)
\end{itemize}

% =============================== %
\section*{Vectorized Form}
% ******************************* %
As usual, we can use some linear algebra magic to get a more compact (and computationally efficient) formula.
First we can use our convention that each training example has an extra $x_0 = 1$ feature, and replace the gradient formulas above by one single equation that is valid for all $j$ components:

$$
\begin{matrix}
\nabla(J)_j & = &\frac{1}{m}\sum_{i=1}^{m}(h_{\theta}(x^{(i)}) - y^{(i)})x_{j}^{(i)} & \text{ for j = 0, ..., n}
\end{matrix}
$$
And this generic equation can then be rewritten in a vectorized form:

$$
\nabla(J) = \frac{1}{m} {X'}^T(X'\theta - y)
$$  
Where:  
\begin{itemize}
    \item $\nabla(J)$ is the gradient vector of dimension $(n + 1)$
    \item $X'$ is a matrix of dimensions $(m \times (n + 1))$, the design matrix onto which a column of $1$'s was added as the first column
    \item ${X'}^T$ means the matrix has been transposed
    \item $\theta$ is a vector of dimension $(n + 1)$: the parameter vector 
    \item $y$ is a vector of dimension $m$: the vector of expected values
\end{itemize}
The vectorized equation can output the entire gradient vector all at once, in one calculation!\\
\newline
So if you understand the linear algebra operations, you can forget about the equations we presented at the top of the page and simply use the vectorized one.

\newpage
\turnindir{ex03}
\exnumber{03}
\exfiles{vec\_log\_loss.py}
\exforbidden{any function that calculates the derivatives for you}
\makeheaderfilesforbidden

% ================================= %
\section*{Objective}
% --------------------------------- %
Understanding and manipulation of loss function in the context of logistic regression.\\
\\
You must implement the following formula as a function:  

$$
J( \theta) = -\cfrac{1} {m} \lbrack y \cdot \log(\hat{y}) + (\vec{1} - y) \cdot \log(\vec{1} - \hat{y})\rbrack
$$
\\
Where:
\begin{itemize}
  \item $\hat{y}$ is a vector of dimension $m$, the vector of predicted values
  \item $y$ is a vector of dimension $m$, the vector of expected values
  \item $\vec{1}$ is a vector of dimension $m$, a vector full of 1's
\end{itemize}


% ================================= %
\section*{Instructions}
% --------------------------------- %
In the \texttt{vec\_log\_loss.py} file, write the following function as per the instructions below:\\
\\
\begin{minted}[bgcolor=darcula-back,formatcom=\color{lightgrey},fontsize=\scriptsize]{python}
def vec_log_loss_(y, y_hat, eps=1e-15):
    """
    Computes the logistic loss value.
    Args:
        y: has to be an numpy.ndarray, a vector of shape m * 1.
        y_hat: has to be an numpy.ndarray, a vector of shape m * 1.
        eps: epsilon (default=1e-15)
    Returns:
        The logistic loss value as a float.
        None on any error.
    Raises:
        This function should not raise any Exception.
    """
\end{minted}

\hint{
  The purpose of epsilon (eps) is to avoid $log(0)$ errors, it is a very small residual value we add to y.
}

% ================================= %
\section*{Examples}
% --------------------------------- %
\begin{minted}[bgcolor=darcula-back,formatcom=\color{lightgrey},fontsize=\scriptsize]{python}
# Example 1:
y1 = np.array([1]).reshape((-1, 1))
x1 = np.array([4]).reshape((-1, 1))
theta1 = np.array([[2], [0.5]])
y_hat1 = logistic_predict_(x1, theta1)
vec_log_loss_(y1, y_hat1)
# Output:
0.018149927917808714

# Example 2:
y2 = np.array([[1], [0], [1], [0], [1]])
x2 = np.array([[4], [7.16], [3.2], [9.37], [0.56]])
theta2 = np.array([[2], [0.5]])
y_hat2 = logistic_predict_(x2, theta2)
vec_log_loss_(y2, y_hat2)
# Output:
2.4825011602472347

# Example 3:
y3 = np.array([[0], [1], [1]])
x3 = np.array([[0, 2, 3, 4], [2, 4, 5, 5], [1, 3, 2, 7]])
theta3 = np.array([[-2.4], [-1.5], [0.3], [-1.4], [0.7]])
y_hat3 = logistic_predict_(x3, theta3)
vec_log_loss_(y3, y_hat3)
# Output:
2.993853310859968
\end{minted}
% ===========================(fin ex 03)         %
% ============================================== %
\newpage
% ============================================== %
% ===========================(start ex 04)       %
\input{exercises/m08ex04.tex}
% ===========================(fin ex 04)         %
% ============================================== %
\newpage
% ============================================== %
% ===========================(start ex 05)       %
\input{exercises/m08ex05.tex}
% ===========================(fin ex 05)         %
% ============================================== %
\newpage
% ============================================== %
% ===========================(start ex 06)       %
\chapter{Exercise 06}
\extitle{Logistic Regression}
%%******************************************************************************%
%                                                                              %
%                                 Interlude                                    %
%                         for Machine Learning module                          %
%                                                                              %
%******************************************************************************%

% ============================================== %
\section*{Interlude}
% ============================================== %
\subsection*{Linear Regression to the Next Level: Ridge Regression}
% ---------------------------------------------- %

Until now we only talked about L$_2$ regularization and its implication on the calculation of the loss function and gradient for both linear and logistic regression.

Now it's time to use proper terminology:
When we apply L$_2$ regularization on a linear regression model, the new model is called a \textbf{Ridge Regression} model.
Besides that brand-new name, Ridge regression is nothing more than linear regression regularized with L$_2$.

We suggest you watch this nice explanation \href{https://www.youtube.com/watch?v=Q81RR3yKn30}{very nice explanation of Ridge Regularization}.
By the way, this Youtube channel, \texttt{\textit{StatQuest}}, is very good to help you understand the gist of a lot of machine learning concepts.
You will not waste your time watching its statistics and machine learning playlists!

%\newpage
\turnindir{ex06}
\exnumber{06}
\exfiles{my\_logistic\_regression.py}
\exforbidden{sklearn}
\makeheaderfilesforbidden

% ================================= %
\section*{Objective}
% --------------------------------- %
The time to use everything you built so far has (finally) come!\\
\\
Demonstrate your knowledge by implementing a logistic regression classifier using 
the gradient descent algorithm.\\
\\
You must have seen the power of \texttt{numpy} for vectorized operations.
 Well let's make something more concrete with that.\\
\\
You may have taken a look at Scikit-Learn's implementation of logistic regression 
and noticed that the \textbf{sklearn.linear\_model.LogisticRegression} class 
offers a lot of options.\\
\\
The goal of this exercise is to make a simplified but nonetheless useful and powerful
 version, with fewer options.\\
\newpage
% ================================= %
\section*{Instructions}
% --------------------------------- %
In the \texttt{my\_logistic\_regression.py} file, write a \texttt{MyLogisticRegression} 
class as in the instructions given below:\\

\begin{minted}[bgcolor=darcula-back,formatcom=\color{lightgrey},fontsize=\scriptsize]{python}
class MyLogisticRegression():
	"""
	Description:
		My personnal logistic regression to classify things.
	"""
    def __init__(self, theta, alpha=0.001, max_iter=1000):
        self.alpha = alpha
        self.max_iter = max_iter
        self.theta = theta
        ... Your code here ...

	... other methods ...
\end{minted}
\\
You will add at least the following methods:
\begin{itemize}
  \item \texttt{predict\_(self, x)}
  \item \texttt{loss\_elem\_(self, y, yhat)}
  \item \texttt{loss\_(self, y, yhat)}
  \item \texttt{fit\_(self, x, y)}
\end{itemize}
\hint{You have already written these functions, you will just need a 
few adjustments in order for them to work well within your \textbf{MyLogisticRegression} class.}

% ================================= %
\subsection*{Examples}
% --------------------------------- %

\begin{minted}[bgcolor=darcula-back,formatcom=\color{lightgrey},fontsize=\scriptsize]{python}
import numpy as np
from my_logistic_regression import MyLogisticRegression as MyLR
X = np.array([[1., 1., 2., 3.], [5., 8., 13., 21.], [3., 5., 9., 14.]])
Y = np.array([[1], [0], [1]])
thetas = np.array([[2], [0.5], [7.1], [-4.3], [2.09]])
mylr = MyLR(thetas)

# Example 0:
mylr.predict_(X)
# Output:
array([[0.99930437],
       [1.        ],
       [1.        ]])

# Example 1:
mylr.loss_(X,Y)
# Output:
11.513157421577002

# Example 2:
mylr.fit_(X, Y)
mylr.theta
# Output:
array([[ 2.11826435]
       [ 0.10154334]
       [ 6.43942899]
       [-5.10817488]
       [ 0.6212541 ]])

# Example 3:
mylr.predict_(X)
# Output:
array([[0.57606717]
       [0.68599807]
       [0.06562156]])

# Example 4:
mylr.loss_(X,Y)
# Output:
1.4779126923052268
\end{minted}
% ===========================(fin ex 06)         %
% ============================================== %
\newpage
% ============================================== %
% ===========================(start ex 07)       %
\chapter{Exercise 07}
\extitle{Practicing Logistic Regression}
%%******************************************************************************%
%                                                                              %
%                                 Interlude                                    %
%                         for Machine Learning module                          %
%                                                                              %
%******************************************************************************%

% =============================================== %
\section*{Interlude - Introducing Polynomial Models}
% ----------------------------------------------- %

You probably noticed that the method we use is called \textit{linear regression} for a reason:
the model generates all of its predictions on a straight line.
However, we often encounter features that do not have a linear relationship with the predicted variable,
like in the figure below:

\begin{figure}[!h]
    \centering
    \includegraphics[scale=0.6]{assets/polynomial_straight_line.png}
    \caption{Non-linear relationship}
\end{figure}
In that case, we are stuck with a straight line that cannot fit the data points properly.\\
\newline
In this example, what if we could express $y$ not as a function of $x$, but also of $x^2$, and maybe even $x^3$ and $x^4$?
We could make a hypothesis that draws a nice \textbf{curve} that would better fit the data.
That's where polynomial features can help!

% =============================================== %
\section*{Interlude - Polynomial features}
% ----------------------------------------------- %
First we get to do some \textit{feature engineering}.
We create new features by raising our initial $x$ feature to the power of 2, and then 3, 4... as far as we want to go.
For each new feature we need to create a new column in the dataset.

% =============================================== %
\section*{Interlude - Polynomial Hypothesis}
% ----------------------------------------------- %
Now that we created our new features, we can combine them in a linear hypothesis that looks just the same as what we're used to:

$$
\hat{y} = \theta_0 + \theta_1 x  +\theta_2 x^{2} + \dots + \theta_n x^{n}
$$  
It's a little strange because we are building a linear combination, not with different features but with different powers of the same feature.
This is a first way of introducing non-linearity in a regression model!
%\newpage
\turnindir{ex07}
\exnumber{07}
\exfiles{mono\_log.py, multi\_log.py}
\exforbidden{sklearn}
\makeheaderfilesforbidden


% ================================= %
\section*{Objective}
% --------------------------------- %
Now it's time to test your Logistic Regression Classifier on real data!\\
\\ 
To do so, you will use the \textbf{solar\_system\_census\_dataset}.

% ================================= %
\section*{Instructions}
% --------------------------------- %
Some words about the dataset:
\begin{itemize}
  \item You will work with data from the last Solar System Census.
  \item The dataset is divided in two files which can be found in the \texttt{resources} folder: \texttt{solar\_system\_census.csv} and \texttt{solar\_system\_census\_planets.csv}.
  \item The first file contains biometric information such as the height, weight, and bone density of several Solar System citizens.
  \item The second file contains the home planet of each citizen, indicated by its Space Zipcode representation (i.e. one number for each planet... :)).
\end{itemize}

\newpage
As you should know, Solar citizens come from four registered areas (zipcodes): 
\begin{itemize}
  \item The flying cities of Venus (0)
  \item United Nations of Earth (1)
  \item Mars Republic (2)
  \item The Asteroids' Belt colonies (3)
\end{itemize}
\bigskip
You are expected to produce 2 programs that will use Logistic Regression to predict which planet 
each citizen comes from, based on the other variables found in the census dataset.\\
\\
But wait... what? There are four different planets! How do you make a classifier 
discriminate between 4 categories?!!\\
\\
Keep calm and take a sip of water, we'll go one step at the time ...\\

% ================================= %
\subsection*{One Label to Discriminate Them All}
% --------------------------------- %
You already wrote a Logistic Regression Classifier that can discriminate between two classes. We can use it 
to solve this problem!\\
\\
Let's start by having it discriminate between citizens who come from your favorite planet and everybody else!\\
\\
Your program (in \texttt{mono\_log.py}) will:
\begin{enumerate}
  \item Take an argument: \texttt{--zipcode=x} with $x$ being $0$, $1$, $2$ or $3$.
        If no argument, usage will be displayed.
  \item Split the dataset into a training and a test set.
  \item Select your favorite Space Zipcode and generate a new \texttt{numpy.array} to label each citizen according to your new selection criterion:
  \begin{itemize}
    \item $1$ if the citizen's zipcode corresponds to your favorite planet.
    \item $0$ if the citizen has another zipcode.
  \end{itemize}
  \item Train a logistic model to predict if a citizen comes from your favorite planet or not, using your brand new label.
  \item Calculate and display the fraction of correct predictions over the total number of predictions based on the test set.
  \item Plot 3 scatter plots (one for each pair of citizen features) with the dataset and the final prediction of the model.
\end{enumerate}
\hint{You can use normalization on your dataset but the question is ... Should you?}
\noindent{You now have a model that can discriminate between citizens that come 
from one specific planet and everyone else.}\\
\\
It's a first step, and a good one, but we still have work to do before we can classify citizens 
among the four planets of our dataset!\\
\\
So how does \textbf{Multiclass Logistic Regression} work?\\
% ================================= %
\subsection*{One Versus All}
% --------------------------------- %
The idea now is to apply what is called \textbf{one-versus-all classification}. 
As you will see, this is quite straightforward.\\
\\
Your program (in \texttt{multi\_log.py}) will:
\begin{enumerate}
  \item Split the dataset into a training and a test set.
  \item Train 4 logistic regression classifiers to discriminate each class 
  from the others (the way you did in part one).
  \item Predict for each example the class according to each classifier and
   select the one with the highest output probability score.
  \item Calculate and display the fraction of correct predictions over the 
  total number of predictions based on the test set.
  \item Plot 3 scatter plots (one for each pair of citizen features) with the
   dataset and the final prediction of the model.
\end{enumerate}

% ================================= %
\section*{Examples}
% --------------------------------- %
If a cititzen got the following classification probabilities: 
\begin{itemize}
  \item Planet 0 vs all: $0.38$
  \item Planet 1 vs all: $0.51$
  \item Planet 2 vs all: $0.12$
  \item Planet 3 vs all: $0.89$
\end{itemize}

Then the citizen should be classified as coming from \textit{Planet 3}. 
% ===========================(fin ex 07)         %
% ============================================== %
\newpage
% ============================================== %
% ===========================(start ex 08)       %
\input{exercises/m08ex08.tex}
% ===========================(fin ex 08)         %
% ============================================== %
\newpage
% ============================================== %
% ===========================(start ex 09)       %
\chapter{Exercise 09}
\extitle{Confusion Matrix}
%%******************************************************************************%
%                                                                              %
%                                 Interlude                                    %
%                         for Machine Learning module                          %
%                                                                              %
%******************************************************************************%

% ============================================== %
\section*{Interlude - Lost in Overfitting}
% ---------------------------------------------- %

The two previous exercises lead you, dear reader, to a very dangerous territory: the realm of \textbf{overfitting}.\\
You did not see it coming but now, you are in a bad situation...\\
\\
By increasing the polynomial degree of your model, you increased its \textbf{complexity}.  
Is it wrong?
Not always.
Some models are indeed very complex because the relationships they represent are very complex as well.\\
\\
But, if you look at the plots for the previous exercise's \textit{best model}, you should feel that something is wrong...\\
\\
% ============================================== %
\section*{Interlude - Something is rotten in the state of our model...}
% ---------------------------------------------- %
Take a look at the following plot. 

\begin{figure}[!h]
    \centering
    \includegraphics[scale=0.6]{assets/overfitt.png}
    \caption{Overfitting hypothesis}
\end{figure}

You can see that the prediction line fits each data point perfectly, but completely misses out on capturing the relationship between $x$ and $y$ properly.
And now, if we add some brand new data points to the dataset, we see that the predictions on those new examples are way off.

\begin{figure}[!h]
    \centering
    \includegraphics[scale=0.6]{assets/overfitt_with_dots.png}
    \caption{Generalization errors resulting from overfitting}
\end{figure}
This situation is called overfitting, because the model is doing an excessively good job at fitting the data.
It is literally bending over backward to account for the data's mini details.
But most of the data's irregularities are just noise, and they should in fact be ignored.
So because the model overfits, it can't generalize to new data.

% ============================================== %
\section*{Interlude - The training set, the test set, and the happy data scientist}
% ---------------------------------------------- %
To be able to detect overfitting, \textbf{you should always evaluate your model on new data}.\\
\\
New data means, data that your model hasn't seen during training.\\
\\
It's the only way to make sure your model isn't \textit{recalling}.
To do so, now and forever, you must always divide your dataset in (at least) two parts: one for the training, and one for the evaluation of your model.
%\newpage
\turnindir{ex09}
\exnumber{09}
\exfiles{confusion\_matrix.py}
\exforbidden{None}
\makeheaderfilesforbidden

% ================================= %
\section*{Objective}
% --------------------------------- %
Manipulate and experiment with the concept of \textbf{Confusion Matrix}.\\
\\
The goal of this exercise is to reimplement the function \texttt{confusion\_matrix} 
available in \textbf{sklearn.metrics} and to understand what a confusion matrix is, and 
what it is used for.\\

% ================================= %
\section*{Instructions}
% --------------------------------- %
For the sake of simplicity, we will only ask you to use three parameters.\\
\
Be careful to respect the order :  \textbf{true labels are rows and predicted labels are columns}\\

\begin{center}
  \begin{tabular}{|c|c|c|c|}
    \cline{3-4}
    \multicolumn{2}{c|}{\multirow{2}{*}{}}  & \multicolumn{2}{|c|}{predicted labels} \\ \cline{3-4}
    \multicolumn{2}{c|}{}       & label 1 & label 2 \\
    \hline
    \multirow{2}{*}{true label} & label 1 &         &         \\
    \cline{2-4}
                                & label 2 &         &         \\
    \hline
  \end{tabular}
\end{center}
In the \texttt{confusion\_matrix.py} file, write the following 
function as per the instructions given below:

\begin{minted}[bgcolor=darcula-back,formatcom=\color{lightgrey},fontsize=\scriptsize]{python}
def confusion_matrix_(y_true, y_hat, labels=None):
    """
    Compute confusion matrix to evaluate the accuracy of a classification.
    Args:
        y_true: numpy.ndarray for the correct labels
        y_hat: numpy.ndarray for the predicted labels
        labels: Optional, a list of labels to index the matrix.
                This may be used to reorder or select a subset of labels. (default=None)
    Returns: 
        The confusion matrix as a numpy ndarray.
        None on any error.
    Raises:
        This function should not raise any Exception.
    """
    ... Your code ...
\end{minted}


% ================================= %
\section*{Examples}
% --------------------------------- %
\begin{minted}[bgcolor=darcula-back,formatcom=\color{lightgrey},fontsize=\scriptsize]{python}
import numpy as np
from sklearn.metrics import confusion_matrix

y_hat = np.array([['norminet'], ['dog'], ['norminet'], ['norminet'], ['dog'], ['bird']])
y = np.array([['dog'], ['dog'], ['norminet'], ['norminet'], ['dog'], ['norminet']])

# Example 1: 
## your implementation
confusion_matrix_(y, y_hat)
## Output:
array([[0 0 0]
       [0 2 1]
       [1 0 2]])
## sklearn implementation
confusion_matrix(y, y_hat)
## Output:
array([[0 0 0]
       [0 2 1]
       [1 0 2]])

# Example 2:
## your implementation
confusion_matrix_(y, y_hat, labels=['dog', 'norminet'])
## Output:
array([[2 1]
       [0 2]])
## sklearn implementation
confusion_matrix(y, y_hat, labels=['dog', 'norminet'])
## Output:
array([[2 1]
       [0 2]])
\end{minted}

\section*{Optional part}

\subsection*{Objective(s):}

For a more visual version, you can add an option to your previous 
confusion\_matrix\_ function to return a \texttt{pandas.DataFrame} instead
 of a numpy array.\\

\subsection*{Instructions:}

In the \texttt{confusion\_matrix.py} file, write the following function 
as per the instructions given below:\\
\\
\begin{minted}[bgcolor=darcula-back,formatcom=\color{lightgrey},fontsize=\scriptsize]{python}
def confusion_matrix_(y_true, y_hat, labels=None, df_option=False):
    """
    Compute confusion matrix to evaluate the accuracy of a classification.
    Args:
        y_true:     a numpy.ndarray for the correct labels
        y_hat:      a numpy.ndarray for the predicted labels
        labels:     optional, a list of labels to index the matrix. 
                        This may be used to reorder or select a subset of labels. (default=None)
        df_option:  optional, if set to True the function will return 
                        a pandas DataFrame instead of a numpy array. (default=False)
    Returns: 
        Confusion matrix as a numpy ndarray or a pandas DataFrame according to df_option value.
        None on any error.
    Raises:
        This function should not raise any Exception.
    """
    ... Your code ...
\end{minted}

\subsection*{Examples:}

\begin{minted}[bgcolor=darcula-back,formatcom=\color{lightgrey},fontsize=\scriptsize]{python}
import numpy as np
y_hat = np.array(['norminet', 'dog', 'norminet', 'norminet', 'dog', 'bird'])
y = np.array(['dog', 'dog', 'norminet', 'norminet', 'dog', 'norminet'])

# Example 1: 
confusion_matrix_(y, y_hat, df_option=True)
# Output:
           bird  dog  norminet
 bird         0    0         0
 dog          0    2         1
 norminet     1    0         2

# Example 2:
confusion_matrix_(y, y_hat, labels=['bird', 'dog'], df_option=True)
# Output:
           bird  dog
 bird         0    0
 dog          0    2
\end{minted}

\info{
  If you fail this exercise on your first attempt, 
  the spirit of Norminet will haunt you until the end of times. MEOWWW ... 
  Yeah, you'd better do it right or you are in trouble my friend, biiig trouble!
}
% ===========================(fin ex 09)         %
% ============================================== %
\newpage
% ============================================== %
% ===========================(Conclusion)        %
\chapter{Conclusion - What you have learnt}

You are now done with the ML Bootcamp module03, well done !\\
\newline
Based on all the notions and problems
tackled today, you should be able to discuss and answer the following questions:

\begin{enumerate}
  \item Why do we use the logistic hypothesis for a classfication problem 
  rather than a linear hypothesis?
  \item What is the decision boundary?
  \item In the case we decide to use a linear hypothesis to tackle a 
  classification problem, why the classification of some data points can be
   altered by considering more examples (for example, extra data points with extrem ordinate)?
  \item In a one versus all classification approach, how many logisitic regressors do we
   need to distinguish between N classes?
  \item Can you explain the difference between accuracy and precision?
   What is the type I and type II errors?
  \item What is the interest of the F1-score?
\end{enumerate}
\info{
  Your feedbacks are essential for us to improve these bootcamps !\newline
  Please take a few minutes to tell us about your experience in this module by filling 
  \href{https://docs.google.com/forms/d/e/1FAIpQLSe2hLbxYFj7CJDqKXUSqYVEXG0DiQb8FGkLQEUIT4__gshtqA/viewform?usp=sharing}{this form}. Thank you in advance ! 
}
% ===========================(Conclusion)        %
% ============================================== %
\newpage
\section*{Contact}
% --------------------------------- %
You can contact 42AI by email: \href{mailto:contact@42ai.fr}{contact@42ai.fr}\\
\newline
Thank you for attending 42AI's Machine Learning Bootcamp !

% ================================= %
\section*{Acknowledgements}
% --------------------------------- %
The Python \& ML bootcamps are the result of a collective effort. We would like to thank:\\
\begin{itemize}
  \item Maxime Choulika (cmaxime),
  \item Pierre Peigné (ppeigne),
  \item Matthieu David (mdavid),
  \item Quentin Feuillade--Montixi (qfeuilla, quentin@42ai.fr)
  \item Mathieu Perez (maperez, mathieu.perez@42ai.fr)
\end{itemize}
who supervised the creation and enhancements of the present transcription.\\
\begin{itemize}
  \item Louis Develle (ldevelle, louis@42ai.fr)
  \item Owen Roberts (oroberts)
  \item Augustin Lopez (aulopez)
  \item Luc Lenotre (llenotre)
  \item Amric Trudel (amric@42ai.fr)
  \item Benjamin Carlier (bcarlier@student.42.fr)
  \item Pablo Clement (pclement@student.42.fr)
  \item Amir Mahla (amahla, amahla@42ai.fr)
\end{itemize}
for your investment in the creation and development of these modules.\\
\begin{itemize}
    \item All prior participants who took a moment to provide their feedbacks, and help us improve these bootcamps !
  \end{itemize}

\vfill
\doclicenseThis

% ================================= %

\end{document}
