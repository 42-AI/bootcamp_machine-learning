\chapter{Exercise 02}
%******************************************************************************%
%                                                                              %
%                                 Interlude                                    %
%                         for Machine Learning module                          %
%                                                                              %
%******************************************************************************%

% =============================== %
\section*{Interlude - Evaluate}
% ------------------------------- %

\begin{figure}[!h]
    \centering
    \includegraphics[scale=0.2]{assets/Evaluate.png}
    %\caption{The Learning Cycle: Evaluate}
\end{figure}

% =============================== %
\section*{Back to the Loss Function}
% ------------------------------- %
How is our model doing?\\
To evaluate our model, remember that we have already used a \textbf{metric} called the \textbf{loss function} (also known as \textbf{cost function}).
The loss function is basically just a measure of how wrong the model is, in all of its predictions.\\
\newline
Two modules ago, we defined the loss function as the average of the squared distances between each prediction and its expected value (distances represented by the dotted lines in the figure below):

\begin{figure}[!h]
    \centering
    \includegraphics[scale=0.5]{assets/bad_pred_with_distance.png}
    \caption{Distances between predicted and expected values}
\end{figure}
\newpage
\noindent{The formula was the following:}

$$
J(\theta) = \frac{1}{2m}\sum_{i=1}^{m}(\hat{y}^{(i)} - y^{(i)})^2
$$
\\
And its vectorized form:

$$
\begin{matrix}
J(\theta) = \frac{1}{2m}(\hat{y} - y)\cdot(\hat{y}- y)
\end{matrix}
$$
\\
\textit{So, now that we moved to multivariate linear regression, what does it change?}\\
\newline
You may have noticed that variables such as $x_j$ and $\theta_j$ are not in the equation.
Indeed, the loss function only uses the predictions ($\hat{y}$) and the expected values ($y$), 
so the inner workings of the model do not have an impact on its evaluation metric.\\
\\
This means we can use the exact same loss function as we did before!

\newpage
\extitle{Logistic Loss Function}
\turnindir{ex02}
\exnumber{02}
\exfiles{log\_loss.py}
\exforbidden{None}
\makeheaderfilesforbidden

% ================================= %
\section*{Objective}
% --------------------------------- %
Understanding and manipulation of the loss function in the context of logistic regression.\\
\\
You must implement the following formula as a function:  

$$
J( \theta) = -\cfrac{1} {m} \lbrack \sum_{i = 1}^{m} y^{(i)}\log(\hat{y}^{(i)})) + (1 - y^{(i)})\log(1 - \hat{y}^{(i)})\rbrack
$$
Where:
\begin{itemize}
  \item $\hat{y}$ is a vector of dimension $m$, the vector of predicted values
  \item $\hat{y}^{(i)}$ is the $i^{th}$ component of the $\hat{y}$ vector
  \item $y$ is a vector of dimension $m$, the vector of expected values
  \item $y^{(i)}$ is the $i^{th}$ component of the $y$ vector
\end{itemize}

% ================================= %
\section*{Instructions}
% --------------------------------- %
In the \texttt{log\_loss.py} file, write the following function as per the instructions below:
\\
\par
\begin{minted}[bgcolor=darcula-back,formatcom=\color{lightgrey},fontsize=\scriptsize]{python}
def log_loss_(y, y_hat, eps=1e-15):
    """
    Computes the logistic loss value.
    Args:
        y: has to be an numpy.ndarray, a vector of shape m * 1.
        y_hat: has to be an numpy.ndarray, a vector of shape m * 1.
        eps: has to be a float, epsilon (default=1e-15)
    Returns:
        The logistic loss value as a float.
        None on any error.
    Raises:
        This function should not raise any Exception.
    """
    ... Your code ...
\end{minted}

\hint{
  The logarithmic function isn't defined in $0$.
  This means that if $y^{(i)} = 0$ you will get an error when you try to compute $log(y^{(i)})$.
  The purpose of the \texttt{eps} argument is to avoid $log(0)$ errors.
  It is a very small residual value we add to \texttt{y}, also referred to as `epsilon`.
}

% ================================= %
\section*{Examples}
% --------------------------------- %
\begin{minted}[bgcolor=darcula-back,formatcom=\color{lightgrey},fontsize=\scriptsize]{python}
# Example 1:
y1 = np.array([1]).reshape((-1, 1))
x1 = np.array([4]).reshape((-1, 1))
theta1 = np.array([[2], [0.5]])
y_hat1 = logistic_predict_(x1, theta1)
log_loss_(y1, y_hat1)
# Output:
0.01814992791780973

# Example 2:
y2 = np.array([[1], [0], [1], [0], [1]])
x2 = np.array([[4], [7.16], [3.2], [9.37], [0.56]])
theta2 = np.array([[2], [0.5]])
y_hat2 = logistic_predict_(x2, theta2)
log_loss_(y2, y_hat2)
# Output:
2.4825011602474483

# Example 3:
y3 = np.array([[0], [1], [1]])
x3 = np.array([[0, 2, 3, 4], [2, 4, 5, 5], [1, 3, 2, 7]])
theta3 = np.array([[-2.4], [-1.5], [0.3], [-1.4], [0.7]])
y_hat3 = logistic_predict_(x3, theta3)
log_loss_(y3, y_hat3)
# Output:
2.9938533108607053
\end{minted}

\info{
  This function is called \textbf{Cross-Entropy loss}, or \textbf{logistic loss}.
  For more information you can look at \href{https://en.wikipedia.org/wiki/Cross_entropy\#Cross-entropy\_error\_function\_and\_logistic\_regression}{this section}
  of the Cross entropy Wikipedia article.
}