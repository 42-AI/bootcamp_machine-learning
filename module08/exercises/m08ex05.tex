\chapter{Exercise 05}
\extitle{Vectorized Logistic Gradient}
%******************************************************************************%
%                                                                              %
%                                 Interlude                                    %
%                         for Machine Learning module                          %
%                                                                              %
%******************************************************************************%

\section*{Interlude - Normalization}

The values inside the $x$ vector can vary quite a lot in magnitude,
depending on the type of data you are working with.
For example, if your dataset contains distances between planets in km, the numbers will be huge.
On the other hand, if you are working with planet masses expressed as a fraction of the solar system's total mass, the numbers will be very small (between 0 and 1).
Both cases may slow down convergence in Gradient Descent (or even sometimes prevent convergence at all).
To avoid that kind of situation, normalization is a very effective way to proceed.


The idea behind this technique is straightforward: \textbf{scaling the data}.  


With normalization, you can transform your $x$ vector into a new $x'$ vector whose values range between $[-1, 1]$ more or less. Doing this allows you to see much more easily how a training example compares to the other ones:
\begin{itemize}
    \item If an $x'$ value is close to $1$, you know it's among the largest in the dataset
    \item If an $x'$ value is close to $0$, you know it's close to the median
    \item If an $x'$ value is close to $-1$, you know it's among the smallest
\end{itemize}

So with the upcoming normalization techniques, you'll be able to map your data to two different value ranges: $[0, 1]$ or $[-1, 1]$. Your algorithm will like it and thank you for it.  

\newpage
\turnindir{ex05}
\exnumber{05}
\exfiles{vec\_log\_gradient.py}
\exforbidden{any function that performs derivatives for you}
\makeheaderfilesforbidden

% ================================= %
\section*{Objective}
% --------------------------------- %
Understand and manipulate the gradient in the context of logistic formulation.\\
\\
You must implement the following formula as a function:

$$
\nabla(J) = \cfrac{1}{m} X'^T(h_\theta(X) - y)
$$  
\\
Where:  
\begin{itemize}
  \item $\nabla(J)$ is the gradient vector of length $(n + 1)$
  \item $X'$ is a matrix of dimensions $(m \times (n + 1))$, the design matrix onto which a column of ones was added as the first column
  \item $X'^T$ means the matrix has been transposed
  \item $h_\theta(X)$ is a vector of length $m$, the vector of predicted values
  \item $y$ is a vector of dimension $m$, the vector of expected values
\end{itemize}


% ================================= %
\section*{Instructions}
% --------------------------------- %
In the \texttt{vec\_log\_gradient.py} file, write the following function as per the instructions given below:
\\
\par
\begin{minted}[bgcolor=darcula-back,formatcom=\color{lightgrey},fontsize=\scriptsize]{python}
def vec_log_gradient(x, y, theta):
    """Computes a gradient vector from three non-empty numpy.ndarray, without any for-loop. The three arrays must have compatible shapes.
    Args:
      x: has to be an numpy.ndarray, a matrix of shape m * n.
      y: has to be an numpy.ndarray, a vector of shape m * 1.
      theta: has to be an numpy.ndarray, a vector (n +1) * 1.
    Returns:
      The gradient as a numpy.ndarray, a vector of shape n * 1, containg the result of the formula for all j.
      None if x, y, or theta are empty numpy.ndarray.
      None if x, y and theta do not have compatible shapes.
    Raises:
      This function should not raise any Exception.
    """
    ... Your code ...
\end{minted}


% ================================= %
\section*{Examples}
% --------------------------------- %
\begin{minted}[bgcolor=darcula-back,formatcom=\color{lightgrey},fontsize=\scriptsize]{python}
# Example 1:
y1 = np.array([1]).reshape((-1, 1))
x1 = np.array([4]).reshape((-1, 1))
theta1 = np.array([[2], [0.5]])

vec_log_gradient(x1, y1, theta1)
# Output:
array([[-0.01798621],
       [-0.07194484]])

# Example 2: 
y2 = np.array([[1], [0], [1], [0], [1]])
x2 = np.array([[4], [7.16], [3.2], [9.37], [0.56]])
theta2 = np.array([[2], [0.5]])

vec_log_gradient(x2, y2, theta2)
# Output:
array([[0.3715235 ],
       [3.25647547]])

# Example 3: 
y3 = np.array([[0], [1], [1]])
x3 = np.array([[0, 2, 3, 4], [2, 4, 5, 5], [1, 3, 2, 7]])
theta3 = np.array([[-2.4], [-1.5], [0.3], [-1.4], [0.7]])

vec_log_gradient(x3, y3, theta3)
# Output:
array([[-0.55711039],
       [-0.90334809],
       [-2.01756886],
       [-2.10071291],
       [-3.27257351]])
\end{minted}