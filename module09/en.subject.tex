% vim: set ts=4 sw=4 tw=80 noexpandtab:

\documentclass{42-en}

%******************************************************************************%
%                                                                              %
%                                   Prologue                                   %
%                                                                              %
%******************************************************************************%
\usepackage[
    type={CC},
    modifier={by-nc-sa},
    version={4.0},
]{doclicense}
\usepackage{amsmath} % The amsmath package provides commands to typeset matrices with different delimiters. 
\usepackage{epigraph}
\setlength\epigraphwidth{.95\textwidth}
\usepackage{multirow}
\usepackage{cancel}
%****************************************************************%
%                  Re/definition of commands                     %
%****************************************************************%

\newcommand{\ailogo}[1]{\def \@ailogo {#1}}\ailogo{assets/42ai_logo.pdf}

%%  Redefine \maketitle
\makeatletter
\def \maketitle {
  \begin{titlepage}
    \begin{center}
	%\begin{figure}[t]
	  %\includegraphics[height=8cm]{\@ailogo}
	  \includegraphics[height=8cm]{assets/42ai_logo.pdf}
	%\end{figure}
      \vskip 5em
      {\huge \@title}
      \vskip 2em
      {\LARGE \@subtitle}
      \vskip 4em
    \end{center}
    %\begin{center}
	  %\@author
    %\end{center}
	%\vskip 5em
  \vfill
  \begin{center}
    \emph{\summarytitle : \@summary}
  \end{center}
  \vspace{2cm}
  %\vskip 5em
  %\doclicenseThis
  \end{titlepage}
}
\makeatother

\makeatletter
\def \makeheaderfilesforbidden
{
  \noindent
  \begin{tabularx}{\textwidth}{|X X  X X|}
    \hline
  \multicolumn{1}{|>{\raggedright}m{1cm}|}
  {\vskip 2mm \includegraphics[height=1cm]{assets/42ai_logo.pdf}} &
  \multicolumn{2}{>{\centering}m{12cm}}{\small Exercise : \@exnumber } &
  \multicolumn{1}{ >{\raggedleft}p{1.5cm}|}
%%              {\scriptsize points : \@exscore} \\ \hline
              {} \\ \hline

  \multicolumn{4}{|>{\centering}m{15cm}|}
              {\small \@extitle} \\ \hline

  \multicolumn{4}{|>{\raggedright}m{15cm}|}
              {\small Turn-in directory : \ttfamily
                $ex\@exnumber/$ }
              \\ \hline
  \multicolumn{4}{|>{\raggedright}m{15cm}|}
              {\small Files to turn in : \ttfamily \@exfiles }
              \\ \hline

  \multicolumn{4}{|>{\raggedright}m{15cm}|}
              {\small Forbidden functions : \ttfamily \@exforbidden }
              \\ \hline

%%  \multicolumn{4}{|>{\raggedright}m{15cm}|}
%%              {\small Remarks : \ttfamily \@exnotes }
%%              \\ \hline
\end{tabularx}
%% \exnotes
\exrules
\exmake
\exauthorize{None}
\exforbidden{None}
\extitle{}
\exnumber{}
}
\makeatother

%%  Syntactic highlights
\makeatletter
\newenvironment{pythoncode}{%
  \VerbatimEnvironment
  \usemintedstyle{emacs}
  \minted@resetoptions
  \setkeys{minted@opt}{bgcolor=black,formatcom=\color{lightgrey},fontsize=\scriptsize}
  \begin{figure}[ht!]
    \centering
    \begin{minipage}{16cm}
      \begin{VerbatimOut}{\jobname.pyg}}
{%[
      \end{VerbatimOut}
      \minted@pygmentize{c}
      \DeleteFile{\jobname.pyg}
    \end{minipage}
\end{figure}}
\makeatother

\usemintedstyle{native}

\begin{document}

% =============================================================================%
%                     =====================================                    %

\title{Machine Learning - Module 04}
\subtitle{Regularization}
\author{
  Maxime Choulika (cmaxime), Pierre Peigné (ppeigne), Matthieu David (mdavid)
}
\summary
{
  Today you will fight overfitting!
  You will discover the concepts of regularization and how to implement it into the algortihms you already saw until now.
}
\maketitle
%******************************************************************************%
%                                                                              %
%                        Section usefull ressources                            %
%                          for ML Modules                                      %
%                                                                              %
%******************************************************************************%


\chapter*{Notions and ressources}

\section*{Notions of the module}
Multivariate linear hypothesis, multivariate linear gradient descent, polynomial models. 
Training and test sets, overfitting.

\section*{Useful Ressources}

You are strongly advise to use the following resource:
\href{https://www.coursera.org/learn/machine-learning/home/week/2}{Machine Learning MOOC - Stanford}
Here are the sections of the MOOC that are relevant for today's exercises: 

\subsection*{Week 2}

\subsubsection*{Multivariate Linear Regression}
\begin{itemize}
  \item Multiple Features (Video + Reading)
  \item Gradient Descent for Multiple Variables (Video + Reading)
  \item Gradient Descent in Practice I- Feature Scaling (Video + Reading)
  \item Gradient Descent in Practice II- Learning Rate (Video + Reading)
  \item Features and Polynomial Regression (Video + Reading)
  \item Review (Reading + Quiz)
\end{itemize}

\input{en.py_proj.tex}
\newpage
\tableofcontents
\startexercices

%                     =====================================                    %
% =============================================================================%


%******************************************************************************%
%                                                                              %
%                                   Exercises                                  %
%                                                                              %
%******************************************************************************%

% ============================================== %
% ===========================(start ex 00)       %
\chapter{Exercise 00}
\extitle{Polynomial models II}
%%******************************************************************************%
%                                                                              %
%                                 Interlude                                    %
%                         for Machine Learning module                          %
%                                                                              %
%******************************************************************************%

% =============================== %
\section*{Interlude}
% =============================== %
\subsection*{Classification: The Art of Labelling Things}
% ******************************* %
Over the last three modules you have implemented your first machine learning algorithm.\\
\\
You are now familiar the three-steps cycle we follow when we build \textbf{learning algorithms}:
\\
\begin{figure}[!h]
    \centering
    \includegraphics[scale=0.25]{assets/Default.png}
    %\caption{The Learning Cycle}
\end{figure}
\\
The first algorithm you discovered, \textbf{Multivariate Linear Regression}, can now be used to predict a numerical value, based on several features.
This algorithm uses gradient descent to optimize its loss function.\\
\\
Now let's introduce your first \textbf{classification algorithm}: the notorious \textbf{Logistic Regression.}
\hint{regression vs classification; discrete vs continuous values}
\newpage
\noindent{\textbf{Logistic regression} performs a \textit{classification task}, which means that you are not predicting a numerical value (like price, age, grades...) 
but \textbf{categories}, or \textbf{labels} (like dog, cat, sick/healty...)}.
\\
\warn{
    Don't be confused by the word \textit{'regression'} in \textbf{Logistic Regression}.
    It really is a \textit{classification task}! The name is a bit tricky but you will quickly get used to it.
    Once again: \textbf{Logistic Regression is a classification algorithm} which assigns a label/category/class to a given example.
}
\info{
    In this module we will use the following terms interchangeably: \textbf{class}, \textbf{category}, and \textbf{label}.
    They all refer to the \textit{groups} to which each training example can be assigned to, in a classification task.
}

% =============================== %
\subsection*{Predict I: Introducing the Sigmoid Function}
% ******************************* %

\begin{figure}[!h]
    \centering
    \includegraphics[scale=0.25]{assets/Predict.png}
    %\caption{The Learning Cycle - Predict}
\end{figure}

% =============================== %
\subsubsection*{Formulating a Hypothesis}
% ******************************* %
Remember that a hypothesis, denoted $h(\theta)$, is an equation that combines a set of \textbf{features} (that characterizes an example) with \textbf{parameters} in order to output a \textbf{prediction}.\\
\\
Remember the hypothesis we used in linear regression?\\
$$
h(\theta) = \theta_0 + \theta_{1} x_{1}^{(i)} + \dots + \theta_{n} x_{n}^{(i)} = \theta \cdot x'^{(i)}
$$
\newline
It worked fine to predict continuous values, but could we also use it to tell, for example, 
if a patient is sick or not?
That's a yes-or-no question, so the output from the hypothesis function should reflect that.\\
\\
To get started, we will assign each class a numerical value: sick patients will be 
assigned a value of 1, and healthy patients will be assigned a value of 0.\\
The goal will be to build a hypothesis that outputs a probability that a patient is sick as a float number in the range of [0, 1].\\
\\
The good news is that we can keep the linear equation we already worked with!\\
\\
All we need to do is squash its output through another function that is bounded between 0 and 1.\\
\\
That's the purpose of the \textbf{Sigmoid function} and your next assignment is to implement it!

%\newpage
\turnindir{ex00}
\exnumber{00}
\exfiles{polynomial\_model\_extended.py}
\exforbidden{sklearn}
\makeheaderfilesforbidden

% ================================== %
\section*{Objective}
% ---------------------------------- %
Create a function that takes a matrix $X$ of dimensions $(m \times n)$ and an integer $p$ as input, and returns a matrix of dimension $(m \times (np))$.
For each column $x_j$ of the matrix $X$, the new matrix contains
$x_j$ raised to the power of $k$, for $k = 1, 2, ..., p$ :

$$
x_1  \mid  \ldots  \mid  x_n  \mid  x_1^2  \mid  \ldots  \mid  x_n^2  \mid  \ldots  \mid  x_1^p  \mid  \ldots  \mid  x_n^p
$$

% ================================== %
\section*{Instructions}
% ---------------------------------- %
In the \texttt{polynomial\_model\_extended.py} file, write the following function as per the instructions given below:

\begin{minted}[bgcolor=darcula-back,formatcom=\color{lightgrey},fontsize=\scriptsize]{python}
def add_polynomial_features(x, power):
	"""Add polynomial features to matrix x by raising its columns to every power in the range of 1 up to the power given in argument.  
	Args:
		x: has to be an numpy.ndarray, a matrix of shape m * n.
		power: has to be an int, the power up to which the columns of matrix x are going to be raised.
	Returns:
		The matrix of polynomial features as a numpy.ndarray, of shape m * (np), containg the polynomial feature values for all training examples.
		None if x is an empty numpy.ndarray.
	Raises:
		This function should not raise any Exception.
	"""
	... Your code ...
\end{minted}


% ================================== %
\section*{Examples}
% ---------------------------------- %

\begin{minted}[bgcolor=darcula-back,formatcom=\color{lightgrey},fontsize=\scriptsize]{python}
import numpy as np
x = np.arange(1,11).reshape(5, 2)

# Example 1:
add_polynomial_features(x, 3)
# Output:
array([[   1,    2,    1,    4,    1,    8],
		[   3,    4,    9,   16,   27,   64],
		[   5,    6,   25,   36,  125,  216],
		[   7,    8,   49,   64,  343,  512],
		[   9,   10,   81,  100,  729, 1000]])

# Example 2:
add_polynomial_features(x, 4)
# Output:
array([[    1,     2,     1,     4,     1,     8,     1,    16],
		[    3,     4,     9,    16,    27,    64,    81,   256],
		[    5,     6,    25,    36,   125,   216,   625,  1296],
		[    7,     8,    49,    64,   343,   512,  2401,  4096],
		[    9,    10,    81,   100,   729,  1000,  6561, 10000]])
\end{minted}

% ===========================(fin ex 00)         %
% ============================================== %

\newpage

% ============================================== %
% ===========================(start ex 01)       %
\chapter{Exercise 01}
\extitle{L2 Regularization}
%******************************************************************************%
%                                                                              %
%                                 Interlude                                    %
%                         for Machine Learning module                          %
%                                                                              %
%******************************************************************************%

% =============================== %
\section*{Linear Algebra Tricks part II}
% ******************************* %

If you tried to run your code on a very large dataset, you would find that it sometimes takes a (very) long time to execute!
That's because it doesn't use the power of Python libraries that are optimized for matrix operations.\\
\newline
Remember the linear algebra trick from the previous module? Let's use it again!  
If you concatenate a column of $1$'s to the left of the $x$ vector, you get what we called matrix $X'$.   
$$
X' = \begin{bmatrix} 1 & x^{(1)} \\ \vdots & \vdots \\ 1 & x^{(m)}\end{bmatrix}
$$
This transformation is very convenient because we can rewrite each $1$ as $x_0^{(i)}$, and each $x^{(i)}$ as $x_1^{(i)}$.
So now the $X'$ matrix looks like this:
$$
X' = \begin{bmatrix} x_0^{(1)} & x_1^{(1)} \\ \vdots & \vdots \\ x_0^{(m)} & x_1^{(m)}\end{bmatrix}
$$
Notice that each $x^{(i)}$ example becomes a vector made of $(x^{(i)}_0, x^{(i)}_1)$.  
The $0$ and $1$ indices on the $x$ features correspond to the indices of the $\theta$ parameters with which they will be multiplied.\\
\newline
Why does this matter?
Well, if we take the equation from the previous exercise:

$$
\nabla(J)_0 = \frac{1}{m}\sum_{i=1}^{m}(h_{\theta}(x^{(i)}) - y^{(i)})
$$
We can multiply it by $1$ without changing its value:
$$
\nabla(J)_0 = \frac{1}{m}\sum_{i=1}^{m}(h_{\theta}(x^{(i)}) - y^{(i)}) \cdot 1
$$
And rewrite $1$ as $x_0^{(i)}$:
$$
\nabla(J)_0 = \frac{1}{m}\sum_{i=1}^{m}(h_{\theta}(x^{(i)}) - y^{(i)})x_{0}^{(i)}
$$
This means that the equation for $\nabla(J)_0$ is now similar to the equation we had for $\nabla(J)_1$, so they can both be captured by ONE \textbf{generic equation}:
$$
\begin{matrix}
\nabla(J)_j = \frac{1}{m}\sum_{i=1}^{m}(h_{\theta}(x^{(i)}) - y^{(i)})x_{j}^{(i)} & & \text{ for j = 0, 1}    
\end{matrix}
$$
And as you probably suspected, a generic equation opens the door to vectorization...

% =============================== %
\subsection*{Vectorizing the Gradient Calculation}
% ******************************* %
Now it's time to learn how to calculate the entire gradient in one short, pretty, linear algebra equation!  
\begin{itemize}
    \item First, we'll use the $X'$ matrix and our vectorized hypothesis equation $h_{\theta}(x)=X'\theta$
    $$
    \begin{matrix}
    \nabla(J)_j = \frac{1}{m} (X'\theta - y)X'_{j} & & \text{ for j = 0, 1}
    \end{matrix}
    $$
    
    \item Second, we need to tweak the equation a bit so that it directly returns a $\nabla(J)$ vector containing both $\nabla(J)_0$ and $\nabla(J)_1$.
    
    $$
    \nabla(J) = \frac{1}{m} {X'}^T(X'\theta - y)    
    $$
\end{itemize}
If the equation does not seems obvious, play a bit with your vectors, on paper and in your code, until you get it.\\

% =============================== %
\subsubsection*{Notation Remark}
% ******************************* %
${X'}^T$: You might wonder what the $^T$ is for.
It means the $X'$ matrix must be \textbf{transposed}.\\
\newline
Transposing a matrix flips it on its diagonal so that its rows become its columns and \textit{vice-versa}.
Here we need to make sure that matrix dimensions are appropriate and allow for multiplication, and to multiply the right items together.
\newpage
\turnindir{ex01}
\exnumber{01}
\exfiles{l2\_reg.py}
\exforbidden{sklearn}
\makeheaderfilesforbidden

% ================================= %
\section*{Objective}
% --------------------------------- %
You must implement the following formulas as functions:  

% ================================= %
\subsection*{Iterative}
% --------------------------------- %
$$
L_2(\theta)^2 = \sum_{j = 1}^n \theta_j^2
$$

Where:
\begin{itemize}
  \item $\theta$ is a vector of dimension $(n + 1)$.
\end{itemize}

% ================================= %
\subsection*{Vectorized}
% --------------------------------- %
$$
L_2(\theta)^2 = \theta' \cdot \theta'
$$

Where:
\begin{itemize}
  \item $\theta'$ is a vector of dimension $(n + 1)$, constructed using the following rules:
\end{itemize}
  
$$
\begin{matrix}
\theta'_0 & =  0 \\
\theta'_j & =  \theta_j & \text{ for } j = 1, \dots, n\\
\end{matrix}
$$

% ================================= %
\section*{Instructions}
% --------------------------------- %
In the \texttt{l2\_reg.py} file, write the following function as per the instructions given below:

\begin{minted}[bgcolor=darcula-back,formatcom=\color{lightgrey},fontsize=\scriptsize]{python}
def iterative_l2(theta):
	"""Computes the L2 regularization of a non-empty numpy.ndarray, with a for-loop.
	Args:
		theta: has to be a numpy.ndarray, a vector of shape n * 1.
	Returns:
		The L2 regularization as a float.
		None if theta in an empty numpy.ndarray.
	Raises:
		This function should not raise any Exception.
	"""
	... Your code ...

def l2(theta):
	"""Computes the L2 regularization of a non-empty numpy.ndarray, without any for-loop.
	Args:
		theta: has to be a numpy.ndarray, a vector of shape n * 1.
	Returns:
		The L2 regularization as a float.
		None if theta in an empty numpy.ndarray.
	Raises:
		This function should not raise any Exception.
	"""
	... Your code ...
\end{minted}

% ================================= %
\section*{Examples}
% --------------------------------- %

\begin{minted}[bgcolor=darcula-back,formatcom=\color{lightgrey},fontsize=\scriptsize]{python}
x = np.array([2, 14, -13, 5, 12, 4, -19]).reshape((-1, 1))

# Example 1: 
iterative_l2(x)
# Output:
911.0

# Example 2: 
l2(x)
# Output:
911.0

y = np.array([3,0.5,-6]).reshape((-1, 1))
# Example 3: 
iterative_l2(y)
# Output:
36.25

# Example 4: 
l2(y)
# Output:
36.25
\end{minted}

% ===========================(fin ex 01)         %
% ============================================== %

\newpage

% ============================================== %
% ===========================(start ex 02)       %
\chapter{Exercise 02}
\extitle{Regularized Linear Loss Function}

\turnindir{ex02}
\exnumber{02}
\exfiles{linear\_loss\_reg.py}
\exforbidden{sklearn}
\makeheaderfilesforbidden

% ================================= %
\section*{Objective}
% --------------------------------- %
You must implement the following formula as a function:  

$$
J(\theta)  =  \frac{1}{2m}[(\hat{y} - y)\cdot(\hat{y} - y) + \lambda (\theta' \cdot \theta')]
$$  

Where:
\begin{itemize}
  \item $y$ is a vector of dimension $m$, the expected values,
  \item $\hat{y}$ is a vector of dimension $m$, the predicted values,
  \item $\lambda$ is a constant, the regularization hyperparameter,
  \item $\theta'$ is a vector of dimension $n$, constructed using the following rules:
\end{itemize}
  
$$
\begin{matrix}
\theta'_0 & =  0 \\
\theta'_j & =  \theta_j & \text{ for } j = 1, \dots, n\\
\end{matrix}
$$

% ================================= %
\section*{Instructions}
% --------------------------------- %
In the \texttt{linear\_loss\_reg.py} file, write the following function as per the instructions given below:

\begin{minted}[bgcolor=darcula-back,formatcom=\color{lightgrey},fontsize=\scriptsize]{python}
def reg_loss_(y, y_hat, theta, lambda_):
	"""Computes the regularized loss of a linear regression model from two non-empty numpy.array, without any for loop. The two arrays must have the same dimensions.
	Args:
		y: has to be an numpy.ndarray, a vector of shape m * 1.
		y_hat: has to be an numpy.ndarray, a vector of shape m * 1.
		theta: has to be a numpy.ndarray, a vector of shape n * 1.
		lambda_: has to be a float.
	Returns:
		The regularized loss as a float.
		None if y, y_hat, or theta are empty numpy.ndarray.
		None if y and y_hat do not share the same shapes.
	Raises:
		This function should not raise any Exception.
	"""
	... Your code ...
\end{minted}

\hint{such situation is a good use case for decorators...}

% ================================= %
\section*{Examples}
% --------------------------------- %
\begin{minted}[bgcolor=darcula-back,formatcom=\color{lightgrey},fontsize=\scriptsize]{python}
y = np.array([2, 14, -13, 5, 12, 4, -19]).reshape((-1, 1))
y_hat = np.array([3, 13, -11.5, 5, 11, 5, -20]).reshape((-1, 1))
theta = np.array([1, 2.5, 1.5, -0.9]).reshape((-1, 1))

# Example :
reg_loss_(y, y_hat, theta, .5)
# Output:
0.8503571428571429

# Example :
reg_loss_(y, y_hat, theta, .05)
# Output:
0.5511071428571429

# Example :
reg_loss_(y, y_hat, theta, .9)
# Output:
1.116357142857143
\end{minted}


% ===========================(fin ex 02)         %
% ============================================== %

\newpage

% ============================================== %
% ===========================(start ex 03)       %
\chapter{Exercise 03}
\extitle{Regularized Logistic Loss Function}
%%******************************************************************************%
%                                                                              %
%                                 Interlude                                    %
%                         for Machine Learning module                          %
%                                                                              %
%******************************************************************************%

% =============================== %
\section*{Interlude - Improve with the Gradient}
% ******************************* %

\begin{figure}[!h]
    \centering
    \includegraphics[scale=0.2]{assets/Improve.png}
    %\caption{The Learning Cycle: Improve}
\end{figure}

% =============================== %
\section*{Multivariate Gradient}
% ******************************* %
From our multivariate linear hypothesis we can derive our multivariate gradient.
It looks a lot like the one we saw during the previous module, but instead of having just two components, the gradient now has as many as there are parameters.
This means that now we need to calculate $\nabla(J)_0,\nabla(J)_1,\dots,\nabla(J)_n$.\\
\newline
If we take the univariate equations we used during the previous module and replace the formula for $\nabla(J)_1$ by a more general $\nabla(J)_j$, we get the following:

$$
\begin{matrix}
\nabla(J)_0 &  = &\frac{1}{m}\sum_{i=1}^{m}(h_{\theta}(x^{(i)}) - y^{(i)}) & \\
\nabla(J)_j & = &\frac{1}{m}\sum_{i=1}^{m}(h_{\theta}(x^{(i)}) - y^{(i)})x_{j}^{(i)} & \text{ for j = 1, ..., n}    
\end{matrix}
$$
Where:
\begin{itemize}
    \item $\nabla(J)$ is a vector of dimension $(n + 1)$, the gradient vector
    \item $\nabla(J)_j$ is the j$^\text{th}$ component of $\nabla(J)$, the partial derivative of $J$ with respect to $\theta_j$
    \item $y$ is a vector of dimension $m$, the vector of expected values
    \item $y^{(i)}$ is a scalar, the i$^\text{th}$ component of vector $y$
    \item $x^{(i)}$ is the feature vector of the i$^\text{th}$ example
    \item $x^{(i)}_j$ is a scalar, the j$^\text{th}$ feature value of the i$^\text{th}$ example
    \item $h_{\theta}(x^{(i)})$ is a scalar, the model's estimation of $y^{(i)}$. (It can also be denoted $\hat{y}^{(i)}$)
\end{itemize}

% =============================== %
\section*{Vectorized Form}
% ******************************* %
As usual, we can use some linear algebra magic to get a more compact (and computationally efficient) formula.
First we can use our convention that each training example has an extra $x_0 = 1$ feature, and replace the gradient formulas above by one single equation that is valid for all $j$ components:

$$
\begin{matrix}
\nabla(J)_j & = &\frac{1}{m}\sum_{i=1}^{m}(h_{\theta}(x^{(i)}) - y^{(i)})x_{j}^{(i)} & \text{ for j = 0, ..., n}
\end{matrix}
$$
And this generic equation can then be rewritten in a vectorized form:

$$
\nabla(J) = \frac{1}{m} {X'}^T(X'\theta - y)
$$  
Where:  
\begin{itemize}
    \item $\nabla(J)$ is the gradient vector of dimension $(n + 1)$
    \item $X'$ is a matrix of dimensions $(m \times (n + 1))$, the design matrix onto which a column of $1$'s was added as the first column
    \item ${X'}^T$ means the matrix has been transposed
    \item $\theta$ is a vector of dimension $(n + 1)$: the parameter vector 
    \item $y$ is a vector of dimension $m$: the vector of expected values
\end{itemize}
The vectorized equation can output the entire gradient vector all at once, in one calculation!\\
\newline
So if you understand the linear algebra operations, you can forget about the equations we presented at the top of the page and simply use the vectorized one.

%\newpage
\turnindir{ex03}
\exnumber{03}
\exfiles{logistic\_loss\_reg.py}
\exforbidden{sklearn}
\makeheaderfilesforbidden

% ================================= %
\section*{Objective}
% --------------------------------- %
You must implement the following formula as a function:

$$
J( \theta) = -\frac{1} {m} \lbrack y \cdot \log(\hat{y}) + (\vec{1} - y) \cdot \log(\vec{1} - \hat{y})\rbrack + \frac{\lambda}{2m} (\theta' \cdot \theta')
$$

Where:
\begin{itemize}
  \item $\hat{y}$ is a vector of dimension $m$, the vector of predicted values,
  \item $y$ is a vector of dimension $m$, the vector of expected values,
  \item $\vec{1}$ is a vector of dimension $m$, a vector full of ones,
  \item $\lambda$ is a constant, the regularization hyperparameter,
  \item $\theta'$ is a vector of dimension $n$, constructed using the following rules: 
\end{itemize}
$$
\begin{matrix}
\theta'_0 & =  0 \\
\theta'_j & =  \theta_j & \text{ for } j = 1, \dots, n\\    
\end{matrix}
$$

% ================================= %
\section*{Instructions}
% --------------------------------- %
In the \texttt{logistic\_loss\_reg.py} file, write the following function as per the instructions given below:

\begin{minted}[bgcolor=darcula-back,formatcom=\color{lightgrey},fontsize=\scriptsize]{python}
def reg_log_loss_(y, y_hat, theta, lambda_):
	"""Computes the regularized loss of a logistic regression model from two non-empty numpy.ndarray, without any for loop. The two arrays must have the same shapes.
	Args:
		y: has to be an numpy.ndarray, a vector of shape m * 1.
		y_hat: has to be an numpy.ndarray, a vector of shape m * 1.
		theta: has to be a numpy.ndarray, a vector of shape n * 1.
		lambda_: has to be a float.
	Returns:
		The regularized loss as a float.
		None if y, y_hat, or theta is empty numpy.ndarray.
		None if y and y_hat do not share the same shapes.
	Raises:
		This function should not raise any Exception.
	"""
	... Your code ...
\end{minted}

\hint{
  this is a good use case for decorators...
}


% ================================= %
\section*{Examples}
% --------------------------------- %
\begin{minted}[bgcolor=darcula-back,formatcom=\color{lightgrey},fontsize=\scriptsize]{python}
y = np.array([1, 1, 0, 0, 1, 1, 0]).reshape((-1, 1))
y_hat = np.array([.9, .79, .12, .04, .89, .93, .01]).reshape((-1, 1))
theta = np.array([1, 2.5, 1.5, -0.9]).reshape((-1, 1))

# Example :
reg_log_loss_(y, y_hat, theta, .5)
# Output:
0.43377043716475955

# Example :
reg_log_loss_(y, y_hat, theta, .05)
# Output:
0.13452043716475953

# Example :
reg_log_loss_(y, y_hat, theta, .9)
# Output:
0.6997704371647596
\end{minted}

% ===========================(fin ex 03)         %
% ============================================== %

\newpage

% ============================================== %
% ===========================(start ex 04)       %
\chapter{Exercise 04}
\extitle{Regularized Linear Gradient}
%******************************************************************************%
%                                                                              %
%                                 Interlude                                    %
%                         for Machine Learning module                          %
%                                                                              %
%******************************************************************************%

% =============================================== %
\section*{Interlude}
% =============================================== %
\subsection*{Improve}
% ----------------------------------------------- %

\begin{figure}[!h]
    \centering
    \includegraphics[scale=0.25]{assets/Improve.png}
    %\caption{The Learning Cycle: Improve}
\end{figure}
\noindent{Now we want to improve the algorithm's 
performance, or in other words, reduce the loss of its predictions.}\\
\\
This brings us (again) to calculating the gradient, which will tell us by
how much and in which direction the theta parameters belonging to the model should be adjusted.

\newpage
% =============================================== %
\subsection*{The logistic gradient}
% ----------------------------------------------- %
If you remember, to calculate the gradient, we start with the loss function and we derive it 
with respect to each of the theta parameters.\\
\\
If you know multivariate calculus already, you can try it for yourself, otherwise we've got you covered:\\

$$
\begin{matrix}
\nabla(J)_0 &  = &\cfrac{1}{m}\sum_{i=1}^{m}(h_{\theta}(x^{(i)}) - y^{(i)}) & \\
\nabla(J)_j & = &\cfrac{1}{m}\sum_{i=1}^{m}(h_{\theta}(x^{(i)}) - y^{(i)})x_{j}^{(i)} & \text{ for j = 1, ..., n}    
\end{matrix}
$$
Where:
\begin{itemize}
    \item $\nabla(J)$ is a vector of dimension $(n + 1)$, the gradient vector
    \item $\nabla(J)_j$ is the j$^\text{th}$ component of $\nabla(J)$, 
    the partial derivative of $J$ with respect to $\theta_j$
    \item $y$ is a vector of dimension $m$, the vector of expected values
    \item $y^{(i)}$ is a scalar, the i$^\text{th}$ component of vector $y$
    \item $x^{(i)}$ is the feature vector of the i$^\text{th}$ example
    \item $x^{(i)}_j$ is a scalar, the j$^\text{th}$ feature value of the i$^\text{th}$ example
    \item $h_{\theta}(x^{(i)})$ is a scalar, the model's estimation of $y^{(i)}$\\
\end{itemize}
This formula should be very familiar to you, as it's the same one you used to calculate the linear regression gradient!\\
\\
The only difference is that $h_{\theta}(x^{(i)})$ corresponds to \textbf{the logistic regression hypothesis instead of the linear regression hypothesis}.\\
\\
In other words:\\
$$
h_{\theta}(x^{(i)}) = \text{sigmoid}( \theta \cdot x'^{(i)}) = \cfrac{1} {1 + e^{-\theta \cdot x'^{(i)}}}
$$
\\
Instead of:
\\
$$
\cancel{h_{\theta}(x^{(i)}) = \theta \cdot x'^{(i)}}
$$

\newpage
\turnindir{ex04}
\exnumber{04}
\exfiles{reg\_linear\_grad.py}
\exforbidden{sklearn}
\makeheaderfilesforbidden


% ================================= %
\section*{Objective}
% --------------------------------- %
You must implement the following formulas as a functions for the \textbf{linear regression hypothesis}:

% ================================= %
\subsection*{Iterative}
% --------------------------------- %
$$
\nabla(J)_0 = \frac{1}{m}\sum_{i=1}^{m}(h_\theta(x^{(i)}) - y^{(i)})
$$
$$
\nabla(J)_j = \frac{1}{m}\left(\sum_{i=1}^{m}(h_\theta(x^{(i)}) - y^{(i)})x_j^{(i)} + \lambda \theta_j\right) \text{ for j = 1, ..., n}
$$

Where:
\begin{itemize}
  \item $\nabla(J)_j$ is the j$^\text{th}$ component of $\nabla(J)$,
  \item $\nabla(J)$ is a vector of dimension $(n + 1)$, the gradient vector,
  \item $m$ is a constant, the number of training examples used,
  \item $h_\theta(x^{(i)})$ is the model's prediction for the i$^\text{th}$ training example,
  \item $x^{(i)}$ is the feature vector (of dimension $n$) of the i$^\text{th}$ training example, found in the i$^\text{th}$ row of the $X$ matrix,
  \item $X$ is a matrix of dimensions $(m \times n)$, the design matrix,
  \item $y^{(i)}$ is the i$^\text{th}$ component of the $y$ vector,
  \item $y$ is a vector of dimension $m$, the vector of expected values,
  \item $\lambda$ is a constant, the regularization hyperparameter,
  \item $\theta_j$ is the j$^\text{th}$ parameter of the $\theta$ vector,
  \item $\theta$ is a vector of dimension $(n + 1)$, the parameter vector.
\end{itemize}

% ================================= %
\subsection*{Vectorized}
% --------------------------------- %
$$
\nabla(J) = \frac{1}{m} [X'^T(h_\theta(X) - y) + \lambda \theta']
$$  

Where:
\begin{itemize}
  \item $\nabla(J)$ is a vector of dimension $(n + 1)$, the gradient vector,
  \item $m$ is a constant, the number of training examples used,
  \item $X$ is a matrix of dimensions $(m \times n)$, the design matrix,
  \item $X'$ is a matrix of dimensions $(m \times (n + 1))$, the design matrix onto which a column of ones is added as a first column,
  \item $X'^T$ is the transpose of tha matrix, with dimensions $((n + 1) \times m)$,
  \item $h_\theta(X)$ is a vector of dimension $m$, the vector of predicted values, 
  \item $y$ is a vector of dimension $m$, the vector of expected values,
  \item $\lambda$ is a constant, the regularization hyperparameter,
  \item $\theta$ is a vector of dimension $(n + 1)$, the parameter vector,
  \item $\theta'$ is a vector of dimension $(n + 1)$, constructed using the following rules: 
\end{itemize}

$$
\begin{matrix}
\theta'_0 & =  0 \\
\theta'_j & =  \theta_j & \text{ for } j = 1, \dots, n\\
\end{matrix}
$$

% ================================= %
\section*{Instructions}
% --------------------------------- %
In the \texttt{reg\_linear\_grad.py} file, write the following functions as per the instructions given below:

\begin{minted}[bgcolor=darcula-back,formatcom=\color{lightgrey},fontsize=\scriptsize]{python}
def reg_linear_grad(y, x, theta, lambda_):
    """Computes the regularized linear gradient of three non-empty numpy.ndarray,
       with two for-loop. The three arrays must have compatible shapes.
    Args:
      y: has to be a numpy.ndarray, a vector of shape m * 1.
      x: has to be a numpy.ndarray, a matrix of dimesion m * n.
      theta: has to be a numpy.ndarray, a vector of shape (n + 1) * 1.
      lambda_: has to be a float.
    Return:
      A numpy.ndarray, a vector of shape (n + 1) * 1, containing the results of the formula for all j.
      None if y, x, or theta are empty numpy.ndarray.
      None if y, x or theta does not share compatibles shapes.
      None if y, x or theta or lambda_ is not of the expected type.
    Raises:
      This function should not raise any Exception.
    """
    ... Your code ...

def vec_reg_linear_grad(y, x, theta, lambda_):
    """Computes the regularized linear gradient of three non-empty numpy.ndarray,
       without any for-loop. The three arrays must have compatible shapes.
    Args:
      y: has to be a numpy.ndarray, a vector of shape m * 1.
      x: has to be a numpy.ndarray, a matrix of dimesion m * n.
      theta: has to be a numpy.ndarray, a vector of shape (n + 1) * 1.
      lambda_: has to be a float.
    Return:
      A numpy.ndarray, a vector of shape (n + 1) * 1, containing the results of the formula for all j.
      None if y, x, or theta are empty numpy.ndarray.
      None if y, x or theta does not share compatibles shapes.
      None if y, x or theta or lambda_ is not of the expected type.
    Raises:
      This function should not raise any Exception.
    """
    ... Your code ...
\end{minted}

\hint{
  this is a good use case for decorators...
}

% ================================= %
\section*{Examples}
% ================================= %
\begin{minted}[bgcolor=darcula-back,formatcom=\color{lightgrey},fontsize=\scriptsize]{python}
x = np.array([
		[ -6,  -7,  -9],
		[ 13,  -2,  14],
		[ -7,  14,  -1],
		[ -8,  -4,   6],
		[ -5,  -9,   6],
		[  1,  -5,  11],
		[  9, -11,   8]])
y = np.array([[2], [14], [-13], [5], [12], [4], [-19]])
theta = np.array([[7.01], [3], [10.5], [-6]])

# Example 1.1:
reg_linear_grad(y, x, theta, 1)
# Output:
array([[ -60.99      ],
		[-195.64714286],
		[ 863.46571429],
		[-644.52142857]])

# Example 1.2:
vec_reg_linear_grad(y, x, theta, 1)
# Output:
array([[ -60.99      ],
		[-195.64714286],
		[ 863.46571429],
		[-644.52142857]])

# Example 2.1:
reg_linear_grad(y, x, theta, 0.5)
# Output:
array([[ -60.99      ],
		[-195.86142857],
		[ 862.71571429],
		[-644.09285714]])

# Example 2.2:
vec_reg_linear_grad(y, x, theta, 0.5)
# Output:
array([[ -60.99      ],
		[-195.86142857],
		[ 862.71571429],
		[-644.09285714]])

# Example 3.1:
reg_linear_grad(y, x, theta, 0.0)
# Output:
array([[ -60.99      ],
		[-196.07571429],
		[ 861.96571429],
		[-643.66428571]])

# Example 3.2:
vec_reg_linear_grad(y, x, theta, 0.0)
# Output:
array([[ -60.99      ],
		[-196.07571429],
		[ 861.96571429],
		[-643.66428571]])
\end{minted}


% ===========================(fin ex 04)         %
% ============================================== %

\newpage

% ============================================== %
% ===========================(start ex 05)       %
\chapter{Exercise 05}
\extitle{Regularized Logistic Gradient}
%%******************************************************************************%
%                                                                              %
%                                 Interlude                                    %
%                         for Machine Learning module                          %
%                                                                              %
%******************************************************************************%

% ============================================== %
\section*{Interlude}
% ============================================== %
\subsection*{Vectorized Logistic Gradient}
% ---------------------------------------------- %

Given the previous logistic gradient formula, it's quite easy to produce a vectorized version of it.
Actually, you almost already implemented it on module02!\\
\\
As with the previous exercise, \textbf{the only thing you have to change is your hypothesis} 
in order to calculate your logistic gradient.\\

$$
\begin{matrix}
\nabla(J)_0 &  = &\cfrac{1}{m}\sum_{i=1}^{m}(h_{\theta}(x^{(i)}) - y^{(i)}) & \\
\nabla(J)_j & = &\cfrac{1}{m}\sum_{i=1}^{m}(h_{\theta}(x^{(i)}) - y^{(i)})x_{j}^{(i)} & \text{ for j = 1, ..., n}    
\end{matrix}
$$

% ============================================== %
\subsection*{Vectorized Version}
% ---------------------------------------------- %

Can be vectorized the same way you did before:

$$
\nabla(J) = \cfrac{1}{m} X'^T(h_\theta(X) - y)
$$  

%\newpage
\turnindir{ex05}
\exnumber{05}
\exfiles{reg\_logistic\_grad.py}
\exforbidden{sklearn}
\makeheaderfilesforbidden

% ================================= %
\section*{Objective}
% --------------------------------- %
You must implement the following formulas as a functions for the \textbf{logistic regression hypothesis}:

% ================================= %
\subsection*{Iterative}
% --------------------------------- %

$$
\nabla(J)_0 = \frac{1}{m}\sum_{i=1}^{m}(h_\theta(x^{(i)}) - y^{(i)})
$$
$$
\nabla(J)_j = \frac{1}{m}\left(\sum_{i=1}^{m}(h_\theta(x^{(i)}) - y^{(i)})x_j^{(i)} + \lambda \theta_j\right) \text{ for j = 1, ..., n}
$$

Where:
\begin{itemize}
  \item $\nabla(J)_j$ is the j$^\text{th}$ component of $\nabla(J)$,
  \item $\nabla(J)$ is a vector of dimension $(n + 1)$, the gradient vector,
  \item $m$ is a constant, the number of training examples used,
  \item $h_\theta(x^{(i)})$ is the model's prediction for the i$^\text{th}$ training example,
  \item $x^{(i)}$ is the feature vector of dimension $n$) of the i$^\text{th}$ training example, found in the i$^\text{th}$ row of the $X$ matrix,
  \item $X$ is a matrix of dimensions $(m \times n)$, the design matrix,
  \item $y^{(i)}$ is the i$^\text{th}$ component of the $y$ vector,
  \item $y$ is a vector of dimension $m$, the vector of expected values,
  \item $\lambda$ is a constant, the regularization hyperparameter,
  \item $\theta_j$ is the j$^\text{th}$ parameter of the $\theta$ vector,
  \item $\theta$ is a vector of dimension $(n + 1)$, the parameter vector.
\end{itemize}

% ================================= %
\subsection*{Vectorized}
% --------------------------------- %
$$
\nabla(J) = \frac{1}{m} [X'^T(h_\theta(X) - y) + \lambda \theta']
$$  

Where:
\begin{itemize}
  \item $\nabla(J)$ is a vector of dimension $(n + 1)$, the gradient vector,
  \item $m$ is a constant, the number of training examples used,
  \item $X$ is a matrix of dimensions $(m \times n)$, the design matrix,
  \item $X'$ is a matrix of dimensions $(m \times (n + 1))$, the design matrix onto which a column of ones is added as a first column,
  \item $X'^T$ is the transpose of tha matrix, with dimensions $((n + 1) \times m)$,
  \item $h_\theta(X)$ is a vector of dimension $m$, the vector of predicted values, 
  \item $y$ is a vector of dimension $m$, the vector of expected values,
  \item $\lambda$ is a constant, the regularization hyperparameter,
  \item $\theta$ is a vector of dimension $(n + 1)$, the parameter vector,
  \item $\theta'$ is a vector of dimension $(n + 1)$, constructed using the following rules: 
\end{itemize}

$$
\begin{matrix}
\theta'_0 & =  0 \\
\theta'_j & =  \theta_j & \text{ for } j = 1, \dots, n\\
\end{matrix}
$$

% ================================= %
\section*{Instructions}
% --------------------------------- %
In the \texttt{reg\_logistic\_grad.py} file, create the following function as per the instructions given below:

\begin{minted}[bgcolor=darcula-back,formatcom=\color{lightgrey},fontsize=\scriptsize]{python}
def reg_logistic_grad(y, x, theta, lambda_):
	"""Computes the regularized logistic gradient of three non-empty numpy.ndarray, with two for-loops. The three arrays must have compatible shapes.
	Args:
		y: has to be a numpy.ndarray, a vector of shape m * 1.
		x: has to be a numpy.ndarray, a matrix of dimesion m * n.
		theta: has to be a numpy.ndarray, a vector of shape n * 1.
		lambda_: has to be a float.
	Returns:
		A numpy.ndarray, a vector of shape n * 1, containing the results of the formula for all j.
		None if y, x, or theta are empty numpy.ndarray.
		None if y, x or theta does not share compatibles shapes.
	Raises:
		This function should not raise any Exception.
	"""
	... Your code ...

def vec_reg_logistic_grad(y, x, theta, lambda_):
	"""Computes the regularized logistic gradient of three non-empty numpy.ndarray, without any for-loop. The three arrays must have compatible shapes.
	Args:
		y: has to be a numpy.ndarray, a vector of shape m * 1.
		x: has to be a numpy.ndarray, a matrix of shape m * n.
		theta: has to be a numpy.ndarray, a vector of shape n * 1.
		lambda_: has to be a float.
	Returns:
		A numpy.ndarray, a vector of shape n * 1, containing the results of the formula for all j.
		None if y, x, or theta are empty numpy.ndarray.
		None if y, x or theta does not share compatibles shapes.
	Raises:
		This function should not raise any Exception.
	"""
	... Your code ...
\end{minted}

\hint{
  this is a good use case for decorators...
}

% ================================= %
\section*{Examples}
% --------------------------------- %
\begin{minted}[bgcolor=darcula-back,formatcom=\color{lightgrey},fontsize=\scriptsize]{python}
x = np.array([[0, 2, 3, 4], 
				[2, 4, 5, 5], 
				[1, 3, 2, 7]])
y = np.array([[0], [1], [1]])
theta = np.array([[-2.4], [-1.5], [0.3], [-1.4], [0.7]])

# Example 1.1:
reg_logistic_grad(y, x, theta, 1)
# Output:
array([[-0.55711039],
		[-1.40334809],
		[-1.91756886],
		[-2.56737958],
		[-3.03924017]])

# Example 1.2:
vec_reg_logistic_grad(y, x, theta, 1)
# Output:
array([[-0.55711039],
		[-1.40334809],
		[-1.91756886],
		[-2.56737958],
		[-3.03924017]])

# Example 2.1:
reg_logistic_grad(y, x, theta, 0.5)
# Output:
array([[-0.55711039],
		[-1.15334809],
		[-1.96756886],
		[-2.33404624],
		[-3.15590684]])

# Example 2.2:
vec_reg_logistic_grad(y, x, theta, 0.5)
# Output:
array([[-0.55711039],
		[-1.15334809],
		[-1.96756886],
		[-2.33404624],
		[-3.15590684]])

# Example 3.1:
reg_logistic_grad(y, x, theta, 0.0)
# Output:
array([[-0.55711039],
		[-0.90334809],
		[-2.01756886],
		[-2.10071291],
		[-3.27257351]])

# Example 3.2:
vec_reg_logistic_grad(y, x, theta, 0.0)
# Output:
array([[-0.55711039],
		[-0.90334809],
		[-2.01756886],
		[-2.10071291],
		[-3.27257351]])
\end{minted}

% ===========================(fin ex 05)         %
% ============================================== %

\newpage

% ============================================== %
% ===========================(start ex 06)       %
\chapter{Exercise 06}
\extitle{Ridge Regression}
%******************************************************************************%
%                                                                              %
%                                 Interlude                                    %
%                         for Machine Learning module                          %
%                                                                              %
%******************************************************************************%

% ============================================== %
\section*{Interlude}
% ============================================== %
\subsection*{Linear Regression to the Next Level: Ridge Regression}
% ---------------------------------------------- %

Until now we only talked about L$_2$ regularization and its implication on the calculation of the loss function and gradient for both linear and logistic regression.

Now it's time to use proper terminology:
When we apply L$_2$ regularization on a linear regression model, the new model is called a \textbf{Ridge Regression} model.
Besides that brand-new name, Ridge regression is nothing more than linear regression regularized with L$_2$.

We suggest you watch this nice explanation \href{https://www.youtube.com/watch?v=Q81RR3yKn30}{very nice explanation of Ridge Regularization}.
By the way, this Youtube channel, \texttt{\textit{StatQuest}}, is very good to help you understand the gist of a lot of machine learning concepts.
You will not waste your time watching its statistics and machine learning playlists!

\newpage
\turnindir{ex06}
\exnumber{06}
\exfiles{ridge.py}
\exforbidden{sklearn}
\makeheaderfilesforbidden

% ================================= %
\section*{Objective}
% --------------------------------- %
Now it's time to implement your \texttt{MyRidge} class, similar to the class of the same name in \texttt{sklearn.linear\_model}.

% ================================= %
\section*{Instructions}
% --------------------------------- %
In the \texttt{ridge.py} file, create the following class as per the instructions given below:

Your \texttt{MyRidge} class will have at least the following methods:
\begin{itemize}
  \item \texttt{\_\_init\_\_}, special method, similar to the one you wrote in \texttt{MyLinearRegression} (module06),
  \item \texttt{get\_params\_}, which get the parameters of the estimator, 
  \item \texttt{set\_params\_}, which set the parameters of the estimator,
  \item \texttt{loss\_}, which return the loss between 2 vectors (numpy arrays),
  \item \texttt{loss\_elem\_}, which return a vector corresponding to the squared diffrence between 2 vectors (numpy arrays),  
  \item \texttt{predict\_}, which generates predictions using a linear model,
  \item \texttt{gradient\_}, which calculates the vectorized regularized gradient,
  \item \texttt{fit\_}, which fits Ridge regression model to a training dataset.
\end{itemize}

\hint{
  You should consider inheritance
}

The difference between \texttt{MyRidge}'s \texttt{loss\_elem\_}, \texttt{loss\_}, \texttt{gradient\_} and \texttt{fit\_} methods and \texttt{MyLinearRegression}'s \texttt{loss\_elem\_}, \texttt{loss\_}, \texttt{gradient\_} and \texttt{fit\_} methods implemented in module 06 is the use of a regularization term.

\begin{minted}[bgcolor=darcula-back,formatcom=\color{lightgrey},fontsize=\scriptsize]{python}
class MyRidge(ParentClass):
	"""
	Description:
		My personnal ridge regression class to fit like a boss.
	"""
	def __init__(self,  thetas, alpha=0.001, max_iter=1000, lambda_=0.5):
		self.alpha = alpha
		self.max_iter = max_iter
		self.thetas = thetas
		self.lambda_ = lambda_
		... Your code here ...

	... other methods ...
\end{minted}

\hint{
  again, this is a good use case for decorators...
}

% ===========================(fin ex 06)         %
% ============================================== %

\newpage

% ============================================== %
% ===========================(start ex 07)       %
\chapter{Exercise 07}
\extitle{Practicing Ridge Regression}
%%******************************************************************************%
%                                                                              %
%                                 Interlude                                    %
%                         for Machine Learning module                          %
%                                                                              %
%******************************************************************************%

% =============================================== %
\section*{Interlude - Introducing Polynomial Models}
% ----------------------------------------------- %

You probably noticed that the method we use is called \textit{linear regression} for a reason:
the model generates all of its predictions on a straight line.
However, we often encounter features that do not have a linear relationship with the predicted variable,
like in the figure below:

\begin{figure}[!h]
    \centering
    \includegraphics[scale=0.6]{assets/polynomial_straight_line.png}
    \caption{Non-linear relationship}
\end{figure}
In that case, we are stuck with a straight line that cannot fit the data points properly.\\
\newline
In this example, what if we could express $y$ not as a function of $x$, but also of $x^2$, and maybe even $x^3$ and $x^4$?
We could make a hypothesis that draws a nice \textbf{curve} that would better fit the data.
That's where polynomial features can help!

% =============================================== %
\section*{Interlude - Polynomial features}
% ----------------------------------------------- %
First we get to do some \textit{feature engineering}.
We create new features by raising our initial $x$ feature to the power of 2, and then 3, 4... as far as we want to go.
For each new feature we need to create a new column in the dataset.

% =============================================== %
\section*{Interlude - Polynomial Hypothesis}
% ----------------------------------------------- %
Now that we created our new features, we can combine them in a linear hypothesis that looks just the same as what we're used to:

$$
\hat{y} = \theta_0 + \theta_1 x  +\theta_2 x^{2} + \dots + \theta_n x^{n}
$$  
It's a little strange because we are building a linear combination, not with different features but with different powers of the same feature.
This is a first way of introducing non-linearity in a regression model!
%\newpage
\turnindir{ex07}
\exnumber{07}
\exfiles{space\_avocado.py, benchmark\_train.py,  models.[csv/yml/pickle]}
\exforbidden{sklearn}
\makeheaderfilesforbidden


% ================================= %
\section*{Objective}
% --------------------------------- %
It's training time!  
Let's practice our brand new Ridge Regression with a polynomial model.

% ================================= %
\section*{Introduction}
% --------------------------------- %
You have already used the dataset \texttt{space\_avocado.csv}.
The dataset is constituted of 5 columns:
\begin{itemize}
  \item \textbf{index}: not relevant,
  \item \textbf{weight}: the avocado weight order (in ton),
  \item \textbf{prod\_distance}: distance from where the avocado ordered is produced (in Mkm),
  \item \textbf{time\_delivery}: time between the order and the receipt (in days),
  \item \textbf{target}: price of the order (in trantorian unit).
\end{itemize}
It contains the data of all the avocado purchase made by Trantor administration (guacamole is a serious business there).

% ================================= %
\section*{Instructions}
% --------------------------------- %
You have to explore different models and select the best you find.
To do this:
\begin{itemize}
  \item Split your \texttt{space\_avocado.csv} dataset into a training, a cross-validation and a test sets.
  \item Use your \texttt{polynomial\_features} method on your training set.
  \item Consider several Linear Regression models with polynomial hypotheses with a maximum degree of $4$.
  \item For each hypothesis consider a regularized factor ranging from $0$ to $1$ with a step of $0.2$.
  \item Evaluate your models on the cross-validation set.
  \item Evaluate the best model on the test set.
\end{itemize}

According to your model evaluations, what is the best hypothesis you can get?
\begin{itemize}
  \item Plot the evaluation curve which help you to select the best model (evaluation metrics vs models + $\lambda$ factor).
  \item Plot the true price and the predicted price obtain via your best model with the different $\lambda$ values (meaning the dataset + the 5 predicted curves).
\end{itemize}


The training of all your models can take a long time.
Thus you need to train only the best one during the correction.
But, you should return in \texttt{benchmark\_train.py} the program which perform the training of all the models and save the parameters of the different models into a file.
In \texttt{models.[csv/yml/pickle]} one must find the parameters of all the models you have explored and trained.
In \texttt{space\_avocado.py} train the model based on the best hypothesis you find and load the other models from \texttt{models.[csv/yml/pickle]}.
Then evaluate the best model on the right set and plot the different graphics as asked before.

% ===========================(fin ex 07)         %
% ============================================== %

\newpage

% ============================================== %
% ===========================(start ex 08)       %
\chapter{Exercise 08}
\extitle{Regularized Logistic Regression}
%******************************************************************************%
%                                                                              %
%                                 Interlude                                    %
%                         for Machine Learning module                          %
%                                                                              %
%******************************************************************************%

% ============================================== %
\section*{Interlude}
% ============================================== %
\subsection*{More Evaluation Metrics!}
% ---------------------------------------------- %
Once your classifier is trained, evaluating its performances is key.\\
\\
You already know about \textit{cross-entropy}, as you have implemented it as your \textit{loss function}.
But when it comes to classification, there are more informative metrics we can use besides the loss function.
Each metric focuses on different error types.  
But what is an error type?\\
\\
A single classification prediction is either right or wrong, nothing in between.
Either an object is assigned to the right class, or to the wrong class.\\
When calculating performance scores for a multiclass classifier, we like to compute a 
separate score for each class that your classifier learned to discriminate (in a one-vs-all manner).\\
\\
In other words, for a given \textit{Class A}, we want a score that can answer the question: "how good is 
the model at assigning \textit{A} objects to \textit{Class A}, and at NOT assigning 
\textit{non-A} objects to \textit{Class A}?" \\
\\
You may not realize it yet, but this question involves measuring two 
very different error types, and this distinction is crucial.\\
\newpage
% ============================================== %
\subsection*{Error Types}
% ---------------------------------------------- %
With respect to a given \textit{Class A}, classification errors fall in two categories:
\begin{itemize}
    \item \textbf{False positive:} when a \textit{non-A} object is assigned to \textit{Class A}.\\
      For example:
      \begin{itemize}
          \item Pulling the fire alarm when there is no fire.
          \item Considering that someone is sick when she isn't.
          \item Identifying a face in an image when in fact it was a Teddy Bear.
      \end{itemize}
    
    \item \textbf{False negative:} when an \textit{A} object is assigned to another class than \textit{Class A}.\\
      For example:
      \begin{itemize}
          \item Not pulling the fire alarm when there is a fire.
          \item Considering that someone is not sick when she is.
          \item Failing to recognize a face in an image that does contain one.
      \end{itemize}
\end{itemize}
\bigskip
\noindent{It turns out that it's really hard to minimize both error types at the same time.}\\
\\
At some point you'll need to decide which one is the most critical, depending on your use case.\\
\\
For example, if you want to detect cancer, of course it's not good if your model 
erroneously diagnoses cancer on a few healthy patients (\textbf{false positives}), 
but you absolutely want to avoid failing at diagnosing cancer on affected patients (\textbf{false negatives}) 
and let them go on with their lives while developing a potentially dangerous cancer.\\

% ============================================== %
\subsection*{Metrics}
% ---------------------------------------------- %
A metric is computed on a set of predictions along with the corresponding set of actual categories.
The metric you choose will focus more or less on those two error types.
If we come back to the \textbf{Class A} classifier:
\begin{itemize}
    \item \textbf{Accuracy}: tells you the percentage of predictions that are accurate (i.e. the correct class was predicted).
          Accuracy doesn't give information about either error type.
    \item \textbf{Precision}: tells you how much you can trust your model when it says that an object belongs to \textit{Class A}.
          More precisely, it is the percentage of the objects assigned to \textit{Class A} that really were \textit{A} objects.
          You use precision when you want to control for \textbf{False positives}.
    \item \textbf{Recall}: tells you how much you can trust that your model is able to recognize ALL \textit{Class A} objects.
          It is the percentage of all \textbf{A} objects that were properly classified by the model as \textit{Class A}.
          You use recall when you want to control for \textbf{False negatives}.
    \item \textbf{F1 score}: combines precision and recall in one single measure.
          You use the F1 score when you want to control both \textbf{False positives} and \textbf{False negatives}.
\end{itemize}

\newpage
\turnindir{ex08}
\exnumber{08}
\exfiles{my\_logistic\_regression.py}
\exforbidden{sklearn}
\makeheaderfilesforbidden

% ================================= %
\section*{Objective}
% --------------------------------- %
In the last exercise, you implemented of a regularized version of the linear regression algorithm, called Ridge regression.
Now it's time to update your logistic regression classifier as well!
In the \texttt{scikit-learn} library, the logistic regression implementation offers a few regularization techniques, which can be selected using the parameter \texttt{penalty} (L$_2$ is default).
The goal of this exercise is to update your old \texttt{MyLogisticRegression} class to take that into account.

% ================================= %
\section*{Instructions}
% --------------------------------- %
In the \texttt{my\_logistic\_regression.py} file, update your \texttt{MyLogisticRegression} class according to the following:

\begin{itemize}
  \item \textbf{add} a \texttt{penalty} parameter which can take the following values:\texttt{'l2'}, \texttt{'none'} (default value is \texttt{'l2'}).
\end{itemize}
\begin{minted}[bgcolor=darcula-back,formatcom=\color{lightgrey},fontsize=\scriptsize]{python}
class MyLogisticRegression():
	"""
	Description:
		My personnal logistic regression to classify things.
	"""
	def __init__(self, theta, alpha=0.001, max_iter=1000, penalty='l2'):
		self.alpha = alpha
		self.max_iter = max_iter
		self.theta = theta
		self.penalty=penalty
		... Your code ...

	... other methods ...
\end{minted}

\begin{itemize}
  \item \textbf{update} the \texttt{fit\_(self, x, y)} method: 
  \begin{itemize}
    \item \texttt{if penalty == 'l2'}: use a \textbf{regularized version} of the gradient descent.
    \item \texttt{if penalty = 'none'}: use the \textbf{unregularized version} of the gradient descent from \texttt{module03}.
  \end{itemize}
\end{itemize}

\hint{
  this is also a great use case for decorators...
}

% ===========================(fin ex 08)         %
% ============================================== %

\newpage

% ============================================== %
% ===========================(start ex 09)       %
\chapter{Exercise 09}
\extitle{Practicing Regularized Logistic Regression}
%%******************************************************************************%
%                                                                              %
%                                 Interlude                                    %
%                         for Machine Learning module                          %
%                                                                              %
%******************************************************************************%

% ============================================== %
\section*{Interlude - Lost in Overfitting}
% ---------------------------------------------- %

The two previous exercises lead you, dear reader, to a very dangerous territory: the realm of \textbf{overfitting}.\\
You did not see it coming but now, you are in a bad situation...\\
\\
By increasing the polynomial degree of your model, you increased its \textbf{complexity}.  
Is it wrong?
Not always.
Some models are indeed very complex because the relationships they represent are very complex as well.\\
\\
But, if you look at the plots for the previous exercise's \textit{best model}, you should feel that something is wrong...\\
\\
% ============================================== %
\section*{Interlude - Something is rotten in the state of our model...}
% ---------------------------------------------- %
Take a look at the following plot. 

\begin{figure}[!h]
    \centering
    \includegraphics[scale=0.6]{assets/overfitt.png}
    \caption{Overfitting hypothesis}
\end{figure}

You can see that the prediction line fits each data point perfectly, but completely misses out on capturing the relationship between $x$ and $y$ properly.
And now, if we add some brand new data points to the dataset, we see that the predictions on those new examples are way off.

\begin{figure}[!h]
    \centering
    \includegraphics[scale=0.6]{assets/overfitt_with_dots.png}
    \caption{Generalization errors resulting from overfitting}
\end{figure}
This situation is called overfitting, because the model is doing an excessively good job at fitting the data.
It is literally bending over backward to account for the data's mini details.
But most of the data's irregularities are just noise, and they should in fact be ignored.
So because the model overfits, it can't generalize to new data.

% ============================================== %
\section*{Interlude - The training set, the test set, and the happy data scientist}
% ---------------------------------------------- %
To be able to detect overfitting, \textbf{you should always evaluate your model on new data}.\\
\\
New data means, data that your model hasn't seen during training.\\
\\
It's the only way to make sure your model isn't \textit{recalling}.
To do so, now and forever, you must always divide your dataset in (at least) two parts: one for the training, and one for the evaluation of your model.
%\newpage
\turnindir{ex09}
\exnumber{09}
\exfiles{solar\_system\_census.py, benchmark\_train.py,  models.[csv/yml/pickle]}
\exforbidden{sklearn}
\makeheaderfilesforbidden

% ================================= %
\section*{Objective}
% --------------------------------- %
It's training time!
Let's practice our updated Logistic Regression with polynomial models.
% ================================= %
\section*{Introduction}
% --------------------------------- %
You have already used the dataset \texttt{solar\_system\_census.csv} and \texttt{solar\_system\_census\_planets.csv}.
\begin{itemize}
	\item The dataset is divided in two files which can be found in the \texttt{resources} folder: \texttt{solar\_system\_census.csv} and \texttt{solar\_system\_census\_planets.csv}.
	\item The first file contains biometric information such as the height, weight, and bone density of several Solar System citizens.
	\item The second file contains the homeland of each citizen, indicated by its Space Zipcode representation (i.e. one number for each planet... :)).  
\end{itemize}

As you should know, Solar citizens come from four registered areas (zipcodes): 

\begin{itemize}
	\item The flying cities of Venus ($0$), 
	\item United Nations of Earth ($1$), 
	\item Mars Republic ($2$), 
	\item The Asteroids' Belt colonies ($3$).
\end{itemize}

% ================================= %
\section*{Instructions}
% --------------------------------- %
% ================================= %
\subsection*{Split the Data}
% --------------------------------- %

Take your \texttt{solar\_system\_census.csv} dataset and split it in a \textbf{training set}, a \textbf{cross-validation set}
and  a \textbf{test set}.

% ================================= %
\subsection*{Training and benchmark}
% --------------------------------- %
One part of your submission will be find in \texttt{benchmark\_train.py} and \texttt{models.[csv/yml/pickle]} files.
You have to:
\begin{itemize}
  \item Train different regularized logistic regression models with a polynomial hypothesis of \textbf{degree 3}.
        The models will be trained with different $\lambda$ values, ranging from $0$ to $1$.
        Use one-vs-all method.
  \item Evaluate the \textbf{f1 score} of each of the models on the cross-validation set.
        You can use the \texttt{f1\_score\_} function that you wrote in the \texttt{ex11} of \texttt{module08}.
  \item Save the different models into a \texttt{models.[csv/yml/pickle]}.
\end{itemize}

% ================================= %
\subsection*{Solar system census program}
% --------------------------------- %
The second and last part of your submission is in \texttt{solar\_system\_census.py}. You have to:
\begin{itemize}
  \item Loads the differents models from \texttt{models.[csv/yml/pickle]} and train from scratch only the best one on a training set.
  \item Visualize the performance of the different models with a bar plot showing the score of the models given their $\lambda$ value.
  \item Print the \textbf{f1 score} of all the models calculated on the test set.
  \item Visualize the target values and the predicted values of the best model on the same scatterplot. Make some effort to have a readable figure.
\end{itemize}

\info{For the second script \texttt{solar\_system\_census.py}, only a train and test set are necessary as one is simply looking to the performance.}

% ===========================(fin ex 09)         %
% ============================================== %

\newpage

% ============================================== %
% ===========================(Conclusion)        %
\chapter{Conclusion - What you have learnt}

The excercises serie is finished, well done!
Based on all the knowledges tackled today, you should be able to discuss and answer the following questions:

\begin{enumerate}
  \item Why do we use logistic hypothesis for a classfication problem rather than a linear hypothesis?
  \item What is the decision boundary?
  \item In the case we decide to use a linear hypothesis to tackle a classification problem, why the classification of some data points can be modified by considering more examples (for example, extra data points with extrem ordinate)?
  \item In a one versus all classification approach, how many logisitic regressor do we need to distinguish between N classes?
  \item Can you explain the difference between accuracy and precision? What is the type I and type II errors?
  \item What is the interest of the F1-score?
\end{enumerate}

% ===========================(Conclusion)        %
% ============================================== %

\newpage

% ================================= %
\section*{Contact}
% --------------------------------- %
You can contact 42AI association by email: contact@42ai.fr\\
You can join the association on \href{https://join.slack.com/t/42-ai/shared_invite/zt-ebccw5r7-YPkDM6xOiYRPjqJXkrKgcA}{42AI slack}
and/or posutale to \href{https://forms.gle/VAFuREWaLmaqZw2D8}{one of the association teams}.

% ================================= %
\section*{Acknowledgements}
% --------------------------------- %
The modules Python \& ML is the result of a collective work, we would like to thanks:
\begin{itemize}
  \item Maxime Choulika (cmaxime),
  \item Pierre Peigné (ppeigne),
  \item Matthieu David (mdavid).
\end{itemize}
who supervised the creation, the enhancement and this present transcription.

\begin{itemize}
  \item Amric Trudel (amric@42ai.fr)
  \item Benjamin Carlier (bcarlier@student.42.fr)
  \item Pablo Clement (pclement@student.42.fr)
\end{itemize}
for your investment for the creation and development of these modules.

\begin{itemize}
  \item Richard Blanc (riblanc@student.42.fr)
  \item Solveig Gaydon Ohl (sgaydon-@student.42.fr)
  \item Quentin Feuillade Montixi (qfeuilla@student.42.fr)
\end{itemize}
who betatest the first version of the modules of Machine Learning.
\vfill
\doclicenseThis
\end{document}
