\chapter{Exercise 06}
\extitle{Ridge Regression}
%******************************************************************************%
%                                                                              %
%                                 Interlude                                    %
%                         for Machine Learning module                          %
%                                                                              %
%******************************************************************************%

% ============================================== %
\section*{Interlude}
% ============================================== %
\subsection*{Linear Regression to the Next Level: Ridge Regression}
% ---------------------------------------------- %

Until now we only talked about L$_2$ regularization and its implication on the calculation of the loss function and gradient for both linear and logistic regression.

Now it's time to use proper terminology:
When we apply L$_2$ regularization on a linear regression model, the new model is called a \textbf{Ridge Regression} model.
Besides that brand-new name, Ridge regression is nothing more than linear regression regularized with L$_2$.

We suggest you watch this nice explanation \href{https://www.youtube.com/watch?v=Q81RR3yKn30}{very nice explanation of Ridge Regularization}.
By the way, this Youtube channel, \texttt{\textit{StatQuest}}, is very good to help you understand the gist of a lot of machine learning concepts.
You will not waste your time watching its statistics and machine learning playlists!

\newpage
\turnindir{ex06}
\exnumber{06}
\exfiles{ridge.py}
\exforbidden{sklearn}
\makeheaderfilesforbidden

% ================================= %
\section*{Objective}
% --------------------------------- %
Now it's time to implement your \texttt{MyRidge} class, similar to
 the class of the same name in \texttt{sklearn.linear\_model}.\\

% ================================= %
\section*{Instructions}
% --------------------------------- %
In the \texttt{ridge.py} file, create the following class as per the instructions given below:\\
\\
\begin{minted}[bgcolor=darcula-back,formatcom=\color{lightgrey},fontsize=\scriptsize]{python}
	class MyRidge(ParentClass):
		"""
		Description:
			My personnal ridge regression class to fit like a boss.
		"""
		def __init__(self,  thetas, alpha=0.001, max_iter=1000, lambda_=0.5):
			self.alpha = alpha
			self.max_iter = max_iter
			self.thetas = thetas
			self.lambda_ = lambda_
			... Your code here ...
	
		... other methods ...
	\end{minted}
\\
Your \texttt{MyRidge} class will have at least the following methods:
\begin{itemize}
  \item \texttt{\_\_init\_\_}, special method, similar to the one you 
  wrote in \texttt{MyLinearRegression} (module06)
  \item \texttt{get\_params\_}, which gets the parameters of the estimator
  \item \texttt{set\_params\_}, which sets the parameters of the estimator
  \item \texttt{loss\_}, which returns the loss between 2 vectors (numpy arrays)
  \item \texttt{loss\_elem\_}, which returns a vector corresponding to the squared 
  diffrence between 2 vectors (numpy arrays)  
  \item \texttt{predict\_}, which generates predictions using a linear model
  \item \texttt{gradient\_}, which calculates the vectorized regularized gradient
  \item \texttt{fit\_}, which fits Ridge regression model to a training dataset
\end{itemize}

\hint{You should consider inheritance from \texttt{MyLinearRegression}.}
\noindent{If \texttt{MyRidge} inheritates from \texttt{MyLinearRegression}, you may not 
need to reimplement the \texttt{predict\_} method.}\\
\\
The difference between the \texttt{MyRidge}'s implementations of \texttt{loss\_elem\_}, \texttt{loss\_}, \texttt{gradient\_} and 
\texttt{fit\_} and the ones in your \texttt{MyLinearRegression} class 
(implemented in module 02) is the use of a regularization term.\\
\hint{
  again, this is a good use case for decorators...
}