\chapter{Exercise 08}
\extitle{Regularized Logistic Regression}
%******************************************************************************%
%                                                                              %
%                                 Interlude                                    %
%                         for Machine Learning module                          %
%                                                                              %
%******************************************************************************%

\section*{Interlude - Fifty Shades of Linear Algebra}

In the last exercise, we implemented the loss function in two subfunctions.
It worked, but it's not very pretty.
What if we could do it all in one step, with linear algebra?   

As we did with the hypothesis, we can use a vectorized equation to improve the calculations of the loss function.

So now let's look at how squaring and averaging can be performed (more or less) in a single matrix multiplication!

$$
J(\theta) = \frac{1}{2m}\sum_{i=1}^{m}(\hat{y}^{(i)} - y^{(i)})^2
$$
$$
J(\theta) = \frac{1}{2m}\sum_{i=1}^{m}[(\hat{y}^{(i)} - y^{(i)}) (\hat{y}^{(i)} - y^{(i)})]
$$

Now, if we apply the definition of the dot product:

$$
J(\theta) = \frac{1}{2m}(\hat{y} - y) \cdot(\hat{y}- y)
$$
\newpage
\turnindir{ex08}
\exnumber{08}
\exfiles{my\_logistic\_regression.py}
\exforbidden{sklearn}
\makeheaderfilesforbidden

% ================================= %
\section*{Objective}
% --------------------------------- %
In the last exercise, you implemented a regularized version 
of the linear regression algorithm, called Ridge regression.\\
\\
Now it's time to update your logistic regression classifier as well!\\
\\
In the \texttt{scikit-learn} library, the logistic regression implementation 
offers a few regularization techniques, which can be selected using 
the parameter \texttt{penalty} (L$_2$ is default).\\
The goal of this exercise is to update your old \texttt{MyLogisticRegression} class to 
take that into account.\\

% ================================= %
\section*{Instructions}
% --------------------------------- %
In the \texttt{my\_logistic\_regression.py} file, update your \texttt{MyLogisticRegression} 
class according to the following instructions:\\
\\

\begin{minted}[bgcolor=darcula-back,formatcom=\color{lightgrey},fontsize=\scriptsize]{python}
class MyLogisticRegression():
	"""
	Description:
		My personnal logistic regression to classify things.
	"""
  supported_penalities = ['l2'] #We consider l2 penalities only. One may want to implement other penalities

	def __init__(self, theta, alpha=0.001, max_iter=1000, penality='l2', lambda_=1.0):
		# Check on type, data type, value ... if necessary
    self.alpha = alpha
		self.max_iter = max_iter
		self.theta = theta
		self.penality = penality
    self.lambda_ = lambda_ if penality in self.supported_penalities else 0
		#... Your code ...

	... other methods ...
\end{minted}
\begin{itemize}
	\item \textbf{add} a \texttt{penalty} parameter which can take the following values:\texttt{'l2'}, \texttt{'none'} (default value is \texttt{'l2'}).
  \end{itemize}
\begin{itemize}
  \item \textbf{update} the \texttt{fit\_(self, x, y)} method: 
  \begin{itemize}
    \item \texttt{if penality == 'l2'}: use a \textbf{regularized version} of the gradient descent.
    \item \texttt{if penality = 'none'}: use the \textbf{unregularized version} of the gradient descent from \texttt{module03}.
  \end{itemize}
\end{itemize}

% ================================= %
\section*{Examples}
% --------------------------------- %
\begin{minted}[bgcolor=darcula-back,formatcom=\color{lightgrey},fontsize=\scriptsize]{python}
from my_logistic_regression import MyLogisticRegression as mylogr

theta = np.array([[-2.4], [-1.5], [0.3], [-1.4], [0.7]])

# Example 1:
model1 = mylogr(theta, lambda_=5.0)

model1.penality
# Output
'l2'

model1.lambda_
# Output
5.0

# Example 2:
model2 = mylogr(theta, penality=None)

model2.penality
# Output
None

model2.lambda_
# Output
0.0

# Example 3:
model3 = mylogr(theta, penality=None, lambda_=2.0)

model3.penality
# Output
None

model3.lambda_
# Output
0.0

\end{minted}

\hint{
  this is also a great use case for decorators...
}
