\chapter{Exercise 05}
\extitle{Regularized Logistic Gradient}
%%******************************************************************************%
%                                                                              %
%                                 Interlude                                    %
%                         for Machine Learning module                          %
%                                                                              %
%******************************************************************************%

% ============================================== %
\section*{Interlude}
% ============================================== %
\subsection*{Vectorized Logistic Gradient}
% ---------------------------------------------- %

Given the previous logistic gradient formula, it's quite easy to produce a vectorized version of it.
Actually, you almost already implemented it on module02!\\
\\
As with the previous exercise, \textbf{the only thing you have to change is your hypothesis} 
in order to calculate your logistic gradient.\\

$$
\begin{matrix}
\nabla(J)_0 &  = &\cfrac{1}{m}\sum_{i=1}^{m}(h_{\theta}(x^{(i)}) - y^{(i)}) & \\
\nabla(J)_j & = &\cfrac{1}{m}\sum_{i=1}^{m}(h_{\theta}(x^{(i)}) - y^{(i)})x_{j}^{(i)} & \text{ for j = 1, ..., n}    
\end{matrix}
$$

% ============================================== %
\subsection*{Vectorized Version}
% ---------------------------------------------- %

Can be vectorized the same way you did before:

$$
\nabla(J) = \cfrac{1}{m} X'^T(h_\theta(X) - y)
$$  

%\newpage
\turnindir{ex05}
\exnumber{05}
\exfiles{reg\_logistic\_grad.py}
\exforbidden{sklearn}
\makeheaderfilesforbidden

% ================================= %
\section*{Objective}
% --------------------------------- %
You must implement the following formulas as functions for the \textbf{logistic regression hypothesis}

% ================================= %
\subsection*{Iterative}
% --------------------------------- %

$$
\nabla(J)_0 = \frac{1}{m}\sum_{i=1}^{m}(h_\theta(x^{(i)}) - y^{(i)})
$$
$$
\nabla(J)_j = \frac{1}{m}\left(\sum_{i=1}^{m}(h_\theta(x^{(i)}) - y^{(i)})x_j^{(i)} + \lambda \theta_j\right) \text{ for j = 1, ..., n}
$$
\\
Where:
\begin{itemize}
  \item $\nabla(J)_j$ is the j$^\text{th}$ component of $\nabla(J)$
  \item $\nabla(J)$ is a vector of dimension $(n + 1)$, the gradient vector
  \item $m$ is a constant, the number of training examples used
  \item $h_\theta(x^{(i)})$ is the model's prediction for the i$^\text{th}$ training example
  \item $x^{(i)}$ is the feature vector of dimension ($n$) of the i$^\text{th}$ training 
  example, found in the i$^\text{th}$ row of the $X$ matrix
  \item $X$ is a matrix of length $(m \times n)$, the design matrix
  \item $y^{(i)}$ is the i$^\text{th}$ component of the $y$ vector
  \item $y$ is a vector of dimension $m$, the vector of expected values
  \item $\lambda$ is a constant, the regularization hyperparameter
  \item $\theta_j$ is the j$^\text{th}$ parameter of the $\theta$ vector
  \item $\theta$ is a vector of dimension $(n + 1)$, the parameter vector
\end{itemize}

% ================================= %
\subsection*{Vectorized}
% --------------------------------- %
$$
\nabla(J) = \frac{1}{m} [X'^T(h_\theta(X) - y) + \lambda \theta']
$$  
\\
Where:
\begin{itemize}
  \item $\nabla(J)$ is a vector of dimension $(n + 1)$, the gradient vector
  \item $m$ is a constant, the number of training examples used
  \item $X$ is a matrix of dimensions $(m \times n)$, the design matrix
  \item $X'$ is a matrix of dimensions $(m \times (n + 1))$, the design matrix 
  onto which a column of ones is added as a first column
  \item $X'^T$ is the transpose of tha matrix, with dimensions $((n + 1) \times m)$
  \item $h_\theta(X)$ is a vector of dimension $m$, the vector of predicted values 
  \item $y$ is a vector of dimension $m$, the vector of expected values
  \item $\lambda$ is a constant, the regularization hyperparameter
  \item $\theta$ is a vector of dimension $(n + 1)$, the parameter vector
  \item $\theta'$ is a vector of dimension $(n + 1)$, constructed using the following rules: 
\end{itemize}

$$
\begin{matrix}
\theta'_0 & =  0 \\
\theta'_j & =  \theta_j & \text{ for } j = 1, \dots, n\\
\end{matrix}
$$
\newpage
% ================================= %
\section*{Instructions}
% --------------------------------- %
In the \texttt{reg\_logistic\_grad.py} file, create the following function as per the 
instructions given below:\\
\\
\begin{minted}[bgcolor=darcula-back,formatcom=\color{lightgrey},fontsize=\scriptsize]{python}
def reg_logistic_grad(y, x, theta, lambda_):
	"""Computes the regularized logistic gradient of three non-empty numpy.ndarray, with two for-loops.
	The three arrays must have compatible shapes.
	Args:
		y: has to be a numpy.ndarray, a vector of shape m * 1.
		x: has to be a numpy.ndarray, a matrix of dimesion m * n.
		theta: has to be a numpy.ndarray, a vector of shape n * 1.
		lambda_: has to be a float.
	Returns:
		A numpy.ndarray, a vector of shape n * 1, containing the results of the formula for all j.
		None if y, x, or theta are empty numpy.ndarray.
		None if y, x or theta does not share compatibles shapes.
	Raises:
		This function should not raise any Exception.
	"""
	... Your code ...

def vec_reg_logistic_grad(y, x, theta, lambda_):
	"""Computes the regularized logistic gradient of three non-empty numpy.ndarray, without 
	any for-loop. The three arrays must have compatible shapes.
	Args:
		y: has to be a numpy.ndarray, a vector of shape m * 1.
		x: has to be a numpy.ndarray, a matrix of shape m * n.
		theta: has to be a numpy.ndarray, a vector of shape n * 1.
		lambda_: has to be a float.
	Returns:
		A numpy.ndarray, a vector of shape n * 1, containing the results of the formula for all j.
		None if y, x, or theta are empty numpy.ndarray.
		None if y, x or theta does not share compatibles shapes.
	Raises:
		This function should not raise any Exception.
	"""
	... Your code ...
\end{minted}

\hint{
  this is a good use case for decorators...
}

% ================================= %
\section*{Examples}
% --------------------------------- %
\begin{minted}[bgcolor=darcula-back,formatcom=\color{lightgrey},fontsize=\scriptsize]{python}
x = np.array([[0, 2, 3, 4], 
				[2, 4, 5, 5], 
				[1, 3, 2, 7]])
y = np.array([[0], [1], [1]])
theta = np.array([[-2.4], [-1.5], [0.3], [-1.4], [0.7]])

# Example 1.1:
reg_logistic_grad(y, x, theta, 1)
# Output:
array([[-0.55711039],
		[-1.40334809],
		[-1.91756886],
		[-2.56737958],
		[-3.03924017]])

# Example 1.2:
vec_reg_logistic_grad(y, x, theta, 1)
# Output:
array([[-0.55711039],
		[-1.40334809],
		[-1.91756886],
		[-2.56737958],
		[-3.03924017]])

# Example 2.1:
reg_logistic_grad(y, x, theta, 0.5)
# Output:
array([[-0.55711039],
		[-1.15334809],
		[-1.96756886],
		[-2.33404624],
		[-3.15590684]])

# Example 2.2:
vec_reg_logistic_grad(y, x, theta, 0.5)
# Output:
array([[-0.55711039],
		[-1.15334809],
		[-1.96756886],
		[-2.33404624],
		[-3.15590684]])

# Example 3.1:
reg_logistic_grad(y, x, theta, 0.0)
# Output:
array([[-0.55711039],
		[-0.90334809],
		[-2.01756886],
		[-2.10071291],
		[-3.27257351]])

# Example 3.2:
vec_reg_logistic_grad(y, x, theta, 0.0)
# Output:
array([[-0.55711039],
		[-0.90334809],
		[-2.01756886],
		[-2.10071291],
		[-3.27257351]])
\end{minted}