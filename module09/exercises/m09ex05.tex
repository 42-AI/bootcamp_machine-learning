\chapter{Exercise 05}
\extitle{Regularized Logistic Gradient}
%%******************************************************************************%
%                                                                              %
%                                 Interlude                                    %
%                         for Machine Learning module                          %
%                                                                              %
%******************************************************************************%

\section*{Interlude - Normalization}

The values inside the $x$ vector can vary quite a lot in magnitude,
depending on the type of data you are working with.
For example, if your dataset contains distances between planets in km, the numbers will be huge.
On the other hand, if you are working with planet masses expressed as a fraction of the solar system's total mass, the numbers will be very small (between 0 and 1).
Both cases may slow down convergence in Gradient Descent (or even sometimes prevent convergence at all).
To avoid that kind of situation, normalization is a very effective way to proceed.


The idea behind this technique is straightforward: \textbf{scaling the data}.  


With normalization, you can transform your $x$ vector into a new $x'$ vector whose values range between $[-1, 1]$ more or less. Doing this allows you to see much more easily how a training example compares to the other ones:
\begin{itemize}
    \item If an $x'$ value is close to $1$, you know it's among the largest in the dataset
    \item If an $x'$ value is close to $0$, you know it's close to the median
    \item If an $x'$ value is close to $-1$, you know it's among the smallest
\end{itemize}

So with the upcoming normalization techniques, you'll be able to map your data to two different value ranges: $[0, 1]$ or $[-1, 1]$. Your algorithm will like it and thank you for it.  

%\newpage
\turnindir{ex05}
\exnumber{05}
\exfiles{reg\_logistic\_grad.py}
\exforbidden{sklearn}
\makeheaderfilesforbidden

% ================================= %
\section*{Objective}
% --------------------------------- %
You must implement the following formulas as functions for the \textbf{logistic regression hypothesis}

% ================================= %
\subsection*{Iterative}
% --------------------------------- %

$$
\nabla(J)_0 = \frac{1}{m}\sum_{i=1}^{m}(h_\theta(x^{(i)}) - y^{(i)})
$$
$$
\nabla(J)_j = \frac{1}{m}\left(\sum_{i=1}^{m}(h_\theta(x^{(i)}) - y^{(i)})x_j^{(i)} + \lambda \theta_j\right) \text{ for j = 1, ..., n}
$$
\\
Where:
\begin{itemize}
  \item $\nabla(J)_j$ is the j$^\text{th}$ component of $\nabla(J)$
  \item $\nabla(J)$ is a vector of length $(n + 1)$, the gradient vector
  \item $m$ is a constant, the number of training examples used
  \item $h_\theta(x^{(i)})$ is the model's prediction for the i$^\text{th}$ training example
  \item $x^{(i)}$ is the feature vector of dimension ($n$) of the i$^\text{th}$ training 
  example, found in the i$^\text{th}$ row of the $X$ matrix
  \item $X$ is a matrix of length $(m \times n)$, the design matrix
  \item $y^{(i)}$ is the i$^\text{th}$ component of the $y$ vector
  \item $y$ is a vector of length $m$, the vector of expected values
  \item $\lambda$ is a constant, the regularization hyperparameter
  \item $\theta_j$ is the j$^\text{th}$ parameter of the $\theta$ vector
  \item $\theta$ is a vector of length $(n + 1)$, the parameter vector
\end{itemize}

% ================================= %
\subsection*{Vectorized}
% --------------------------------- %
$$
\nabla(J) = \frac{1}{m} [X'^T(h_\theta(X) - y) + \lambda \theta']
$$  
\\
Where:
\begin{itemize}
  \item $\nabla(J)$ is a vector of dimension $(n + 1)$, the gradient vector
  \item $m$ is a constant, the number of training examples used
  \item $X$ is a matrix of dimensions $(m \times n)$, the design matrix
  \item $X'$ is a matrix of dimensions $(m \times (n + 1))$, the design matrix 
  onto which a column of ones is added as a first column
  \item $X'^T$ is the transpose of tha matrix, with dimensions $((n + 1) \times m)$
  \item $h_\theta(X)$ is a vector of length $m$, the vector of predicted values 
  \item $y$ is a vector of length $m$, the vector of expected values
  \item $\lambda$ is a constant, the regularization hyperparameter
  \item $\theta$ is a vector of length $(n + 1)$, the parameter vector
  \item $\theta'$ is a vector of length $(n + 1)$, constructed using the following rules: 
\end{itemize}

$$
\begin{matrix}
\theta'_0 & =  0 \\
\theta'_j & =  \theta_j & \text{ for } j = 1, \dots, n\\
\end{matrix}
$$
\newpage
% ================================= %
\section*{Instructions}
% --------------------------------- %
In the \texttt{reg\_logistic\_grad.py} file, create the following function as per the 
instructions given below:\\
\\
\begin{minted}[bgcolor=darcula-back,formatcom=\color{lightgrey},fontsize=\scriptsize]{python}
def reg_logistic_grad(y, x, theta, lambda_):
	"""Computes the regularized logistic gradient of three non-empty numpy.ndarray, with two for-loops.
	The three arrays must have compatible shapes.
	Args:
		y: has to be a numpy.ndarray, a vector of shape m * 1.
		x: has to be a numpy.ndarray, a matrix of dimesion m * n.
		theta: has to be a numpy.ndarray, a vector of shape n * 1.
		lambda_: has to be a float.
	Returns:
		A numpy.ndarray, a vector of shape n * 1, containing the results of the formula for all j.
		None if y, x, or theta are empty numpy.ndarray.
		None if y, x or theta does not share compatibles shapes.
	Raises:
		This function should not raise any Exception.
	"""
	... Your code ...

def vec_reg_logistic_grad(y, x, theta, lambda_):
	"""Computes the regularized logistic gradient of three non-empty numpy.ndarray, without 
	any for-loop. The three arrays must have compatible shapes.
	Args:
		y: has to be a numpy.ndarray, a vector of shape m * 1.
		x: has to be a numpy.ndarray, a matrix of shape m * n.
		theta: has to be a numpy.ndarray, a vector of shape n * 1.
		lambda_: has to be a float.
	Returns:
		A numpy.ndarray, a vector of shape n * 1, containing the results of the formula for all j.
		None if y, x, or theta are empty numpy.ndarray.
		None if y, x or theta does not share compatibles shapes.
	Raises:
		This function should not raise any Exception.
	"""
	... Your code ...
\end{minted}

\hint{
  this is a good use case for decorators...
}

% ================================= %
\section*{Examples}
% --------------------------------- %
\begin{minted}[bgcolor=darcula-back,formatcom=\color{lightgrey},fontsize=\scriptsize]{python}
x = np.array([[0, 2, 3, 4], 
				[2, 4, 5, 5], 
				[1, 3, 2, 7]])
y = np.array([[0], [1], [1]])
theta = np.array([[-2.4], [-1.5], [0.3], [-1.4], [0.7]])

# Example 1.1:
reg_logistic_grad(y, x, theta, 1)
# Output:
array([[-0.55711039],
		[-1.40334809],
		[-1.91756886],
		[-2.56737958],
		[-3.03924017]])

# Example 1.2:
vec_reg_logistic_grad(y, x, theta, 1)
# Output:
array([[-0.55711039],
		[-1.40334809],
		[-1.91756886],
		[-2.56737958],
		[-3.03924017]])

# Example 2.1:
reg_logistic_grad(y, x, theta, 0.5)
# Output:
array([[-0.55711039],
		[-1.15334809],
		[-1.96756886],
		[-2.33404624],
		[-3.15590684]])

# Example 2.2:
vec_reg_logistic_grad(y, x, theta, 0.5)
# Output:
array([[-0.55711039],
		[-1.15334809],
		[-1.96756886],
		[-2.33404624],
		[-3.15590684]])

# Example 3.1:
reg_logistic_grad(y, x, theta, 0.0)
# Output:
array([[-0.55711039],
		[-0.90334809],
		[-2.01756886],
		[-2.10071291],
		[-3.27257351]])

# Example 3.2:
vec_reg_logistic_grad(y, x, theta, 0.0)
# Output:
array([[-0.55711039],
		[-0.90334809],
		[-2.01756886],
		[-2.10071291],
		[-3.27257351]])
\end{minted}